\documentclass[pmath450]{subfiles}

%% ========================================================
%% document

\begin{document}

    \section{Measurable Functions}

    \subsection{Non-measurable Sets}

    As a result of the effort made in the preceding section, we now have in hand
    \begin{enumerate}
        \item a collection $\mM$ of \textit{Lebesgue measurable} subsets of $\R$; and
        \item a length-measuring map $\lambda:\mM\to\left[ 0,\infty \right]$, called \textit{Lebesgue measure}.
    \end{enumerate}
    Moreover, we checked that $\mM$ is a $\sigma$-algebra and that $\lambda$ is a positive measure. In connection to these general notions, here is a bit of additional terminology which is part of the measure and integration theory.

    \begin{definition}{\textbf{Measurable Space}, \textbf{Measure Space}}
        Let $X$ be a nonempty set.
        \begin{enumerate}
            \item When $\mA$ is a $\sigma$-algebra on $X$, we call the pair $\left( X,\mA \right)$ a \emph{measurable space}.
            \item If $\mu:\mA\to\left[ 0,\infty \right]$ is a positive measure in addition, then we say $\left( X,\mA,\mu \right)$ is a \emph{measure space}.
        \end{enumerate}
    \end{definition}

    \np In the terms set by Def'n 2.1, we have that $\left( \R,\mM \right)$ is a measurable space and that $\left( \R,\mM,\lambda \right)$ is a measure space.

    The notion of measurable space offers a convenient framework for talking about \textit{measurable functions}, which is pretty much the topic of this section. Then the notion of measure space offers a convenient framework for discussing about \textit{integrable functions}, which is what we will do when we are done with setting the basics for measurable functions.

    \np Knowing that we are dealing with a $\sigma$-algebra is, without a doubt, a rather comfortable thing for what we do. Indeed, since all the open sets are min $\mM$, we know that we can start from open sets and we can do any kind of set operations we want, involving \textit{countably many} sets at a time, and the set $M$ resulting out of these operations will still be in $\mM$, so we can fearlessly talk about the length (i.e. Lebesgue measure) $\lambda\left( M \right)$. Thus $\lambda\left( M \right)$ can surely be considered when $M$ is a closed set, then also for sets that are of type $G_{\delta}$, of type $F_{\sigma}$, and so on.

    In connection to the above, here are the two questions that we consider in this and next subsections.

    \begin{itemize}
        \item \textit{Question 1. Couldn't it actually be that \textbf{every} subset of $\R$ is Lebesgue measurable?}
    \end{itemize} 
    The answer will be negative. But, exactly because the $\sigma$-algebra $\mM$ is so comfortable to deal with, it's not that easy to produce an example of non-measurable set. 
    \begin{itemize}
        \item \textit{Question 2. Is $\mM$ the \textbf{minimal} $\sigma$-algebra of subsets of $\R$ that we get by starting from open sets?}
    \end{itemize} 
    Once again, the answer is negative. There is a slightly smaller $\sigma$-algebra, called the \textit{Borel $\sigma$-algebra}, which has this minimality property. This is discussed in the next subsection.

    \np The next lemma shows the blueprint we will use in order to produce an example of non-measurable set. 

    \begin{definition}{\textbf{Translation} of a Set}
        Let $S\subseteq\R, x\in\R$. We define the \emph{translation} of $S$ by $x$, denoted as $S+x$, by
        \begin{equation*}
            S+x = \left\lbrace s+x: s\in S \right\rbrace.
        \end{equation*}
    \end{definition}

    \begin{exercise}{}
        Let $M\in\mM$. Then for all $x\in\R$, show that $M+x\in\mM$ and that $\lambda\left( M+x \right)=\lambda\left( M \right)$.
    \end{exercise}

    \placeqed[Assignment]

    \begin{lemma}{A Criterion for Non-measurability}
        Let $S\subseteq\left( 0,1 \right)$. If there exists a countable set $\left\lbrace q_n \right\rbrace^{\infty}_{n=1}$ of numbers in $\left( -1,1 \right)$ such that
        \begin{equation}
            \forall m,n\in\N\left[ m\neq n\implies \left( S+q_m \right)\cap \left( S+q_n \right)=\emptyset \right],
        \end{equation}
        and that
        \begin{equation}
            \bigcup^{\infty}_{n=1}\left( S+q_n \right)\supseteq\left( 0,1 \right),
        \end{equation}
        then $S\notin\mM$.
    \end{lemma}

    \begin{proof}
        Let us assume, for contradiction, that $S\notin\mM$, and let us denote $\lambda\left( S \right)=c\in\left[ 0,\infty \right]$. Since it is given that, we can actually be sure that
        \begin{equation*}
            c = \lambda\left( S \right)\leq\lambda\left( \left( 0,1 \right) \right)=1,
        \end{equation*}
        hence that $c\in\left[ 0,1 \right]$.

        For every $n\in\N$, denote $M_n=S+q_n$. Sicne we assumed that $S\in\mM$, it follows that every $M_n$ is in $\mM$, with $\lambda\left( M_n \right)=\lambda\left( S \right)=c$, in view of Exercise 2.1. Another observation about the sets $M-n$ is that we have
        \begin{equation}
            \forall n\in\N\left[ M_n\subseteq\left( -1,2 \right) \right].
        \end{equation}
        This is because $S\subseteq\left( 0,1 \right)$ and each $q_n$ is in $\left( -1,1 \right)$.

        Now consider the set $M=\bigcup^{\infty}_{n=1}M_n$. Then $m\in\mM$, because the $M_n$'s are from $\mM$, and $\mM$ is a $\sigma$-algebra. From [2.3] it follows that $M\subseteq\left( -1,2 \right)$ and, on the other hand, the hypothesis [2.2] of the lemam says precisely that $M\supseteq\left( 0,1 \right)$. This yields some estimates on the Lebesgue measure of $M$: sicne $\left( 0,1 \right)\subseteq M\subseteq\left( -1,2 \right)$, we have $\lambda\left( \left( 0,1 \right) \right)\leq\lambda\left( M \right)\leq\lambda\left( \left( -1,2 \right) \right)$. We thus find that
        \begin{equation}
            1\leq\lambda\left( M \right)\leq 3.
        \end{equation}

        We are very close to a contradiction. Indeed, recall that $M=\bigcup^{\infty}_{n=1}M_n$, where the hypothesis [2.1] tells us that $M_m\cap M_n=\emptyset$ for all distinct $n,m\in\N$. So then by the $\sigma$-additivity of the Lebesgue measure implies that
        \begin{equation*}
            \lambda\left( M \right)= \sum^{\infty}_{n=1}\lambda\left( M_n \right) = \lim_{p\to\infty}\sum^{p}_{n=1}\lambda\left( M_n \right)=\lim_{p\to\infty}pc =
            \begin{cases} 
                0 &\text{if $c=0$}\\
                \infty &\text{if $0<c\leq 1$}
            \end{cases}.
        \end{equation*}
        Either way, this contradicts with the estimate [2.4]. 

        Thus we conclude that $S$ is not Lebesgue measurable.
    \end{proof}

    \np Now, of course, we have to ask: \textit{how do we find a set $S$ and a set of numbers $\left\lbrace q_n \right\rbrace^{\infty}_{n=1}$ which satisfy the hypothesis of Lemma 2.1?} It turns out that we can arrange that to happen in connection to an equivalence relation, which we define on $\left( 0,1 \right)$, as follows. For all $x,y\in\left( 0,1 \right)$, write
    \begin{equation*}
        x\sim y \iff x-y\in\Q.
    \end{equation*}
    Verifying $\sim$ is an equivalence relation amounts to the fact that $\left( \Q,+ \right)$ is an additive group.

    Now, having an equivalence relation $\sim$ \textit{splits} $\left( 0,1 \right)$ into a disjoint union of equivalence classes with respect to $\sim$:
    \begin{equation*}
        \left( 0,1 \right) = \bigcup^{}_{i\in I}E_i
    \end{equation*}
    with $E_i\cap E_j=\emptyset$ for all distinct $i,j\in I$. 

    From every equivalence class $E_i$, we \textit{choose} a point, say $s_i\in E_i$, by the axiom of choice. Let us define
    \begin{equation*}
        S = \left\lbrace s_i: i\in I \right\rbrace,
    \end{equation*}
    which indeed is a subset of $\left( 0,1 \right)$.

    \begin{prop}{}
        Consider $S$ defiend as above and let $\left( q_{n} \right)^{\infty}_{n=1}$ be an enumeration of the rational numbers in $\left( -1,1 \right)$. Then $S$ and $\left\lbrace q_n \right\rbrace^{\infty}_{n=1}$ satisfy the hypotheses of Lemma 2.1. 
    \end{prop}

    \begin{proof}
        We must check that the conditions [2.1] and [2.2] from Lemma 2.1 are satisfied.

        For [2.1], we pick some distinct indices $n,m\in\N$, for which we verify that $\left( S+q_m \right)\cap\left( S+q_n \right)=\emptyset$.

        Assume for contradiction that there is $x\in\left( S+q_m \right)\cap\left( S+q_n \right)$. Then there are $s',s''\in S$ such that $x=s'+q_m=s''+q_n$. In view of the definition of $S$, these numbers $s'$ and $s''$ have to be of the form $s_i$ and $s_j$ for some $i,j\in I$, respectively. We thus have equalities
        \begin{equation}
            x = s_i+q_m = s_j+q_n.
        \end{equation}
        But from [2.5], combined how the elements of $S$ were selected to represent the equivalence classes of $\sim$, it follows that
        \begin{equation*}
            s_i-s_j = q_n-q_m \in\Q\implies s_i\sim s_j\implies i=j.
        \end{equation*}
        So then $s_i=s_j$, and from the equality $s_i+q_m=s_j+q_n$ we infur that $q_m=q_n$, which is a contradiction.

        We thus conclude that $\left( S+q_m \right)\cap\left( S+q_n \right)=\emptyset$.

        For [2.2], inview of what $\bigcup^{\infty}_{n=1}\left( S+q_n \right)$ means, the verification that needs to be done here is this: for all $x\in\left( 0,1 \right)$, there exists $i\in I$ such that $x\in E_i$. In the equivalence class $E_i$ we have selected representative $s_i$; so then, it is the case that $x-s_i\in\Q$. We also observe that, since $x,s_i\in\left( 0,1 \right)$, we have $-1<x-s_i<1$. Thus $x-s_i\in\Q\cap\left( -1,1 \right)$, and there there exists an $n\in\N$ such that $x-s_i=q_N$. This $n$ is exactly what we need, since $x-s_i=q_n$ implies $x=s_i+q_n\in S+q_n$.
    \end{proof}

    \np As a consequence to Proposition 2.2, $S$ is not Lebesgue measurable.

    \subsection{Borel $\sigma$-algebras}
    
    In this subsection, we discuss a more general method to produce $\sigma$-algebras: start with an arbitrary collection $\mC$ of subsets of a set $X$, and consider \textit{the smallest possible $\sigma$-algebra which contains $\mC$.} We then use this to construct the said Borel $\sigma$-algebra.

    The precise description of what the above means appears in Proposition 2.4. Before stating that proposition, we record, in the next lemma, a simple observation coming out directly from the definition of a $\sigma$-algebra. 

    \begin{lemma}{}
        Let $X$ be a nonempty set and let $\left\lbrace \mA_i \right\rbrace^{}_{i\in I}$ be a family of $\sigma$-algebras of subsets of $X$. Denote $\mA=\bigcap^{}_{i\in I}\mA_i$. Then $\mA$ is a $\sigma$-algebra.
    \end{lemma}

    \begin{proof}
        We verify three things.
        \begin{enumerate}
            \item Since $\emptyset,X\in\mA_i$ for all $i\in I$, $\emptyset,X\in\mA$.
            \item Let $N,M\in\mA$. Then for all $i\in I$, $N,M\in\mA_i$, so that $N\setminus M\in\mA_i$.
            \item Let $\mC$ be a countable collection of sets in $\mA$. Then $\mC\subseteq\mA_i$ for all $i\in I$, so that $\bigcup^{}_{}\mC\in\mA_i$. Thus $\bigcup^{}_{}\mC\in\mA$.
        \end{enumerate}
    \end{proof}

    \begin{prop}{}
        Let $X$ be a nonempty set and let $\mC$ be a collction of subsets of $X$. Then there exists a unique collection $\mA_0$ of subsets of $X$ such that
        \begin{enumerate}
            \item $\mA_0$ is a $\sigma$-algebra of subsets of $X$ containing $\mC$; and
            \item every $\sigma$-algebra containing $\mC$ contains $\mA_0$ also.
        \end{enumerate}
    \end{prop}

    \begin{proof}
        We first show that such $\mA_0$ exists.

        Let $\left\lbrace \mA_i \right\rbrace^{}_{i\in I}$ be the family of all the $\sigma$-algebras of subsets of $X$ which contain the given $\mC$.\footnotemark[1] We put
        \begin{equation*}
            \mA_0 = \bigcap^{}_{i\in I}\mA_i,
        \end{equation*}
        which is a $\sigma$-algebra of subset of $X$ by Lemma 2.3 and contains $\mC$ since every $\mA_i$ contains $\mC$. Hence $\mA_0$ satisfies (a).

        On the other hand, let $\mA$ be a $\sigma$-algebra of subsets of $X$ containing $\mC$. Then $\mA=\mA_j$ for some $j\in I$, so that
        \begin{equation*}
            \mA = \mA_j \supseteq\bigcap^{}_{i\in I}\mA_i = \mA_0.
        \end{equation*}
        Thus $\mA_0$ satisfies (b) as well.

        For the uniqueness, let $\mA_0$ be defined as above. Let $\mA_0'$ be a $\sigma$-algebra of subsets of $X$ which also satisfies (a), (b). We have to prove that $\mA_0'=\mA_0$.

        But (b) applies to both $\mA_0,\mA_0'$, so that $\mA_0\subseteq\mA_0'$ and $\mA_0'\subseteq\mA_0$. Thus we conclude $\mA_0'=\mA_0$, as required.

        \noindent
        \begin{minipage}{\textwidth}
            \footnotetext[1]{We note that $2^X$, the power set of $X$, is a $\sigma$-algebra of subsets of $X$. Hence such family is nonempty.}
        \end{minipage}
    \end{proof}

    \begin{definition}{$\sigma$-algebra \textbf{Generated} by a Collection}
        Let $X$ be a nonempty set and let $\mC$ be a collection of subsets of $X$. Then the $\sigma$-algebra $\mA_0$ found in Proposition 2.4 is called the $\sigma$-algebra of subsets of $X$ \emph{generated} by $\mC$, which we shall denote as $\sigalg\left( \mC \right)$.
    \end{definition}

    \np The argument used to prove Proposition 2.4 is qutie far from being constructive. Nevertheless, we will see that useful things can be proved about $\sigalg\left( \mC \right)$, by just exploiting the conditions in Proposition 2.4. An easy example of how this goes is provided by the next exercise.

    \begin{exercise}{}
        Let $X$ be a nonempty set and let $\mC_1$ and $\mC_2$ be collections of subsets of $X$ such that $\mC_1\subseteq\mC_2$. Prove that $\sigalg\left( \mC_1 \right)\subseteq\sigalg\left( \mC_2 \right)$.
    \end{exercise}

    \begin{proof}
        It suffices to note that $\sigalg\left( \mC_2 \right)\supseteq\mC_2\supseteq\mC_1$, so $\sigalg\left( \mC_2 \right)$ contains $\sigalg\left( \mC_1 \right)$, the smallest $\sigma$-algebra containing $\mC_1$.
    \end{proof}

    \begin{definition}{\textbf{Borel $\sigma$-algebra}}
        We call $\sigalg\left( \mT \right)$, the $\sigma$-algebra generated by the open subsets of $\R$, the \emph{Borel $\sigma$-algebra} of $\R$, denoted as $\mB$.
    \end{definition}

    \np $\mB$ is the smallest $\sigma$-algebra of subsets of $\R$ which contains all the open sets. In order to prove things about Borel sets, we will typically just fall back on the description of $\mB$ via the properties (a), (b) from Proposition 2.4.

    In connection to the above, observe that it is possible (and sometimes convenient) to approach the Borel $\sigma$-algebra $\mB$ by using a collection of generators different from $\mT$. Here is an instructive exercise on these lines, which says that \textit{one can generate $\mB$ by using compact sets}.

    \begin{exercise}{}
        Recall that we denote $\mK$ to be the collection of compact subsets of $\R$. Prove that $\sigalg\left( \mK \right)=\mB$.
    \end{exercise}

    \placeqed[tl;dr]

    \np One thing which we clearly have at this point is that $\mB\subseteq\mM$. This is because $\mM$ is a $\sigma$-algebra which contains $\mT$, while $\mB$ is the \textit{minimal} $\sigma$-algebra which contains $\mT$. In order to determine that $\mM\neq\mB$, we need to work a bit more.

    For the time being, we just record the fact that, since $\mB\subseteq\mM$, we can in any case restrict the Lebesgue measure $\lambda$ to $\mB$, which will clearly give us a positive measure on $\mB$. We are thus getting a measure space $\left( \mR,\mB,\lambda \right)$, which will turn out to be of great interest for us.

    \np When we look at how $\mM$ and $\lambda$ appeared in our considerations, we see that they were \textit{packed together}. For instance, look at where $\mM_{\bdd}$ was introduced. We see that we were using there $\lambda_{\op}$ and $\lambda_{\cp}$, the preliminary instances of $\lambda$ we had developed for open and for compact sets.

    So the definition of $\mM$ is intimately related to the notion of length. The definition of $\mB$ is not like that. In order to define $\mB$ we only need to know what are the open subsets of $\R$. This immediately prompts the thought that we can define a Borel $\sigma$-algebra associated to any topological space. Later in the course it will be useful to play with Borel subsets of such metric spaces $X$, for instance make $X$ be the unit circle in the complex plane.

    \begin{definition}{\textbf{Borel $\sigma$-algebra} of a Topological Space}
        Let $\left( X,\mT \right)$ be a topological space. We call $\sigalg\left( \mT \right)$ the \emph{Borel $\sigma$-algebra} on $X$, which is denoted as $\mB_X$.\footnotemark[1] The elements of $\mB_X$ are called the \emph{Borel sets}.

        \noindent
        \begin{minipage}{\textwidth}
            \footnotetext[1]{We should really write $\mB_{\left( X,\mT \right)}$ instead, since the definition heavily depends on $\mT$. However, often times $\mT$ is well-understood, so we shall write $\mB_X$ for convenience.}
        \end{minipage}
    \end{definition}

    \np As before, we are writing $\mB$ to mean $\mB_{\R}$.

    \np Suppose that our metric space is $\left[ 0,1 \right]$, endowed with the usual distance $d\left( x,y \right)=\left| x-y \right|$ for all $x,y\in\left[ 0,1 \right]$. This is a compact metric space. We can talk about the collection $\mT_{\left[ 0,1 \right]}$ of subsets of $\left[ 0,1 \right]$ relatively open in $\left[ 0,1 \right]$, and we can then consider the corresponding Borel $\sigma$-algebra:
    \begin{equation*}
        \mB_{\left[ 0,1 \right]} = \sigalg\left( \mT_{\left[ 0,1 \right]} \right).
    \end{equation*}
    Here is a natural question that pops up in connection to this. Namely, since $\left[ 0,1 \right]\subseteq\R$ and we already have $\mB_{\R}$, \textit{why don't we actually work with the collection $\hat{\mB}_{\left[ 0,1 \right]}$ of subsets of $\left[ 0,1 \right]$ defiend by}
    \begin{equation}
        \hat{\mB}_{\left[ 0,1 \right]} = \left\lbrace M\in\mB_{\R}: M\subseteq\left[ 0,1 \right] \right\rbrace.
    \end{equation}
    The good news is that $\hat{\mB}_{\left[ 0,1 \right]}$ coincides with $\mB_{\left[ 0,1 \right]}$ (which is a special case of the situation considered in Proposition 2.6). This means [2.6] can be used as a description of the Borel $\sigma$-algebra of $\left[ 0,1 \right]$.

    \np Here is an example which lives in the complex plane, and will be important in the final part of this course: consider the unit circle
    \begin{equation*}
        \T = \left\lbrace z\in\CC: \left| z \right|=1 \right\rbrace
    \end{equation*}
    on the complex plane, endowed with the usual distance between complex numbers, $d\left( z,w \right)=\left| z-w \right|$ for all $z,w\in\T$. Then $\left( \T,d \right)$ is a compact metric space, which has its collection $\mT_{\T}$ of open sets, and the corresponding Borel $\sigma$-algebra $\mB_{\T}=\sigalg\left( \mT_{\T} \right)$.

    \begin{prop}{}
        Let $\left( X,\mA \right)$ be a measurable space. Let $X_0$ be a nonempty set in $\mA$ and let
        \begin{equation*}
            \mA_0 = \left\lbrace A\in\mA: A\subseteq X_0 \right\rbrace.
        \end{equation*}
        Then $\mA_0$ is a $\sigma$-algebra of subsets of $X_0$.
    \end{prop}

    \placeqed[Quite Clear!]

    \begin{definition}{\textbf{Restriction} of a $\sigma$-algebra}
        Consider the setting of Proposition 2.5. We call $\left( X_0,\mA_0 \right)$ the \emph{restriction} of $\left( X,\mA \right)$ to $X_0$.
    \end{definition}

    \begin{prop}{}
        Let $\left( X,d \right)$ be a metric space. Consider the collection of open sets $\mT_X$ of $\left( X,d \right)$ and the Borel $\sigma$-algebra $\mB_X$.

        Let $X_0\in\mB_X$ be nonempty and let $\left( X_0,\mA_0 \right)$ be the restriction of the measurable space $\left( X,\mB_X \right)$ to $X_0$. On the other hand, let $d_0=d|_{X_0}$, the restriction of the metric on $X$, which is a metric on $X_0$. Consider the Borel $\sigma$-algebra $\mB_{X_0}$.

        Then $\mB_{x_0} = \mA_0$.
    \end{prop}

    \placeqed[tl;dr]

    \np We now return to the problem showing $\mB\neq\mM$. That is, there exists $M\subseteq\R$ that is Lebesgue measurable but is not a Borel set. The idea is that we will find such $M$ to be \textit{negligible}. Recall the definition of such sets.

    \begin{recall}{\textbf{Negligible} Set}
        We say $N\subseteq\R$ is \emph{negligible} if $N$ is Lebesgue measurable with measure $0$.
    \end{recall}

    \np From assignments, we have an alternative description of negligible sets, which is recorded in the next proposition.

    \begin{prop}{}
        Let $N\subseteq\R$. The following are equivalent.
        \begin{enumerate}
            \item $N$ is negligible.
            \item For all $\epsilon>0$, there exists open $G\subseteq\R$ that contains $N$ with $\lambda_{\op}\left( G \right)<\epsilon$.
        \end{enumerate}
    \end{prop}

    \placeqed[Assignment!]

    \clearpage
    \np A benefit of the characterization appearing in Proposition 2.7 is that it has the following immediate consequence.

    \begin{cor}{}
        Every subset of a negligible set is negligible.
    \end{cor}	

    \placeqed[Proof by Inspection!]

    \np Let us put to use the work on the tenary Canor set on assignments.

    \begin{cor}{}
        Let $C$ be the tenary Cantor set. Then every subset of $C$ is measurable.
    \end{cor}	

    \begin{proof}
        We saw that $C$ is negligible, which means every subset of $C$ is negligible (and in particular, measurable) by Corollary 2.7.1.
    \end{proof}

    \begin{exercise}{}
        Let $\mN$ be a countable collection of negligible subsets of $\R$. Prove that $\bigcup\mN$ is also negligible.
    \end{exercise}

    \begin{proof}
        We may assume $\mN$ is countably infinite, so denote $\mN = \left\lbrace N_n \right\rbrace^{\infty}_{n=1}$. Suppose $\epsilon>0$, for which we verify that there is open $G\subseteq\R$ containing $\bigcup\mN$ with $\lambda_{\op}\left( G \right)<\epsilon$.

        For all $n\in\N$, let $G_n$ be an open set containing $N_n$ with $\lambda_{\op}\left( G_n \right) < \frac{\epsilon}{2^{n+1}}$. Such $G_n$ exists by Proposition 2.7 since $N_n$ is negligible. Now define
        \begin{equation*}
            G = \bigcup^{\infty}_{n=1}G_n,
        \end{equation*}
        which is open as a union of open sets. Moreover, $G$ contains $N=\bigcup^{\infty}_{n=1}N_n$ since each $G_n$ contains $N_n$, with
        \begin{equation*}
            \lambda\left( G \right) \leq \sum^{\infty}_{n=1}\lambda\left( G_n \right) \leq \sum^{\infty}_{n=1} \frac{\epsilon}{2^{n+1}} = \frac{\epsilon}{2} \sum^{\infty}_{n=1}\frac{1}{2^n} = \frac{\epsilon}{2}<\epsilon,
        \end{equation*}
        as required.
    \end{proof}

    \np Our goal of proving the existence of measurable non-Borel sets will then achieved with the following proposition.

    \begin{prop}{}
        Let $C$ be the tenary Cantor set. Then there is a subset of $C$ which is not Borel.
    \end{prop}

    \placeqed[Proof Postponed]

    \np The proof of Proposition 2.8 goes a bit outside the tools that we have available at this moment. We list below three facts that we combine in order to get this proof.

    \begin{fact}{}
        \vspace{-11pt}
        \begin{enumerate}
            \item Let $K\subseteq\left[ 0,1 \right]$ be a compact non-negligible set. Then there is a non-measurable subset of $K$.
            \item Let $K_1,K_2$ be nonempty compact subsets of $\R$ and let $\phi:K_1\to K_2$ be a homeomorphism. Consider the Borel $\sigma$-algebras $\mB_{K_1}$ and $\mB_{K_2}$ associated to $K_1$ and $K_2$, respectively. Then $\phi\left( \mB_{K_1} \right)\subseteq\mB_{K_2}$.
            \item Let $C$ be the tenary Cantor set. Then there exists a continuous injection $\phi:C\to\left[ 0,1 \right]$ such that the compact set $K=\phi\left( C \right)$ has a nonzero measure.
        \end{enumerate}
    \end{fact}

    \np Here are some comments concerning these facts.
    
    (a) is an (not-so-immediate) upgrade of the argument shown when we constructed a non-measurable set.

    (b) and (c) refer to functions between measurable spaces, and are best treated after we discuss a bit about such functions. In fact the function $\phi$ of (c) will be very interesting to look at. It must \textit{stretch} distances between points quite widely if it succeeds to start with the negligible compact set $C$ and map it onto a non-negligible set $\phi\left( C \right)$.

    We now demonstrate how Proposition 2.8 can be proved if we assume Fact 2.9. 

    \begin{boxyproof}{Proof of Proposition 2.8}
        Let us assume that every subset of $C$ is a Borel set for contradiction. Then the restriction of the measurable space $\left( \R,\mB \right)$ to $C$ is $\left( C,2^C \right)$. Thus when we view $C$ as a compact metric space, its Borel $\sigma$-algebra $\mB_C$ comes out as $\mB_C=2^C$.

        Now consider the homeomorphism $\phi:C\to K$ that is provided to us by (c) of Fact 2.9.\footnotemark[1] We recall that $\lambda\left( K \right)>0$.

        \begin{itemize}
            \item \textit{Claim 1. $\mB_K=2^K$.}

                \begin{subproof}
                    What we have to prove is that every subset of $T$ is in $\mB_K$ (note that $\mB_k\subseteq 2^K$ is immediate). To that end, fix $T\subseteq K$.

                    Let $S=\phi^{-1}\left( T \right)\subseteq C$. Since $\mB_C=2^{C}$, we have that $S\in\mB_C$. Then (b) of Fact 2.9 implies $\phi\left( S \right)\in\mB_K$. But $\phi\left( S \right)=T$, so that $T\in\mB_K$.
                \end{subproof}
        \end{itemize} 
        But by using Proposition 2.6, Claim 1 can be read as saying that \textit{every subset of $K$ belongs to the Borel $\sigma$-algebra $\mB$}. Since $K$ is a compact subset of $\left[ 0,1 \right]$ with a nonzero measure, we reached a contradiction with (a) of Fact 2.9.

        \noindent
        \begin{minipage}{\textwidth}
            \footnotetext[1]{Note that (c) of Fact 2.9 gives only a continuous \textit{injection} $\phi:C\to\left[ 0,1 \right]$ such that $\phi\left( C \right)$ has a nonzero measure. By restricting its codomain to $\phi\left( C \right)$, we obtain a homeomorphism, so that we can apply (b) of Fact 2.9.} 
        \end{minipage}
    \end{boxyproof}

    \np If one looks into the literature, one will also find another approach to proving Proposition 2.8, which is entirely set-theoretic: one can compare the infinite cardinalities of $\mB_C$ and $2^C$, and arrive to the conclusion that $\left| \mB_C \right|<\left| 2^C \right|$. Hence in particular $\mB_C\subset 2^C$, as Proposition 2.8 is stating.

    \np Finnaly, here is a little proposition driving the idea that, while $\mB$ is a proper subcollection of $\mM$, the difference between two really is about how they handle negligible sets. From the point of view of integration theory, this is not such a crucial difference, since negligible sets can most of the time be ignored in the process of integration.

    \begin{prop}{}
        Let $M$ be a measurable set. Then there is $B\subseteq M$ such that $B$ is a Borel set and $N=M\setminus B$ is negligible.
    \end{prop}

    \begin{proof}
        When $M$ is bounded, from assignment we know that $M = F\cup N$ for some $F_{\sigma}$ set $F$ and some negligible $N$. Since every $F_{\sigma}$ set is Borel (as a countable union of closed sets), we reach the desired conclusion by taking $B=F$.

        Suppose $M$ is not bounded. For all $n\in\N$, let $M_n=M\cap\left( -n,n \right)$, which is a bounded measurable set. Then the observation in the preceding paragraph can be applied to every $M_n$; it gives a set $B_n\subseteq M_n$ such that $B_n\in\mB$ and such that $N_n=M_n\setminus B_n$ is negligible. We then take
        \begin{equation*}
            B = \bigcup^{\infty}_{n=1}B_n
        \end{equation*}
        and
        \begin{equation*}
            N = M\setminus B.
        \end{equation*}
        Then a direct inspection gives that $B\subseteq\bigcup^{\infty}_{n=1}M_n=M$ and that $N\subseteq\bigcup^{\infty}_{n=1}\left( M_n\setminus B_n \right)=\bigcup^{\infty}_{n=1}N_n$. Hence $B\in\mB$ since $\mB$ is closed under countable intersections and $N$ is negligible as a countable union of negligible sets (Exercise 2.4). Thus $B$ and $N$ achieve a decomposition of $M$ as required by the proposition.
    \end{proof}

    \subsection{Measurable Functions}
    
    \np For this subsection, we will momentarily forget about $\lambda$, and just look at the measurable space $\left( \R,\mB \right)$. We are interested in functions $f:\R\to\R$ which are \textit{measurable} with respect to $\mB$, in a sense that we will define today. It turns out to come at no cost if instead of functions $f:\R\to\R$ we want to write our definitions and basic propositions in reference to functions $f:X\to Y$ where $\left( X,\mfX \right)$ and $\left( Y,\mfY \right)$ are arbitrary measurable spaces.

    \np For convenience, let $\left( X,\mfX \right), \left( Y,\mfY \right), \left( Z,\mfZ \right)$ be measurable spaces throughout this subsection unless otherwise stated.

    \begin{definition}{\textbf{Measurable} Function}
        Let $f:X\to Y$. We say $f$ is \emph{$\mfX /\mfY$-measurable} if $f^{-1}\left( S \right)\in\mfX$ for all $S\in\mfY$.
    \end{definition}

    \np The preceding definition calls on the notion of preimage under $f$. Here are some useful properties of preimages.

    \begin{prop}{}
        Let $\left\lbrace S_i \right\rbrace^{}_{i\in I}$ be a collection of subsets of $Y$ and let $f:X\to Y$
        \begin{enumerate}
            \item $f^{-1}\left( \bigcup^{}_{i\in I}S_i \right)=\bigcup^{}_{i\in I}f^{-1}\left( S_i \right)$.
            \item $f^{-1}\left( \bigcap^{}_{i\in I}S_i \right)=\bigcap^{}_{i\in I}f^{-1}\left( S_i \right)$.
            \item For any $S\subseteq Y$, $f^{-1}\left( Y\setminus S \right) = f^{-1}\left( Y \right)\setminus f^{-1}\left( S \right)$.
            \item Let $g:Y\to Z$. Then for all $T\subseteq Z$, then $\left( g\circ f \right)^{-1}\left( T \right) = f^{-1}\left( g^{-1}\left( T \right) \right)$.
        \end{enumerate}
    \end{prop}

    \placeqed[See ProofWiki!]

    \np The most important instance of Def'n 2.8 is the one where $\left( Y,\mfY \right)$ is the real line considered with its Borel $\sigma$-algebra, that is, $Y=\R$ and $\mfY=\mB_{\R}$ (in what follows, we will write $\mB_{\R}$ instead of $\mB$ to avoid confusions).

    \begin{definition}{\textbf{Borel} Real-valued Functions}
        Given a measurable space $\left( X,\mfX \right)$, we will denote
        \begin{equation*}
            \bor\left( X,\R \right) = \left\lbrace f\in\R^X: \text{$f$ is $\mfX /\mB_{\R}$-measurable} \right\rbrace.
        \end{equation*}
        We shall call the elements of $\bor\left( X,\R \right)$ \emph{Borel} functions.
    \end{definition}

    \np The notation introduced in Def'n 2.9 is bit imprecise, because it does not mention explicitly what $\sigma$-algebra of subsets of $X$ is being considered. This is usually harmless, because it is clear from the context what is the $\sigma$-algebra $\mfX$ we are working with. In a situation where there could be a possibility of confusion of what is $\mfX$, we shall write $\bor\left( \left( X,\mfX \right),\R \right)$ instead.

    Our main concern for the lecture is to examine what properties we can expect from the \textit{space} of functions $\bor\left( X,\R \right)$. When calling $\bor\left( X,\R \right)$ space of functions, we are anticipating some good closure properties under various natural operations (addition, multiplication, \ldots) with functions. And we will indeed be able to establish a number of such properties. But first we put into evidence some simple general tools which cna be used in order to study measurable functions. A couple of such \textit{tools} are provided by the next two propositions.

    \begin{prop}{Composition of Measurable Functions Is Measurable}
        Let $f:X\to Y$ be an $\mfX /\mfY$-measurable function and let $g:Y\to Z$ be an $\mfY /\mfZ$-measurable function. Then $h=g\circ f$ is $\mfX /\mfZ$-measurable.
    \end{prop}

    \begin{proof}
        For all $C\subseteq Z$, we know that
        \begin{equation*}
            h^{-1}\left( C \right) = \left( g\circ f \right)^{-1}\left( C \right) = f^{-1}\left( g^{-1}\left( C \right) \right)
        \end{equation*}
        by Proposition 2.11. This means
        \begin{flalign*}
            && C\in\mfZ & \implies g^{-1}\left( C \right)\in\mfY && \\ 
            && & \implies f^{-1}\left( g^{-1}\left( C \right) \right)\in\mfX && \\
            && & \implies h^{-1}\left( C \right)\in\mfX.
        \end{flalign*}
        Thus we conclude that $h$ is $\mfX /\mfZ$-measurable, as required.
    \end{proof}

    \begin{prop}{}
        Let $f:X\to Y$ and let $\mC\subseteq\mfY$ be a collection of subsets of $Y$ that generates $\mfY$: that is, $\sigalg\left( \mC \right)=\mfY$. If $f^{-1}\left( C \right)\in\mfX$ for all $C\in\mC$, then $f$ is $\mfX /\mfZ$-measurable.
    \end{prop}

    \clearpage
    \begin{proof}
        Let
        \begin{equation*}
            \mfG = \left\lbrace S\subseteq Y: f^{-1}\left( S \right)\in\mfX \right\rbrace.
        \end{equation*}
        We have two claims.
        \begin{itemize}
            \item \textit{Claim 1. $\mfG$ is a $\sigma$-algebra of subsets of $Y$.}

                \begin{subproof}
                    This is easily done by using basic properties of preimages. Suppose that $\mS$ be a countable collection of sets from $\mfG$, where we desire to check $\bigcup\mS\in\mfG$. The latter fact amounts to checking that $f^{-1}\left( \bigcup^{}_{}\mS \right)\in\mfX$. And indeed, using Proposition 2.11 we find
                    \begin{equation}
                        f^{-1}\left( \bigcup^{\infty}_{n=1}S_n \right) = \bigcup^{\infty}_{n=1}f^{-1}\left( S_n \right).
                    \end{equation}
                    But for all $n\in\N$, $f^{-1}\left( S_n \right)\in\mfX$, since $S_n\in\mfG$. Hence the right-hand side of [2.7] is a countable union of sets from $\mfX$, so is in $\mfX$, since $\mfX$ is a $\sigma$-algebra. Thus $\bigcup^{\infty}_{n=1}S_n\in\mfG$, as required.
                \end{subproof}

            \item \textit{Claim 2. $\mfG\supseteq\mfY$.}

                \begin{subproof}
                    Since $f^{-1}\left( C \right)\in\mfX$ for all $C\in\mC$ by assumption, it follows that $\mC\subseteq\mfG$. This means $\mfG$ is a $\sigma$-algebra containing $\mC$, while, on the other hand, the hypothesis $\mfY=\sigalg\left( \mC \right)$ tells us that $\mfY$ is the minimal $\sigma$-algebra containing $\mC$. It follows that $\mfG\supseteq\mfY$.
                \end{subproof}
        \end{itemize} 
        In conclusion, for every $S\in\mfY$, we have $S\in\mfG$ by Claim 2, hence we have that $f^{-1}\left( S \right)\in\mfX$. This amounts precisely to saying that $f$ is $\mfX /\mfY$-measurable.
    \end{proof}

    \begin{cor}{}
        Let $X,Y$ be metric spaces and let $f:X\to Y$ be continuous. Then $f$ is $\mB_X /\mB_Y$-measurable.
    \end{cor}	

    \begin{proof}
        Let $\mT_X,\mT_Y$ denote the collections of open sets of $X,Y$, respectively.
        
        By Proposition 2.13, it suffices to show that
        \begin{equation*}
            \forall G\in\mT_Y\left[ f^{-1}\left( G \right)\in\mB_X \right].
        \end{equation*}
        But by a characterization of continuity,
        \begin{equation*}
            G\in\mT_Y\implies f^{-1}\left( G \right)\in\mT_X\implies f^{-1}\left( G \right)\in\mB_X
        \end{equation*}
        since $\mB_X\supseteq\mT_X$.
    \end{proof}

    \np We will also use a tool which deals specifically with functions that take values in a space $\R^n$. 

    \begin{prop}{}
        Let $f=\left( f_1,\ldots,f_n \right):X\to\R^n$ (that is, $f_1,\ldots,f_n:X\to\R$ are the \textit{component functions} of $f$). The following are equivalent.
        \begin{enumerate}
            \item $f$ is $\mfX /\mB_{\R^n}$-measurable.
            \item Each $f_j$ is $\mfX /\mB_{\R}$-measurable (i.e. $f_1,\ldots,f_n\in\bor\left( X,\R \right)$).
        \end{enumerate}
    \end{prop}

    \begin{proof}
        Fix $j\in\left\lbrace 1,\ldots,n \right\rbrace$, for which we will prove that $f_j\in\bor\left( X,\R \right)$. Let $P_j:\R^n\to\R$ be the $j$th projection map, defined by
        \begin{equation*}
            P\left( t_1,\ldots,t_n \right)=t_j
        \end{equation*}
        for all $\left( t_1,\ldots,t_n \right)\in\R^n$. Then $P_j$ is continuous, so $\mB_{\R^n}/\mB_{\R}$-measurable. It follows that $f_j=P_j\circ f$ is $\mfX /\mB_{\R}$-measurable.

        Conversely, suppose that $f_1,\ldots,f_n$ are $\mfX /\mB_{\R}$-measurable, where we want to prove that $f$ is $\mB_{\R^n} /\mB_{\R}$-measurable. We will do the proof by using Proposition 2.13 in connection to the collection $\mC$ of \textit{open cubes} in $\R^n$, defined as follows:
        \begin{equation*}
            \mC = \left\lbrace \left( a_1-r,a_1+r \right)\times\cdots\times\left( a_n-r,a_n+r \right): a_1,\ldots,a_n\in\R, r\in\left( 0,\infty \right) \right\rbrace.
        \end{equation*}
        In connection this collection of subsets of $\R^n$, we make two claims.
        \begin{itemize}
            \item \textit{Claim 1. $\sigalg\left( \mC \right)=\mB_{\R^n}$.}

                \begin{subproof}
                    Note that $\mC$ is precisely the collection of open balls with respect to $\left\lVert \cdot \right\rVert_{\infty}$. But by recalling the fact that every norm on $\R^n$ are equivalent and that the collection of open balls generate the usual topology (i.e. $\mC$ is a basis for the usual topology on $\R^n$), we arrive to the conclusion that $\sigalg\left( \mC \right)=\mB_{\R^n}$.
                \end{subproof}

            \item \textit{\textit{Claim 2. For all $C\in\mC$, $f^{-1}\left( C \right)\in\mfX$.}}

                \begin{subproof}
                    Let $C=\left( a_1-r,a_1+r \right)\times\cdots\times\left( a_n-r,a_n+r \right)\in\mC$. Since each $f_j$ is $\mfX /\mB_{\R}$-measurable, it follows that $f_j^{-1}\left( \left( a_j-r,a_j+r \right) \right)\in\mfX$. This means
                    \begin{equation*}
                        f^{-1}\left( S \right) = \bigcap^{n}_{j=1}f_j^{-1}\left( \left( a_j-r,a_j+r \right) \right)
                    \end{equation*}
                    is also in $\mfX$.
                \end{subproof}
        \end{itemize} 
        By invoking Proposition 2.13, we are done.
    \end{proof}

    \np We now arrive to the main point of this subsection, concerning the closure of $\bor\left( X,\R \right)$ under various algebraic operations that can be performed with real-valued functions. Here are some \textit{pointwise} operations we can perform on $f,g:X\to\R$.
    \begin{enumerate}
        \item Given $\alpha,\beta\in\R$, consider $\alpha f+\beta g$, defined by $\left( \alpha f+\beta g \right)\left( x \right)=\alpha f\left( x \right)+\beta g\left( x \right)$ for all $x\in\R$.\hfill\textit{linear combination}
        \item Consider $fg$, defined by $\left( fg \right)\left( x \right) = f\left( x \right)g\left( x \right)$ for all $x\in\R$.\hfill\textit{product}
        \item We define $f\vee g, f\wedge g$ by
            \begin{equation*}
                \left( f\vee g \right)\left( x \right) = \max\left( f\left( x \right),g\left( x \right) \right)
            \end{equation*}
            and by
            \begin{equation*}
                \left( f\wedge g \right)\left( x \right)=\min\left( f\left( x \right),g\left( x \right) \right)
            \end{equation*}
            for all $x\in\R$.\hfill\textit{maximum and minimum}
    \end{enumerate}
    It turns out that $\bor\left( X,\R \right)$ is closed under all these operations.

    \begin{prop}{}
        Let $f,g\in\bor\left( X,\R \right)$ and let $\alpha,\beta\in\R$. Then $fg,f\vee g, f\wedge g,\alpha f+\beta g\in\bor\left( X,\R \right)$.
    \end{prop}

    \begin{proof}
        We verify $f\vee g\in\bor\left( X,\R \right)$ only.

        Note that $F:X\to\R^{2}$ by $F=\left( f,g \right)$ is $\mfX /\mB_{\R^{2}}$-measurable by Proposition 2.14. Moreover, $\max:\R^{2}\to\R$ is continous, so is $\mB_{\R^{2}} /\mB_{\R}$-measurable. Thus $f\vee g = \max\circ F$ is $\mfX /\mB_{\R}$-measurable.
    \end{proof}

    \np Proposition 2.15 tells us that $\bor\left( X,\R \right)$ is a (unital) algebra of functions and is also a lattice of functions.

    \begin{exercise}{}
        \vspace{-11pt}
        \begin{enumerate}
            \item Let $\mC=\left\lbrace \left( -\infty,b \right]: b\in\Q \right\rbrace$. Prove that $\sigalg\left( \mC \right)=\mB_{\R}$.
            \item Let $f:X\to\R$. Suppose for every $b\in\Q$, $\left\lbrace x\in X: f\left( x \right)\leq b \right\rbrace$ is in $\mfX$. Prove $f\in\bor\left( X,\R \right)$.
        \end{enumerate}
    \end{exercise}

    \placeqed[tl;dr]

    \subsection{Convergence Properties of $\bor\left( X,\R \right)$}

    \np We now turn to properties of $\bor\left( X,\R \right)$ which have to do with \textit{pointwise convergnece} of sequences of functions. These will turn out to be  as good as well: the space $\bor\left( X,\R \right)$ is closed under pointwise convergence of sequences of functions, and we also have some variations of this fact where instead of a limit we play with $\inf$ and $\sup$ constructions.

    \np The main result of this subsection is in Proposition 2.16, which can be viewed as yet another tool for establishing measurability of a function. We will not give the proof of Proposition 2.16 right away; it will be a particular case of the more general Proposition 2.18.

    \begin{prop}{}
        Let $\left( X,\mA \right)$ be a measurable space and let $f:X\to\R$. If there exists $\left( f_{n} \right)^{\infty}_{n=1}\in\bor\left( X,\R \right)^{\N}$ such that $\lim_{n\to\infty}f_n = f$ pointwise, then $f\in\bor\left( X,\R \right)$.
    \end{prop}

    \placeqed[This Is a Particular Instance of Proposition 2.18]

    \begin{example}{}
        Here is an example of how Proposition 2.16 can be used. It goes in the framework of the measurable space $\left( \R,\mB_{\R} \right)$ -- the space $\bor\left( X,\R \right)$ of this example is thus $\bor\left( \R,\R \right)$.
        
        We consider the following statement: \textit{for all differentiable $f:\R\to\R$, $f'\in\bor\left( \R,\R \right)$.}

        From calculus, we know that derivatives can sometimes be \textit{nasty} (e.g. a derivative need not be continuous). The above statement says that $f'$ cannot be \textit{that} nasty as to not be a Borel function.

        In order to prove that $f'\in\bor\left( \R,\R \right)$, we write it as a pointwise limit
        \begin{equation*}
            f'\left( t \right) = \lim_{n\to\infty} f_n\left( t \right)
        \end{equation*}
        where $f_n:\R\to\R$ is defined by putting
        \begin{equation*}
            f_n\left( t \right) = \frac{f\left( t+\frac{1}{n} \right)-f\left( t \right)}{\frac{1}{n}}
        \end{equation*}
        for all $t\in\R$. Then note that each $f_n$ is continuous, so $f_n\in\bor\left( \R,\R \right)$. Thus $f'\in\bor\left( \R,\R \right)$ as a pointwise limit of $f_n$.
    \end{example}

    \rruleline

    \np Now let us start to work towards proving that Proposition 2.16 is indeed available. We will actually establish a more refined version of this statement, which instead of $\lim$ calls in for a $\limsup$ or a $\liminf$. On the way towards that, we first get a version which is merely using $\sup$ or $\inf$.

    \begin{prop}{}
        Let $\left( X,\mA \right)$ be a measurable space.
        \begin{enumerate}
            \item Let $f:X\to\R$. If there exists $\left( f_{n} \right)^{\infty}_{n=1}\in\bor\left( X,\R \right)^{\N}$ such that
                \begin{equation*}
                    \sup_{n\in\N}f_n\left( x \right) = f\left( x \right)
                \end{equation*}
                for all $x\in X$, then $f\in\bor\left( X,\R \right)$.
            \item Let $g:X\to\R$. If there exists $\left( g_{n} \right)^{\infty}_{n=1}\in\bor\left( X,\R \right)^{\N}$ such that
                \begin{equation*}
                    \inf_{n\in\N}g_n\left( x \right) = g\left( x \right)
                \end{equation*}
                for all $x\in X$, then $g\in\bor\left( X,\R \right)$.
        \end{enumerate}
    \end{prop}

    \begin{proof}
        We make use of the criterion
        \begin{equation*}
            \forall t\in\R\left[ f^{-1}\left( \left( -\infty,t \right] \right)\in\mA \right]\implies f\in\bor\left( X,\R \right)
        \end{equation*}
        provided in Exercise 2.5.\footnotemark[1] To that end, fix $t\in\R$. In connection to $t$, for every $x\in X$ we have
        \begin{equation*}
            f\left( x \right)\leq t\iff\sup_{n\in\N}f_n\left( x \right)\leq t\iff\forall n\in\N\left[ f_n\left( x \right)\leq t \right].
        \end{equation*}
        This implies we have
        \begin{equation}
            f^{-1}\left( \left( -\infty,t \right] \right) = \bigcap^{\infty}_{n=1} f_n^{-1}\left( \left( -\infty,t \right] \right).
        \end{equation}
        For every $n\in\N$, $f_n^{-1}\left( \left( -\infty,t \right] \right)\in\mA$, since it is the preimage of the interval $\left( -\infty,t \right]\in\mB_{\R}$ by the $\mA /\mB_{\R}$-measurable functoin $f_n$. This means that $f^{-1}\left( \left( -\infty,t \right] \right)$ is written as a countable intersection of sets from $\mA$ in [2.8]. It follows that $f^{-1}\left( \left( -\infty,t \right] \right)\in\mA$, as required. This completes (a).

        For (b), it suffices to note that 
        \begin{equation*}
            g\left( x \right) = -\sup_{n\in\N}-g_n\left( x \right)
        \end{equation*}
        for all $x\in X$, and use the result that $\bor\left( X,\R \right)$ is closed under linear combinations.
        
        \noindent
        \begin{minipage}{\textwidth}
            \footnotetext[1]{In fact, it suffices to check $f^{-1}\left( \left( -\infty,t \right] \right)\in\mA$ for all $t\in\Q$ only.}
        \end{minipage}
    \end{proof}

    \np The next (general) version of Proposition 2.16 will rely on the notions of $\limsup$ and $\liminf$ for a sequence of real numbers -- let us recall what those are.

    Let $\left( t_{n} \right)^{\infty}_{n=1}\in\R^{\N}$, and let us consider the set of all limit points of this sequence. That is, we look $P\subseteq\R\cup\left\lbrace \pm\infty \right\rbrace$ defined as follows
    \begin{equation*}
        P = \left\lbrace l\in\R\cup\left\lbrace \pm\infty \right\rbrace: \exists n_1,n_2,\ldots\in\N\left[ n_1<n_2<\cdots, \lim_{k\to\infty}t_{n_k} = l \right] \right\rbrace.
    \end{equation*}
    It can be proved that $P$ has a largest element and a smallest element. We define
    \begin{equation*}
        \limsup_{n\to\infty}t_n = \max\left( P \right), \liminf_{n\to\infty}t_n = \min\left( P \right).
    \end{equation*}

    An alternative way to reach the quantity $\max\left( P \right)$ is described as follows. Consider
    \begin{equation*}
        \beta_k = \sup_{n\geq k}t_n
    \end{equation*}
    for all $k\in\N$. Then $\left( \beta_{k} \right)^{\infty}_{k=1}$ is a decreasing sequence of real numbers, and it can be proved that
    \begin{equation*}
        \max\left( P \right) = \lim_{k\to\infty}\beta_k = \inf_{k\in\N}\beta_k.
    \end{equation*}
    Therefore, we obtain
    \begin{equation}
        \limsup_{n\to\infty} t_n = \inf_{k\geq 1}\sup_{n\ge qk}t_n.
    \end{equation}
    Similarly
    \begin{equation}
        \liminf_{n\to\infty}t_n = \sup_{k\geq 1}\inf_{n\geq k}\left( t_n \right).
    \end{equation}

    \begin{prop}{}
        Let $\left( X,\mA \right)$ be a measurable space.
        \begin{enumerate}
            \item Let $f:X\to\R$. If there exists $\left( f_{n} \right)^{\infty}_{n=1}\in\bor\left( X,\R \right)^{\N}$ such that
                \begin{equation*}
                    \limsup_{n\to\infty}f_n\left( x \right) = f\left( x \right)
                \end{equation*}
                for all $x\in X$, then $f\in\bor\left( X,\R \right)$.
            \item Let $g:X\to\R$. If there exists $\left( g_{n} \right)^{\infty}_{n=1}\in\bor\left( X,\R \right)^{\N}$ such that
                \begin{equation*}
                    \liminf_{n\to\infty}g_n\left( x \right) = g\left( x \right)
                \end{equation*}
                for all $x\in X$, then $g\in\bor\left( X,\R \right)$.
        \end{enumerate}
    \end{prop}

    \begin{proof}
        We start by noticing that, for every $x\in X$, the sequence $\left( f_{n}\left( x \right) \right)^{\infty}_{n=1}$ is bounded above. Indeed, if that was not the case, then we would be able to extract a subsequence of $\left( f_{n}\left( x \right) \right)^{\infty}_{n=1}$ which converges to $\infty$, and this would imply $\limsup_{n\to\infty}f_n\left( x \right)=\infty$, in contradiction with the given hypothesis that $\limsup_{n\to\infty}f_n\left( x \right)=f\left( x \right)\in\R$. So it makes sense to define $h_1:X\to\R$ by putting
        \begin{equation*}
            h_1\left( x \right) = \sup_{n\in\N}f_n\left( x \right)
        \end{equation*}
        for all $x\in X$. Moreover, Proposition 2.17 applies here, to give us that $h_1\in\bor\left( X,\R \right)$.

        We can similarly define, for all $k\geq 2$,
        \begin{equation*}
            h_x\left( x \right) = \sup_{n\geq k}f_n\left( x \right)
        \end{equation*}
        for all $x\in X$. Again, each $h_k\in\bor\left( X,\R \right)$.

        We note that, for all $x\in X$, the sequence $\left( h_{k}\left( x \right) \right)^{\infty}_{k=1}$ is decreasing, with
        \begin{equation*}
            \inf_{k\in\N}h_k\left( x \right) = \inf_{k\geq 1}\sup_{n\geq k}f_n\left( x \right) = \limsup_{n\to\infty}f_n\left( x \right)=f\left( x \right),
        \end{equation*}
        where the second equality sign we invoked [2.9].

        This means $f$ is presented as the (pointwise) infimum of the sequence of functions $\left( n_{k} \right)^{\infty}_{k=1}$. Since $h_k\in\bor\left( X,\R \right)$ for all $k\geq 1$, we can invoke Proposition 2.17 to conclude that $f\in\bor\left( X,\R \right)$. This marks the end of (a).

        For (b), we once again use the trick $f=-g, f_n=-g_n$ for all $N\in\N$.
    \end{proof}

    \np At this moment, we have also settled the Proposition 2.16 stated at the beginning of this subsection. Indeed, let $f$ and $\left( f_{n} \right)^{\infty}_{n=1}$ be as in Proposition 2.16. Then for every $x\in X$, $f\left( x \right)$ is the unique limit poitn of $\left( f_{n}\left( x \right) \right)^{\infty}_{n=1}$, so that
    \begin{equation*}
        \limsup_{n\to\infty}f_n\left( x \right)=\liminf_{n\to\infty}f_n\left( x \right)=f\left( x \right).
    \end{equation*}
    By invoking Proposition 2.18, we conclude that $f\in\bor\left( X,\R \right)$, as required.

    \np The statement of Proposition 2.18 was arranged in such a way that all the $\limsup$'s and $\liminf$'s considered there are forced to be finite. This is because there is a rescribed function $f$, taking finite values, which is arranged to be the $\limsup$ or $\liminf$ in question. Let us also have a look at what happens when we don't prescribe such an $f$.

    So let $\left( X,\mA \right)$ be a measurable space, and let $\left( f_{n} \right)^{\infty}_{n=1}\in\bor\left( X,\R \right)^{\N}$. Say we want to look at the $\limsup$ of $\left( f_{n} \right)^{\infty}_{n=1}$. It is obvious how we should define it, only that now we have to deal with the possibly nonempty sets
    \begin{equation*}
        A_+ = \left\lbrace x\in X: \limsup_{n\to\infty}f_n\left( x \right)=0 \right\rbrace, A_- = \left\lbrace x\in X:\limsup_{n\to\infty}f_n\left( x \right)=-\infty \right\rbrace.
    \end{equation*}

    \begin{exercise}{}
        Prove that $A_+,A_-\in\mA$.
    \end{exercise}

    \placeqed[tl;dr]

    \np Suppose next that we have decided to ignore what happens for the ponits $x\in A_+\cup A_-$, and we go ahead to define $f:X\to\R$ by putting
    \begin{equation*}
        f\left( x \right) = 
        \begin{cases} 0 & \text{if $x\in A_+\cup A_-$} \\ \limsup_{n\to\infty}f_n\left( x \right) & \text{otherwise} \end{cases}
    \end{equation*}
    for all $x\in X$.

    \begin{exercise}{}
        Prove that $f\in\bor\left( X,\R \right)$.
    \end{exercise}

    \placeqed[tl;dr]

    \np Upon solving it, we will thus have proved a result about \textit{$\bor\left( X,\R \right)$ being well-behaved under $\limsup$}, where we don't prescribe a target function $f$ which is to appear as $\limsup$.

    \clearpage
    \subsection{Simple Functions}

    Let $\left( X,\mA \right)$ be a measurable space throughout this subsection.

    \begin{definition}{\textbf{Simple} Function}
        We say $f\in\bor\left( X,\R \right)$ is \emph{simple} when it takes finitely many values (i.e. $\left| f\left( X \right) \right|<\left| \N \right|$). We write $\bors\left( X,\R \right)$ to denote the set of simple functions in $\bor\left( X,\R \right)$.
    \end{definition}

    \np One has a natural way of writing a general function $f\in\bor\left( X,\R \right)$ as a pointwise limit of functions from $\bors\left( X,\R \right)$. For convenience, we will focus here on the case when $f$ only takes values in $\left[ 0,\infty \right)$. The procedure for finding functions in $\bors\left( X,\R \right)$ which approximate $f$ can be thought of as a special way of \textit{binning the values of $f$}.
    
    More precisely, suppose we are given a function $f\in\bor\left( X,\R \right)$ such that $f\left( x \right)\geq 0$ for all $x\in X$. For all $n\in\N$, let us use the name \textit{$n$-th binning} of $f$ for $f_n:X\to\R$ defined as follows:
    \begin{equation*}
        f_n\left( x \right) =
        \begin{cases} 
            n & \text{if $f\left( x \right)\geq n$} \\
            \frac{k-1}{2^n} & \text{otherwise}
        \end{cases}
    \end{equation*}
    for all $n\in\N$, where, in case $f\left( x \right)<n$, $k$ is the unique element of $\left\lbrace 1,\ldots,n 2^n \right\rbrace$ such that $\frac{k-1}{2^n}\leq f\left( x \right)<\frac{k}{2^n}$.

    \begin{prop}{}
        Let $f\in\bor\left( X,\R \right)$ and let $\left( f_{n} \right)^{\infty}_{n=1}$ be defined as above. Then
        \begin{enumerate}
            \item for all $n\in\N$, $f_n\in\bors\left( X,\R \right)$; 
            \item for all $n\in\N$, $f_n\leq f_{n+1}$;\footnotemark[1]
            \item for all $x\in X$, $f_n\left( x \right)\nearrow f\left( x \right)$.\footnotemark[2]
        \end{enumerate}
        
        \noindent
        \begin{minipage}{\textwidth}
            \footnotetext[1]{We write $f\leq g$ for $f,g:X\to\R$ whenever $f\left( x \right)\leq g\left( x \right)$ for all $x\in\R$.}
            \footnotetext[2]{We write $f_n\left( x \right)\nearrow f\left( x \right)$ whenever $\left( f_{n}\left( x \right) \right)^{\infty}_{n=1}$ is an \textit{increasing sequence that converges to $f\left( x \right)$.}}
        \end{minipage}
    \end{prop}

    \placeqed[Exercise]

    \np A salient detail which is sure to appear during the verifications is this: 
    \begin{equation*}
        \forall x\in X\exists n_0\in\N\forall n\geq n_0 \left[ f\left( x \right)-\frac{1}{2^n}\leq f_n\left( x \right)\leq f\left( x \right) \right],
    \end{equation*}
    from where (c) of Proposition 2.19 follows.











































\end{document}
