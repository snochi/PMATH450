\documentclass[pmath450]{subfiles}

%% ========================================================
%% document

\begin{document}

    \section{Construction of Lebesgue Measure}
    
    \subsection{Length of Open Subsets of $\R$}
    
    \begin{recall}{\textbf{Equivalence Class}}
        Let $X$ be a nonempty set and $\sim$ be a relation on $X$. We say $\sim$ is an \emph{equivalence relation} if for every $x,y,z\in X$,
        \begin{enumerate}
            \item $x\sim x$;\hfill\textit{reflexivity}
            \item $x\sim y$ implies $y\sim x$; and\hfill\textit{symmetry}
            \item $x\sim y, y\sim z$ imples $x\sim z$.\hfill\textit{transitivity}
        \end{enumerate}
    \end{recall}

    \np
    Let $X$ be a nonempty set and let $\sim$ be an equivalence relation on $X$. This gives rise to a decomposition
    \begin{equation*}
        X = \bigcup^{}_{i\in I}C_i
    \end{equation*}
    where $\left\lbrace C_i \right\rbrace^{}_{i\in I}$ is a disjoint collection of subsets of $X$, and such that for all $x,y\in X$,
    \begin{enumerate}
        \item if $x,y\in C_i$ for some $i\in I$, then $x\sim y$; and
        \item if $x\in C_i, y\in C_j$ for some distinct $i,j\in I$, then $x\nsim y$.
    \end{enumerate}

    \begin{recall}{\emph{Equivalence Class} of an Equivalence Relation}
        Consider the above setting. We call each $C_i$ an \emph{equivalence class} of $\sim$.
    \end{recall}

    \np
    Conversely, if we have a partition $\left\lbrace C_i \right\rbrace^{}_{i\in I}$ of a nonempty set $X$, then we can define an equivalence relation $\sim$ on $X$ as follows: given $x,y\in X$,
    \begin{equation}
        x\sim y \iff \exists i\in I\left[ x,y\in C_i \right].
    \end{equation}

    \np
    We next look at how the notion of equivalence relation goes into the material of today's class.

    \begin{exercise}{}
        Let $A$ be a nonempty subset of $\R$. Define $\sim$ on $A$ as follows: for every $x,y\in A$ (say $x\leq y$ without loss of generality),
        \begin{equation}
            x\sim y\iff \left[ x,y \right]\subseteq A.
        \end{equation}
        \begin{enumerate}
            \item Prove that $\sim$ is an equivalence relation on $A$.
            \item Let $A=\bigcup^{}_{i\in I}C_i$ be the decomposition into equivalence classes for the above relation $\sim$. Prove that every $C_i$ is an interval.
        \end{enumerate}
    \end{exercise}

    \begin{proof}
        \begin{enumerate}
            \item Let $x,y,z\in A$. It is clear that $\sim$ is reflexive and symmetric. To show that $\sim$ is transitive, suppose $x\sim y, y\sim z$. We break down into few cases.

                If $x\leq y\leq z$, then $\left[ x,z \right]=\left[ x,y \right]\cup\left[ y,z \right]\subseteq A$.

                If $x\leq z\leq y$, then $\left[ x,z \right]\subseteq\left[ x,y \right]\subseteq A$.

                Other cases can be verified in a similar manner.

            \item Suppose $C_i$ is not an interval. Then there exists $x,y,z\in A$ such that $x<y<z$ and $x,z\in C_i$ but $y\notin C_i$. But this means $\left[ x,z \right]$ is not contained in $C_i$, which is a contradiction.
        \end{enumerate}
    \end{proof}

    \clearpage
    \np In the framework of Exercise 1.1, we will refer to the equivalence classes of $\sim$ by calling them \textit{interval components} of $A$.

    \np A silly question that comes to mind: what is a convenient definition, to be used in the solution to Exercise 1.1(b), for the notion of interval? It is preferable to go with the following unified description:
    \textit{a set $J\subseteq R$ is said to be an interval when it has the property that, if $x\leq y\leq z$ with $x,z\in J$,then $y\in J$.}

    \begin{definition}{\textbf{Open} Subset of $\R$}
        A subset $A\subseteq\R$ is \emph{open} if for every $x\in A$, there exists $r>0$ such that $\left( x-r,x+r \right)\subseteq A$.
    \end{definition}
    
    \begin{prop}{}
        Let $A\subseteq\R$ be nonempty and open and let $A=\bigcup^{}_{i\in I}C_i$ be the decomposition of $A$ into interval components. Then every $C_i$ is an open interval.
    \end{prop}

    \begin{proof}
        Fix $i\in I$ for which we will prove that $C_i$ is open (we know that $C_i$ is an interval from Exercise 1.1(b)). To that end, let us also fix a point $x\in C_i$. We need to find $r>0$ such that $\left( x-r,x+r \right)\subseteq C_i$.

        Since $x\in A$ and $A$ is open, there exists $r>0$ such that $\left( x-r,x+r \right)\subseteq A$. We will prove that this $r$ is what we need -- that is, we can strengthen the inclusion $\left( x-r,x+r \right)\subseteq A$ to $\left( x-r,x+r \right)\subseteq C_i$.

        Choose a point $y\in\left( x-r,x+r \right)$, for which we have to check $y\in C_i$. We argue like this
        \begin{flalign*}
            && y\in\left( x-r,x+r \right) & \implies \text{the whole interval with endpoints at $x,y$ is contained in $\left( x-r,x+r \right)$} && \\ 
            && & \implies \text{the whole interval with endpoints at $x,y$ is contained in $A$} && \\
            && & \implies x\sim y && \\
            && & \implies y\in C_i.
        \end{flalign*}
    \end{proof}

    \np Let us summarize what is the decomposition into interval components for a nonempty open $A\subseteq\R$: it is a decomposition $A=\bigcup^{}_{i\in I}C_i$, where
    \begin{enumerate}
        \item every $C_i$ is a nonempty open interval;
        \item $C_i\cap C_j=\emptyset$ for all distinct $i,j\in I$; and
        \item if $x,y\in A$ (without loss of generality, $x<y$) belong to different component intervals, then there is $z\in\R\setminus A$ such that $x<z<y$.
    \end{enumerate}
    It is instructive to look a bit more detail at the condition (c). This was written in a way which simply stated the fact that if $x\in C_i, y\in C_j$ for some distinct $i,j\in I$, then $x\nsim y$. It is easy to check that (c) can be rephrased in a strong form, as follows.

    \np\textit{Pick distinct indices $i,j\in I$ and pick two points $x_0\in C_i, y_0\in C_j$. Without loss of generality say $x_0<y_0$. Then there exists $z\in\R\setminus A$ such that $x<z$ for all $x\in C_i$ and $z<y$ for all $y\in C_j$.}

    \np What the above condition says is that the point $z$ \textit{separates} $C_i$ from $C_j$ in the stronger sense that all of $C_i$ is to the \textit{left} of $z$ while all of $C_j$ is to the \textit{right} of $z$.

    Here is another useful fact about (c): it actually follows for free if we have (a), (b). This is formally stated in the next exercise.

    \begin{exercise}{}
        Let $A\subseteq\R$ be nonempty and open and suppose we are given a decomposition $A=\bigcup^{}_{i\in I}C_i$, where every $C_i$ is a nonempty open interval and $C_i\cap C_j=\emptyset$ for all distinct $i,j\in I$.
        \begin{enumerate}
            \item Let $x,y\in A$ with $x<y$ and suppose that $x\in C_i, y\in C_j$ with $i\neq j$. Prove that there exists $z\in\R\setminus A$ such that $x<z<y$.
            \item Prove that the intervals $C_i$ are precisely the equivalence classes for the equivalence relation $\sim$ defined in [1.2]. 
        \end{enumerate}
    \end{exercise}

    \clearpage
    \begin{proof}
        \begin{enumerate}
            \item Write $C_i=\left( a_i,b_i \right), C_j=\left( a_j,b_j \right)$ for some $a_i,a_j,b_i,b_j\in\R$. Now note that
                \begin{equation*}
                    \emptyset = C_i\cap C_j = \left\lbrace \max\left\lbrace a_i,a_j \right\rbrace, \min\left\lbrace b_i,b_j \right\rbrace \right\rbrace.
                \end{equation*}
                This means $\max\left\lbrace a_i,a_j \right\rbrace>\min\left\lbrace b_i,b_j \right\rbrace$. But we know that $a_i<x<y<b_j, a_i<b_i, a_j<b_j$. Hence we conclude that $b_i<a_j$.

                Now the interval $\left[ b_i,a_j \right]$ is a nonempty closed interval, so cannot be written as a union of open intervals (since union of open sets is open, and the only clopen sets in $\R$ are $\emptyset,\R$; from PMATH 351, we know a metric space $X$ is connected if and only if $\emptyset, X$ are the only clopen sets and $\R$ is a connected space). Hence there exists $z\in \left[ b_i,a_j \right]\setminus A$, and by construction
                \begin{equation*}
                    x<b_i\leq z\leq a_j<y,
                \end{equation*}
                as required.

            \item Suppose $x,y\in C_i$ (with $x<y$ without loss of generality) for some $i\in I$. Since $C_i$ is an interval, $C_i$ is convex, so $\left[ x,y \right]\subseteq C_i\subseteq A$. Hence $x\sim y$.

                Conversely, suppose $x,y\in A$ are such that $x\sim y$ but $x\in C_i, y\in C_j$ for some distinct $i,j\in I$ for contradiction. Then by (a), we know there is $z\in\R\setminus A$ such that $x<z<y$. But $x\sim y$ if and only if $\left[ x,y \right]\subseteq A$ and $z\in \left[ x,y \right]$. This is a contradiction.
        \end{enumerate}
    \end{proof}

    \begin{exercise}{}
        Consider nonempty open $A\subseteq\R$ and the decomposition $A=\bigcup^{}_{i\in I}C_i$ of $A$ into interval components. Prove that $I$ is countable.
    \end{exercise}

    \begin{proof}
        For each $i\in I$, choose $q_i\in C_i$ such that $q_i\in\Q$. This is possible since $\Q$ is dense in $\R$ and every $C_i$ is a nonempty \textit{open} interval, so that $\Q\cap C_i\neq\emptyset$ for all $i\in I$. Since $Q_i$'s are disjoint, this defines an injection $\varphi:I\to\Q$ by
        \begin{equation*}
            \varphi\left( i \right) = q_i
        \end{equation*}
        for all $i\in I$. Hence
        \begin{equation*}
            \left| I \right| \leq \left| \Q \right| = \aleph_0.
        \end{equation*}
    \end{proof}

    \np The notion of length for an open subset of $\R$ is now easy to define, since we have a clear idea of what should be the length of an open \textit{interval}, and we can use the structural result from Proposition 1.1.

    \begin{definition}{\textbf{Length} of an Open Subset of $\R$}
        Let $A\subseteq\R$ be open. We define a quantity $\lambda\left( A \right)\in\left[ 0,\infty \right]$, which we will call (for now) \emph{length} of $A$, as follows.
        \begin{enumerate}
            \item If $A=\emptyset$, then $\lambda\left( A \right)=0$.
            \item Suppose $A\neq\emptyset$ and consider the decomposition $A=\bigcup^{}_{i\in I}C_i$ into interval components.
                \begin{enumerate}
                    \item If there is $i\in I$ such that $C_i$ is unbounded, then $\lambda\left( A \right)=\infty$.
                    \item If every $C_i$ is bounded, say $C_i=\left( a_i,b_i \right)$ for some $a_i,b_i\in\R$. Then we define
                        \begin{equation}
                            \lambda\left( A \right) = \sum^{}_{i\in I}\left( b_i-a_i \right)\in\left[ 0,\infty \right].
                        \end{equation}
                \end{enumerate}
        \end{enumerate}
    \end{definition}

    \np Here is a little discussion around the meaining of the sum in [1.3]. Note that, due to Exercise 1.3, the index set $I$ can be re-denoted in a way which makes it that either $I=\N$ or $I=\left\lbrace 1,\ldots,n \right\rbrace$ for some $k\in\N$. In other words, we are dealing either with a finite sum, or with a series of nonnegative real numbers (for which the order of summation does not matter).

    There actually is a way to handle the sum [1.3] which does not require a discussion around the cardinality of $I$, and is simply based on the notion of \textit{supremum} for a subset of $\left[ 0,\infty \right]$. More precisely, given open $A\subseteq\R$ which falls in (c) of Def'n 1.4, it may sometimes be convenient to rewrite the formula [1.3] in the form
    \begin{equation}
        \lambda\left( A \right)=\sup\left\lbrace \sum^{n}_{j=1}b_j-a_j: n\in\N, \text{$\left( a_1,b_1 \right),\ldots,\left( a_n,b_n \right)$ are pairwise disjoint interval components of $A$} \right\rbrace
    \end{equation}

    \np If $A$ is just a bounded open interval, say $A=\left( a,b \right)$, then we come to the unsurprising conclusion:
    \begin{equation*}
        \lambda\left( \left( a,b \right) \right)=b-a.
    \end{equation*}
    A consequence of this is yet another way of writing [1.3]:
    \begin{equation}
        \lambda\left( A \right) = \sum^{}_{i\in I}\lambda\left( C_i \right).
    \end{equation}
    It is useful to generalze the latter formula to a situation where the sets on the right-hand side don't have to be intervals.

    \begin{prop}{}
        Let $\left\lbrace A_j \right\rbrace^{}_{j\in J}$ be a collection of pairwise disjoint open subsets of $\R$. Then
        \begin{equation}
            \lambda\left( \bigcup^{}_{j\in J}A_J \right) = \sum^{}_{j\in J}\lambda\left( A_j \right)\in\left[ 0,\infty \right].
        \end{equation}
    \end{prop}
    
    \placeqed[Tl;dr]

    \np The collection of all open subsets of $\R$,
    \begin{equation*}
        \mT = \left\lbrace A\subseteq\R:A\text{ is open} \right\rbrace
    \end{equation*}
    goes under the name of \textit{topology} of $\R$. $\mT$ has some good properties in connection to set-operations:
    \begin{enumerate}
        \item for every $\mE\subseteq\mT$, $\bigcup\mE\in\mT$; and\hfill\textit{closure under union}
        \item for every finite $\mE\subseteq\mT$, $\bigcap\mE\in\mT$.\hfill\textit{closure under finite intersection}
    \end{enumerate}
    Then the association $A\mapsto\lambda\left( A \right)$ discussed before can be pitched by saying that: \textit{we have defined a function $\lambda:\mT\to\left[ 0,\infty \right]$}.

    \np We can restate the properties of $\lambda$.
    \begin{enumerate}
        \item $\lambda\left( \emptyset \right)=0$.
        \item $\lambda\left( \left( a,b \right) \right)=b-a$ for all $a,b\in\R\cup\left\lbrace \pm\infty \right\rbrace$ with $a<b$.\hfill\textit{values on open intervals}
        \item If $\left\lbrace A_n \right\rbrace^{\infty}_{n=1}\subseteq\mT$ is a collection of disjoint sets, then $\lambda\left( \bigcup^{\infty}_{n=1}A_n \right)=\sum^{\infty}_{n=1}\lambda\left( A_n \right)$.\hfill\textit{additivity for disjoint union}
    \end{enumerate}
    These properties turn out to completely determine $\lambda$, as stated by the next exercise.

    \begin{exercise}{}
        Let $\mu:\mT\to\left[ 0,\infty \right]$ which has the same properties with $\lambda$. That is:
        \begin{enumerate}
            \item $\mu\left( \emptyset \right)=0$;
            \item $\mu\left( \left( a,b \right) \right)=b-a$ for all $a,b\in\R\cup\left\lbrace \pm\infty \right\rbrace$ with $a<b$; and
            \item if $\left\lbrace A_n \right\rbrace^{\infty}_{n=1}\subseteq\mT$ is a collection of disjoint sets, then $\mu\left( \bigcup^{\infty}_{n=1}A_n \right)=\sum^{\infty}_{n=1}\mu\left( A_n \right)$.
        \end{enumerate}
        Prove that $\mu=\lambda$.
    \end{exercise}

    \begin{proof}
        Let $A\in\mT$. Since $A$ is open, there exists a countable partition $\left\lbrace A_i \right\rbrace^{}_{i\in I}$ of $A$ into open intervals. Then
        \begin{equation*}
            \mu\left( A \right) = \mu\left( \bigcup^{}_{i\in I}A_i \right) = \sum^{}_{i\in I}\mu\left( A_i \right) = \sum^{}_{i\in I}\lambda\left( A_i \right) = \lambda\left( \bigcup^{}_{i\in I}A_i \right) = \lambda\left( A \right).
        \end{equation*}
    \end{proof}

    \np Now that we started on the path of studying $\lambda:\mT\to\left[ 0,\infty \right]$, let us look for some other natural properties of $\lambda$.

    \begin{prop}{}
        If $A,B\in\mT$ are such that $A\subseteq B$, then $\lambda\left( A \right)\leq\lambda\left( B \right)$.
    \end{prop}

    \begin{pf}{\pffont Proof?}{\placeqed[???]}
        Let $A,B\in\mT$ with $A\subseteq A$. We denote $A_1=A, A_2=B\setminus A$. Then $A_1\cup A_2=B$. It is also clear that $A_1\cap A_2=\emptyset$. Hence
        \begin{equation*}
            \lambda\left( A_1\cup A_2 \right)=\lambda\left( A_1 \right)+\lambda\left( A_2 \right).
        \end{equation*}
        We can thus write
        \begin{equation*}
            \lambda\left( B \right) = \lambda\left( A_1\cup A_2 \right) = \lambda\left( A_1 \right)+\lambda\left( A_2 \right) \geq \lambda\left( A_1 \right)=\lambda\left( A \right),
        \end{equation*}
        as required.
    \end{pf}

    \np The issue of the above \textit{proof} is that $\mT$ is \textit{not closed under set difference} (so that $\lambda\left( A_2 \right)$ is \textit{not defined}). That is something we will have to cope with.

    \subsection{Positive Measure on a $\sigma$-algebra}
    
    Let us now have a look at what brand of set-function we \textit{would like} to have -- \textit{positive measure} is how it is called. We start by clarifying what kind of collection of subsets (of the real line, or more generally of some space $X$) we want to use, in order to talk about a positive measure.

    \begin{definition}{\textbf{$\sigma$-algebra} of Subsets}
        Let $X$ be a nonempty set and let $\mA$ be a colelction of subsets of $X$. We say $\mA$ is a \emph{$\sigma$-algebra} of subsets of $X$ to mean that
        \begin{enumerate}
            \item $\emptyset\in\mA$;
            \item for all countable $\mF\subseteq\mA$, $\bigcup\mF\in\mA$; and\hfill\textit{closure under countable union}
            \item for all $A\in\mA$, $X\setminus A\in\mA$.\hfill\textit{closure under complement}
        \end{enumerate}
    \end{definition}

    \begin{definition}{\textbf{Positive Measure} on a $\sigma$-algebra}
        Let $X$ be a nonempty set and let $\mA$ be a $\sigma$-algebra of subsets of $X$. A \emph{positive measure} on $\mA$ is a function $\mu:\mA\to\left[ 0,\infty \right]$ such that
        \begin{enumerate}
            \item $\mu\left( \emptyset \right)=0$; and
            \item for all collection $\mF\subseteq\mA$ of pairwise disjoint subsets of $X$, $\bigcup\mF\in\mA$.\hfill\textit{$\sigma$-additivity}
        \end{enumerate}
    \end{definition}

    \np The framework of a positive measure on a $\sigma$-algebra is the one which works like a charm when we want to build the integration theory of Lebesgue. It will be quite desirable to streamline our considerations and make them fall in this framework!

    We note that the topology $\mT$ satisfies (a), (b) of Def'n 1.5, and the set-function $\lambda:\mT\to\left[ 0,\infty \right]$ has $\lambda\left( \emptyset \right)$ and is countably additive. However, $\mT$ is actually quite far from satisfying (c) of Def'n 1.5. We will have to look into that as we proceeds.

    The remaining part of this subsection is devoted to making some easy (but useful) remarks about properties of $\sigma$-algebras and of positive measures automatically follow.

    \begin{prop}{Some Properties of a $\sigma$-algebra}
        Let $X$ be a nonempty set and let $\mA$ be a $\sigma$-algebra of $X$.
        \begin{enumerate}
            \item $X\in\mA$.
            \item For all countable $\mF\subseteq\mA$, $\bigcap\mF\in\mA$.\hfill\textit{closure under countable intersection}
            \item For all $A,B\in\mA$, $A\setminus B\in\mA$.\hfill\textit{closure under set difference}.
        \end{enumerate}
    \end{prop}

    \begin{proof}
        \begin{enumerate}
            \item Since $\emptyset\in\mA$ and $X\setminus\emptyset=X$, $X\in\mA$.
            \item Let $\mF\subseteq\mA$ be countable, say $\mF=\left\lbrace F_i \right\rbrace^{}_{i\in I}$. Then
                \begin{equation*}
                    \bigcap\mF = \bigcap^{}_{i\in I}F_i = \bigcap^{}_{i\in I}X\setminus\left( X\setminus F_i \right) = X\setminus\bigcup^{}_{i\in I}\left( X\setminus F_i \right).
                \end{equation*}
                Since $\mA$ is closed under complement and countable union, $\bigcup^{}_{i\in I}\left( X\setminus F_i \right)\in\mA$. Hence $\bigcap\mF\in\mA$.
            \item Since $X\in A$, $A\setminus B = A\cap \left( X\setminus B \right)\in\mA$.
        \end{enumerate}
    \end{proof}

    \begin{prop}{Monotonicity of Positive Measures}
        Let $X$ be a nonempty set and let $\mA$ be a $\sigma$-algebra of subsets of $X$. Let $\mu:\mA\to\left[ 0,\infty \right]$ be a positive measure. Then $\mu$ is increasing: for all $A,B\in\mA$ such that $A\subseteq B$, $\mu\left( A \right)\leq\mu\left( B \right)$.
    \end{prop}

    \begin{proof}
        Observe that
        \begin{equation*}
            \mu\left( B \right) = \mu\left( A\cup \left( B\setminus A \right) \right) = \mu\left( A \right)+\mu\left( B\setminus A \right)\geq \mu\left( A \right).
        \end{equation*}
    \end{proof}

    \np We remark that above proof, although it looks identical to the failed one given for Proposition 1.3, works since $\mA$ is closed under set difference.

    \subsection{Back to Proposition 1.3}

    Subsection 1.2 was a bit of a detour from the main object of the present lecture, the set-function $\lambda:\mT\to\left[ 0,\infty \right]$. A good detour, as the notion of positive measure on a $\sigma$-algebra will turn out to be of great importance for this course.

    But, going back to the Proposition 1.3 we started with: since $\mT$ is not a $\sigma$-algebra, Proposition 1.5 doe snot apply to $\lambda:\mT\to\left[ 0,\infty \right]$, so we still don't have a proof that $\lambda$ is increasing. We can still go ahead and prove Proposition 1.3 via a direct analysis of how open sets decompose into interval components. For clarity, we separate some parts of the argument as lemmas.

    \begin{lemma}{}
        Let $a_1<b_1,\ldots,a_n<b_n$ be in $\R$, where $a_1<\cdots<a_n$, and suppose that the open intervals $\left( a_1,b_1 \right),\ldots,\left( a_n,b_n \right)$ are pairwise disjoint. Then $a_1<b_1\leq a_2<b_2\leq\cdots\leq a_{n-1}<b_{n-1}\leq a_n<b_n$.
    \end{lemma}
    
    \begin{proof}
        It suffices to examine the case $n=2$ and show $b_1\leq a_2$.

        For contradiction, suppose that $b_1>a_2$. Then observe that $\left( a_2,\min\left\lbrace b_1,b_2 \right\rbrace \right)$ is a nonempty interval contained in both $\left( a_1,b_1 \right),\left( a_2,b_2 \right)$. This contradicts the fact that $\left( a_1,b_1 \right)\cap\left( a_2,b_2 \right)=\emptyset$.
    \end{proof}

    \begin{lemma}{}
        Let $a,b\in\R$ with $a<b$ and let $A\subseteq\R$ be open and such that $A\subseteq\left( a,b \right)$. Then $\lambda\left( A \right)\leq b-a$.
    \end{lemma}

    \begin{proof}
        The statement is clear for $A=\emptyset$, so we will assume $A\neq\emptyset$. We use the description of $\lambda\left( A \right)$ as a supremum. In view of the definition of a supermum, the required inequality $\lambda\left( A \right)\leq b-a$ will follow if we can verify the following claim.
        \begin{itemize}
            \item \textit{Claim 1. Let $\left( a_1,b_1 \right),\ldots,\left( a_n,b_n \right)$ be some pairwise disjoint interval components of $A$. Then $\sum^{n}_{j=1}\left( b_j-a_j \right)\leq b-a$.}

                \begin{subproof}
                    Fix some intervals $\left( a_1,b_1 \right),\ldots,\left( a_n,b_n \right)$ as in the claim. Reordering them if necessary, we may assume $a_1<\cdots<a_n$. Then,
                    \begin{flalign*}
                        && \sum^{n}_{j=1}\left( b_j-a_j \right) & \leq \sum^{n}_{j=1}\left( b_j-a_j \right) + \left( \left( a_2-b_1 \right)+\cdots+\left( a_n-b_{n-1} \right) \right) && \text{by Lemma 1.6}\\ 
                        && & = \left( b_1-a_1 \right) + \left( a_2-b_1 \right) + \left( b_2-a_2 \right) + \cdots + \left( a_n-b_{n-1} \right) + \left( b_n-a_n \right) && \\
                        && & = b_n-a_1 && \text{\textit{telescopic} sum} \\
                        && & \leq b-a. && \text{since $A\subseteq\left( a,b \right)$ implies $a\leq a_1, b_n\leq b$.}
                    \end{flalign*}
                    This verifies the claim and thus the statement.
                \end{subproof}
        \end{itemize} 
    \end{proof}

    \begin{boxyproof}{Proof of Proposition 1.3}
        Let $A,B\in\mT$ be such that $A\subseteq B$. We have to prove that $\lambda\left( A \right)\leq\lambda\left( B \right)$. If $\lambda\left( B \right)=\infty$, then the inequality to be proved is obvious; we will therefore assume $\lambda\left( B \right)<\infty$. We will also assume $B\neq\emptyset$.

        We consider the (countable) decomposition $B=\bigcup^{}_{i\in I}C_i$ of $B$ into open intervals. From the assumption that $\lambda\left( B \right)<\infty$, it follows that every $C_i$ is a bounded interval, say $C_i=\left( a_i,b_i \right)$ for some $a_i,b_i\in\R$ with $a_i<b_i$. The value $\lambda\left( B \right)$ is defined as $\sum^{}_{i\in I}\left( b_i-a_i \right)$.

        For every $i\in I$, let us denote $A_i=A\cap\left( a_i,b_i \right)$. This is a set in $\mT$, as the intersection of two open sets is still open. It is clear, moreover, that $A_i\subseteq\left( a_i,b_i \right)$, hence Lemma 1.7 applies and gives us the upper bound
        \begin{equation*}
            \lambda\left( A_i \right) \leq b_i-a_i
        \end{equation*}
        for all $i\in I$.

        We observe that $A_i\cap A_j=\emptyset$ for any distinct $i,j\in I$, since $A_i\subseteq\left( a_i,b_i \right), A_j\subseteq\left( a_j,b_j \right)$ and $\left( a_i,b_i \right)\cap\left( a_j,b_j \right)=\emptyset$. Moreover,
        \begin{equation*}
            \bigcup^{}_{i\in I}A_i = \bigcup^{}_{i\in I}\left( A\cap \left( a_i,b_i \right) \right) = A\cap\left( \bigcup^{}_{i\in I}\left( a_i,b_i \right) \right) = A\cap B = A.
        \end{equation*}
        We thus find that
        \begin{flalign*}
            && \lambda\left( A \right) & = \sum^{}_{i\in I}\lambda\left( A_i \right) && \\ 
            && & \leq \sum^{}_{i\in I}\left( b_i-a_i \right) && \\
            && & = \lambda\left( B \right),
        \end{flalign*}
        and we conclude that $\lambda\left( A \right)\leq\lambda\left( B \right)$, as required.
    \end{boxyproof}

    \subsection{Length of Compact Subsets of $\R$}

    We noticed there are some difficulties coming from the fact that the collection $\mT$ of open subsets of $\R$ is not closed under complements. This issue is in fact quite drastic:
    \begin{equation}
        \forall A\in\mT\left[ A\neq\emptyset,A\neq\R\implies\R\setminus A\notin\mT \right].
    \end{equation}
    Indeed, taking complements of open sets gives the \textit{closed} subsets of the real line. The statement in [1.7] thus says the only subsets of $\R$ which are \textit{clopen} are the empty set $\emptyset$ and the total set $\R$. This is a statement that arises when discussing about the fact that \textit{the real line is connected}.

    We are now interested in defining notion of length for closed sets. It will be in fact easier to figure the things out if we focus on \textit{compact} subsets of $\R$ (i.e. closed sets $K\subseteq\R$ which are also \textit{bounded}). Some of the properties of compact sets (e.g. open covers, finite intersection property, sequential compactness,\ldots) from PMATH 351 may come in handy in our discussions as well.

    Now, how should we proceed in order to define the length $\lambda\left( K \right)$ for a compact subset $K\subseteq\R$? We do not have at our disposal some neat result about the structure of $K$, in the way we had when we discussed about open sets; but we can actually \textit{fall back} on what we know about lengths of open sets, in the way described as follows. We start by enclosing $K$ in a bounded open set $A$ (which is possible since $K$ is bounded -- we can even arrange $A$ to be an open interval, if we want). Then $A\setminus K$ is a bounded open set as well.

    \begin{exercise}{}
        Let $K\subseteq\R$ be compact and let $A\subseteq\R$ be an open set containing $K$. Show that $A\setminus K$ is open.
    \end{exercise}

    \begin{proof}
        Since $K$ is compact, $K$ is closed, so $\R\setminus K$ is open. This means $A\setminus K = A\cap\left( \R\setminus K \right)$ is open.
    \end{proof}
    
    \np The sets $K,A\setminus K$ are disjoint, and their union is $A$, since we are hoping that the notion of length is additive, we should then have
    \begin{equation*}
        \lambda\left( K \right)+\lambda\left( A\setminus K \right)=\lambda\left( A \right)\in\left[ 0,\infty \right),
    \end{equation*}
    and we can turn this equation into a formula used to define the length of $K$:
    \begin{equation}
        \lambda\left( K \right)=\lambda\left( A \right)-\lambda\left( A\setminus K \right).
    \end{equation}
    Can this really work? In [1.8] we see an \textit{idea} for how to proceed, but can we be really sure that the right-hand side of the equation does not depend on how $A$ was chosen?

    \np In [1.8] it feels bit uncomfortable that we are using the same letter $\lambda$ bot hfor lengths of open sets (a well-defined notion studied in the preceding subsections) and for lengths of compact sets, a notion which is, for the moment, not rigorously defined. In order to avoid any confusions, let us make the following conventions of notation.

    \begin{notation}{$\lambda_{\op}$}
        We introduced the length-measuring map $\lambda:\mT\to\left[ 0,\infty \right]$, where $\mT$ is the collection of the open subsets of $\R$. Let us (at least temporarily) change the name of this map to $\lambda_{\op}$, with the subscript \textit{op} meant to remind us of open sets.
    \end{notation}

    \np On the other hand, let us write
    \begin{equation*}
        \mK = \left\lbrace K\subseteq\R:K\text{ is compact} \right\rbrace.
    \end{equation*}
    The goal is to define a function $\lambda_{\cp}:\mK\to\left[ 0,\infty \right)$, where for every $K\in\mK$ the number $\lambda_{\cp}$ is our notion of \textit{length of $K$}. The subscript \textit{cp} is meant to remind us that $\lambda_{\cp}$ is addressing lengths of compact sets.

    The plan presented before is now phrased as follows: the new length-measuring map $\lambda_{\cp}:\mK\to\left[ 0,\infty \right)$ has to defined in such a way that we have the formula
    \begin{equation}
        \forall K\in\mK\forall A\in\mT \left[ K\subseteq A\implies \lambda_{\cp}\left( K \right)+\lambda_{\op}\left( A\setminus K \right) = \lambda_{\op}\left( A \right)\right].
    \end{equation}
    As noticed before, we can try to use the formula [1.9] as a lever in order to actually \textit{define} what is $\lambda_{\cp}$ (but some proof is required, in order to be certain that the definition makes sense).

    \np We start with an easy observation involving finite subsets of an open set.

    \begin{lemma}{}
        Let $A\subseteq\R$ be open and let $F\subseteq A$ be finite. Then $A\setminus F$ is open and
        \begin{equation*}
            \lambda_{\op}\left( A\setminus F \right)=\lambda_{\op}\left( A \right).
        \end{equation*}
    \end{lemma}

    \begin{proof}
        We consider the following claim.
        \begin{itemize}
            \item \textit{Claim 1. Let $x\in A$. Then $A\setminus \left\lbrace x \right\rbrace$ is open and $\lambda_{\op}\left( A\setminus \left\lbrace x \right\rbrace \right)=\lambda_{\op}\left( A \right)$.}

                \begin{subproof}
                    We consider the decomposition
                    \begin{equation*}
                        A = \bigcup^{}_{i\in I}C_i
                    \end{equation*}
                    of $A$ into interval components, and let $i_0$ be the index in $I$ for which $x\in C_{i_0}$. Clearly, we have
                    \begin{equation}
                        A\setminus \left\lbrace x \right\rbrace = \left( C_{i_0}\setminus \left\lbrace x \right\rbrace \right)\cup \bigcup^{}_{i\in I:i\neq i_0} C_i.
                    \end{equation}
                    Direct inspection shows that $C_{i_0}\setminus \left\lbrace x \right\rbrace$ is a union of two disjoint open intervals $C'$ and $C''$, where
                    \begin{equation}
                        \lambda_{\op}\left( C' \right)+\lambda_{\op}\left( C'' \right) = \lambda_{\op}\left( C_{i_0} \right)\in\left[ 0,\infty \right].
                    \end{equation}
                    So then [1.10] becomes
                    \begin{equation}
                        A\setminus \left\lbrace x \right\rbrace = C'\cup C''\cup \left( \bigcup^{}_{i\in I:i\neq i_0}C_i \right).
                    \end{equation}
                    On the right-hand side of [1.12] we have a countable union of open intervals which are pairwise disjoint, and we can thus compute as follows:
                    \begin{flalign*}
                        && \lambda_{\op}\left( A\setminus \left\lbrace x \right\rbrace \right) & = \lambda_{\op}\left( C' \right)+\lambda_{\op}\left( C'' \right)+\sum^{}_{i\in I:i\neq i_0}\lambda_{\op}\left( C_i \right) && \\ 
                        && & = \lambda_{\op}\left( C_{i_0} \right) + \sum^{}_{i\in I:i\neq i_0}\lambda_{\op}\left( C_i \right) && \text{by [1.11]} \\
                        && & = \lambda_{\op}\left( A \right),
                    \end{flalign*}
                    where the last equality sign of this derivation we simply have the definition of $\lambda_{\op}\left( A \right)$.\hfill\textit{(Claim 1 is verified)}
                \end{subproof}
        \end{itemize} 
        We now proceed inductively on the cardinality $\left| F \right|$. The case when $\left| F \right|=0$ is trivial, and the one when $\left| F \right|=1$ is covered by Claim 1. 

        For the inductive step, we fix $k\geq 1$. Consider an open set $A\subseteq\R$ and a subset $F\subseteq A$ such that $\left| F \right|=k+1$. We isolate one of the elements $x\in F$ and we write $F=F_0\cup \left\lbrace x \right\rbrace$ where $\left| F_0 \right|=k$. Then
        \begin{equation*}
            A\setminus F = B\setminus \left\lbrace x \right\rbrace
        \end{equation*}
        where $B=A\setminus F_0$. Then by induction $B$ is open with $\lambda_{\op}\left( A \right)=\lambda_{\op}\left( B \right)$, while Claim 1 gives us that $\lambda_{\op}\left( B \right)=\lambda_{\op}\left( B\setminus \left\lbrace x \right\rbrace \right)$. Putting these things together we find
        \begin{equation*}
            \lambda_{\op}\left( A \right) = \lambda_{\op}\left( B\setminus \left\lbrace x \right\rbrace \right) = \lambda_{\op}\left( A\setminus F \right),
        \end{equation*}
        as required.
    \end{proof}

    \begin{lemma}{}
        Let $K\subseteq\R$ be compact. Let $\left( a',b' \right),\left( a'',b'' \right)$ be open intervals containing $K$ and let $D'=\left( a',b' \right)\setminus K,D''=\left( a'',b'' \right)\setminus K$. Then
        \begin{equation}
            \left( b'-a' \right)-\lambda_{\op}\left( D' \right) = \left( b''-a'' \right)-\lambda_{\op}\left( D'' \right)\in\left[ 0,\infty \right).
        \end{equation}
    \end{lemma}

    \begin{proof}
        The quantities indicated on the two sides of [1.13] are indeed numbers in $\left[ 0,\infty \right)$. 

        The first thought about the proof of [1.13] is that we have to distinguish some cases (can have $a'<a''$ or $a'=a''$ or $a'>a''$). This can, however, be avoided, if we go as follows: fix some real numbers $a,b$ such that
        \begin{equation*}
            a<\min\left( a',a'' \right), b>\max\left( b',b'' \right)
        \end{equation*}
        and look at the open interval $\left( a,b \right)$; this contains both $\left( a',b' \right)$ and $\left( a'',b'' \right)$, hence in particular contains $K$. We consider the open set $D=\left( a,b \right)\setminus K$, and we will prove that either side of [1.13] is equal to $\left( b-a \right)-\lambda_{\op}\left( K \right)$. This will prove, in particular, that the required inequality [1.13] is holding.

        By symmetry, it is clearly sufficient to prove that one of the two sides of [1.13] is equal to $\left( b-a \right)-\lambda_{\op}\left( K \right)$. Say we focus on checking that
        \begin{equation*}
            \left( b'-a' \right)=\lambda_{\op}\left( D' \right)=\left( b-a \right)-\lambda_{\op}\left( D \right).
        \end{equation*}
        A bit of algebra shows the latter equation to be equivalent to
        \begin{equation}
            \lambda_{\op}\left( D \right)=\lambda_{\op}\left( D' \right)+\left( a'-a \right)+\left( b-b' \right)
        \end{equation}
        and thus it will suffice to verify the validity of [1.14].

        But now, if we start from the fact that $a<a'<b'<b$ with $K\subseteq\left( a',b' \right)$ and with $D=\left( a,b \right)\setminus K, D'=\left( a',b' \right)\setminus K$, then direct inspection gives us the relation
        \begin{equation*}
            D = D'\cup\left( a,a' \right]\cup\left[ b',b \right),
        \end{equation*}
        which in turn implies the equality of open sets
        \begin{equation}
            D\setminus \left\lbrace a',b' \right\rbrace = D'\cup\left( a,a' \right)\cup\left( b',b \right).
        \end{equation}
        We note, moreover, that the union on the right-hand side of [1.15] involves three open sets that are pairwise disjoint. We can then write
        \begin{flalign*}
            && \lambda_{\op}\left( D \right) & = \lambda_{\op}\left( D\setminus \left\lbrace a',b' \right\rbrace \right) && \text{by Lemma 1.8} \\
            && & = \lambda_{\op}\left( D'\cup\left( a,a' \right)\cup\left( b',b \right) \right) && \text{by [1.15]}\\
            && & = \lambda_{\op}\left( D' \right)+\lambda_{\op}\left( \left( a,a' \right) \right)+\lambda_{\op}\left( \left( b',b \right) \right) && \\
            && & = \lambda_{\op}\left( D' \right)+\left( a'-a \right)+\left( b-b' \right),
        \end{flalign*}
        which gives precisely [1.14] we had been left to prove.
    \end{proof}

    \begin{definition}{\textbf{Length} of a Compact Set}
        Let $K\subseteq\R$ be compact. We define the \emph{length} of $K$, denoted as $\lambda_{\cp}\left( K \right)$, by the following procedure. Pick an open interval $\left( a,b \right)$ that contains $K$, consider the open set $D=\left( a,b \right)\setminus K$, and define
        \begin{equation}
            \lambda_{\cp}\left( K \right)=\left( b-a \right)-\lambda_{\op}\left( D \right).\footnotemark[1]
        \end{equation}
        
        \noindent
        \begin{minipage}{\textwidth}
            \footnotetext[1]{The quantity on the right-hand side of [1.16] depends only on $K$ due to Lemma 1.9.}
        \end{minipage}
    \end{definition}

    \np The length-measuring map $\lambda_{\cp}:\mK\to\left[ 0,\infty \right)$ is now rigorously defined, and it was indeed obtained according to the plan -- by using a special case of the formula [1.9], the case where the enclosing bounded open set $A$ is an interval. For future use, we would like to have said formula [1.9] available in its general case; thus, we do some bootstrapping -- from the special case of $A=\left( a,b \right)$ we move up to the one where $A$ is a general open subset of $\R$.

    We first go from the case $A=\left( a,b \right)$ to the one where $A$ is a finite disjoint union of bounded open intervals.

    \begin{lemma}{}
        Let $a_1<b_1\leq a_2<b_2\leq\cdots\leq a_k<b_k$ be real numbers, and consider the open set $A=\bigcup^{k}_{j=1}\left( a_j,b_j \right)$. Suppose that $K\subseteq\R$ is compact and contained in $A$. Then
        \begin{equation}
            \lambda_{\cp}\left( K \right)+\lambda_{\op}\left( A\setminus K \right) = \sum^{k}_{j=1} \left( b_j-a_j \right).
        \end{equation}
    \end{lemma}

    \begin{proof}
        Consider the open interval $\left( a_1,b_k \right)$. This clearly contains both $A,K$. Now,
        \begin{equation}
            \lambda_{\cp}\left( K \right) = \left( b_k-a_1 \right) - \lambda_{\op}\left( D \right)
        \end{equation}
        where $D=\left( a_1,b_k \right)\setminus K$. Moreover, since $K\subseteq A = \bigcup^{k}_{j=1}\left( a_j,b_j \right)\subseteq\left( a_1,b_k \right)$,
        \begin{equation}
            D = \left( \bigcup^{k-1}_{j=1} \left[ b_j,a_{j+1} \right] \right) \cup \left( A\setminus K \right),
        \end{equation}
        where the right-hand side of [1.19] is a disjoint union. Hence
        \begin{flalign*}
            && \lambda_{\op}\left( D \right) & = \lambda_{\op}\left( D\setminus \left\lbrace b_1,a_2,b_2,\ldots,a_{k-1},b_{k-1},a_k \right\rbrace \right) && \\ 
            && & = \lambda_{\op}\left(\left( \bigcup^{k-1}_{j=1}\left( b_j,a_{j+1} \right) \right)\cup\left( A\setminus K \right)\right) && \\
            && & = \lambda_{\op}\left( A\setminus K \right) + \sum^{k-1}_{j=1} \left( a_{j+1}-b_j \right).
        \end{flalign*}
        Thus [1.18] becomes
        \begin{flalign*}
            && \lambda_{\cp}\left( K \right) & = \left( b_k-a_1 \right) - \sum^{k-1}_{j=1}\left( a_{j+1}-b_j \right) - \lambda_{\op}\left( A\setminus K \right) && \\ 
            && & = \left( b_k-a_1 \right) + \sum^{k-1}_{j=1} \left( b_j-a_{j+1} \right) - \lambda_{\op}\left( A\setminus K \right) && \\
            && & = \sum^{k}_{j=1}\left( b_j-a_j \right) - \lambda_{\op}\left( A\setminus K \right),
        \end{flalign*}
        rearranging which gives
        \begin{equation*}
            \lambda_{\cp}\left( K \right)-\lambda_{\op}\left( A\setminus K \right) = \sum^{k}_{j=1}\left( b_j-a_j \right),
        \end{equation*}
        which is what we intended to show.
    \end{proof}

    \begin{prop}{}
        Let $K\subseteq\R$ be compact and let $A\subseteq\R$ be an open set that contains $K$. Then
        \begin{equation}
            \lambda_{\cp}\left( K \right) + \lambda_{\op}\left( A\setminus K \right) = \lambda_{\op}\left( A \right)\in\left[ 0,\infty \right].
        \end{equation}
    \end{prop}

    \begin{proof}
        It is convenient that we first set aside the case when the following happens: there exists an unbounded open interval contained in $A$. In this case, upon invoking Proposition 1.3 we find that $\lambda_{\op}\left( A \right)\leq\lambda_{\op}\left( U \right)=\infty$, hence that $\lambda_{\op}\left( A \right)=\infty$. On the other hand, it is immediate that $U\setminus K$ must contain some unbounded open interval $V$; hence $A\setminus K\supseteq U\setminus K\supseteq V$, which implies that $\lambda_{\op}\left( A\setminus K \right)$ is infinite as well. We conclude that in this case the formula [1.20] does indeed hold, with both its sides begin equal to $\infty$.

        For the rest of the proof, we assume that there is no unbounded open interval contained in $A$> We consider the decomposition of $A$ into inteval components
        \begin{equation}
            A = \bigcup^{}_{i\in I}C_i,
        \end{equation}
        and we note that every $C_i$ is a bounded open interval.

        We know that $I$ can be finite or countably infinite. If $I$ is finite, then we can arrange $C_i$'s to look like in Lemma 1.10, and the equality [1.20] that is need here will follow from Lemma 1.10. We will thus assume that $I$ is countably infinite.

        Now comes the punchline: we have $K\subseteq A = \bigcup^{}_{i\in I}$, so $\left\lbrace C_i \right\rbrace^{}_{i\in I}$ is an \textit{open cover} of the compact set $K$. Hence by the compactness of $K$, there exists $i_1,\ldots,i_k\in I$ such that $K\subseteq\bigcup^{k}_{j=1} C_{i_j}$. 

        Our next move is then to break $A$ as the disjoint union $A=A'\cup A''$, where
        \begin{equation*}
            A' = \bigcup^{k}_{j=1} C_{i_j} , A'' = \bigcup^{}_{i\in I\setminus \left\lbrace i_1,\ldots,i_k \right\rbrace}C_i.
        \end{equation*}
        Lemma 1.10 applies in connection to the inclusion $K\subseteq A'$, and gives us that
        \begin{equation*}
            \lambda_{\cp}\left( K \right)+\lambda_{\op}\left( A'\setminus K \right) = \lambda_{\op}\left( A' \right).
        \end{equation*}
        So then we can write
        \begin{flalign*}
            && \lambda_{\op}\left( A \right) & = \lambda_{\op}\left( A' \right)+\lambda_{\op}\left( A'' \right) && \\ 
            && & = \lambda_{\cp}\left( K \right)+\lambda_{\op}\left( A'\setminus K \right)+\lambda_{\op}\left( A'' \right) && \\
            && & = \lambda_{\cp}\left( K \right)+\lambda_{\op}\left( A\setminus K \right), && \text{since $K\subseteq A'$ and $A'\cap A''=\emptyset$}
        \end{flalign*}
        as required.
    \end{proof}

    \subsection{Continuity along Decreasing Chain}

    So far we have defined the notion of length for open subsets of $\R$ and for compact subsets of $\R$. These were formalized as some length-measuring maps $\lambda_{\op}:\mT\to\left[ 0,\infty \right]$ and $\lambda_{\cp}:\mK\to\left[ 0,\infty \right)$, where $\mT,\mK$ are the collections of open subsets of $\R$ and compact subsets of $\R$, respectively. We aim to eventually find a collection $\mM$ of subsets of $\R$, called \textit{measurable sets}, such that $\mM\supseteq\mT\cup\mK$, and a length-measuring function $\lambda:\mM\to\left[ 0,\infty \right]$ which extends both $\lambda_{\op}$ and $\lambda_{\cp}$.

    $\mM$ will turn out to be a $\sigma$-algebra, and $\lambda:\mM\to\left[ 0,\infty \right]$ will turn out to be a positive measure, in the sense discussed in Def'n 1.5, 1.6. This positive measure $\lambda$ is the Lebesgue measure on the real line.

    On our way towards constructing measurable sets, we will need to use the following fact, which is stated just in terms of open sets.

    \begin{prop}{Continuity along Decreasing Chains of Open Sets}
        Let $\left( A_{n} \right)^{\infty}_{n=1}$ be a sequence of bounded open subsets of $\R$ such that
        \begin{equation}
            A_1\supseteq A_2\supseteq\cdots
        \end{equation}
        and such that $\bigcap^{\infty}_{n=1}A_n=\emptyset$. Then $\lim_{n\to\infty}\lambda_{\op}\left( A_n \right)=0$.
    \end{prop}

    \rruleline

    \np When we arrive to examine the basic properties of the Lebesgue measure $\lambda:\mM\to\left[ 0,\infty \right]$, we will have a general result about the continuity of $\lambda$ along a decreasing chain. Proposition 1.12 is a special case of that result. But it would not be all right if we stated that result now and used it in order to derive Proposition 1.12 -- that would create a circular argument (it would be a circular argument because we will use Proposition 1.12 in our construction of the Lebesgue measure).

    This means we will have to find a way to prove Proposition 1.12 which is only using facts that we established about $\lambda_{\op}$ so far.

    \np To prove Proposition 1.12, it will be convenient to go for the so-called \textit{contrapositive} of what we are asked to prove. More precisely, instead of Proposition 1.12, we will focus on the following statement.

    \begin{prop}{}
        Let $\left( A_{n} \right)^{\infty}_{n=1}$ be a decreasing chain of bounded open subsets of $\R$. Suppose there exists $c>0$ such that $\lambda_{\op}\left( A_n \right)\geq c$ for every $n\geq 1$. Then $\bigcap^{\infty}_{n=1}A_n\neq\emptyset$.
    \end{prop}

    \rruleline

    \np Proving Proposition 1.13 will give us what we need, because Proposition 1.12 can be reduced to it. Here is the argument for the reduction.

    \begin{boxyproof}{Proof of Proposition 1.12 (assuming that Proposition 1.13 is true)}
        Let $\left( A_{n} \right)^{\infty}_{n=1}$ be a decreasing chain of bounded open subsets of $\R$, such that $\bigcap^{\infty}_{n=1}A_n=\emptyset$. Then for every $n\geq 1$, we have $\lambda_{\op}\left( A_n \right)\geq\lambda_{\op}\left( A_{n+1} \right)$. Hence $\left( \lambda_{\op}\left( A_n \right) \right)^{\infty}_{n=1}$ is a decreasing sequence in $\left[ 0,\infty \right)$, and we know that such a sequece is sure to be convergent to a limit $c\geq 0$. In order to prove Proposition 1.12, we have to show that $c=0$.

        Let us assume, for contradiction, that $c\neq 0$. This means that $c>0$. From the general properties of a decreasing convergent sequence it follows that $\lambda_{\op}\left( A_n \right)$ for every $n\geq 1$. This implies $\bigcap^{\infty}_{n=1}A_n\neq\emptyset$ by Proposition 1.13, so we have a contradiction.

        So the assumption $c\neq 0$ leads to contradiction. Thus $c=0$, as we had to prove.
    \end{boxyproof}

    \np When working on Proposition 1.13, it will be convenient to first address the special case when the open sets $A_n$ considered there are of an \textit{easily tractable} kind, in the sense of the next definition.

    \begin{definition}{\textbf{Easily Tractable} Open Set}
        We will say that an open set $A\subseteq\R$ is \emph{easily tractable}\footnotemark[1] when it is bounded and only has finitely many interval components.
        
        \noindent
        \begin{minipage}{\textwidth}
            \footnotetext[1]{This is a term that was concocted for specific use in our discussion (i.e. this is not a standard term).}
        \end{minipage}
    \end{definition}

    \np Let us elaborate a bit the meaning of this definition. If $A$ is an open subset of $\R$ which is easily tractable, then every interval component of $A$ must be bounded (because $A$ is so), and hense it is an open interval $\left( a,b \right)$, with $a<b$ in $\R$. Let us denote the number of interval components of $A$ by $k$; by suitably ordering these $k$ intervals, we arrive to the following description for $A$: either $A=\emptyset$ or $A$ can be written as
    \begin{equation*}
        A = \bigcup^{k}_{j=1}\left( a_j,b_j \right)
    \end{equation*}
    with $k\geq 1$ and $a_1<b_1\leq\cdots\leq a_k<b_k$ in $\R$. This really is an \textit{easily tractable} description.

    In order to get better acquainted with easily tractable open sets, here is a little exercise.

    \begin{exercise}{}
        Let $A,B$ be easily tractable open subsets of $\R$. Prove that $A\cap B$ is an easily tractable open set as well.
    \end{exercise}

    \begin{proof}
        Clearly $A\cap B$ is bounded.

        Consider decompositions $A=\bigcup^{n}_{j=1} A_j, B=\bigcup^{m}_{k=1} B_k$ of $A,B$ into interval components, respectively. Then note that
        \begin{equation*}
            A\cap B= \bigcup^{n}_{j=1}\bigcup^{m}_{k=1} A_j\cap B_k,
        \end{equation*}
        which is a union of disjoint open intervals. Since there are finitely many $A_j\cap B_k$'s, $A\cap B$ is easily tractable.
    \end{proof}

    \np Consider now the following statement, which is a special case of Proposition 1.13.

    \begin{prop}{}
        Let $\left( A_{n} \right)^{\infty}_{n=1}$ be a decreasing chain of subsets of $\R$, where every $A_n$ is an easily tractable open set. Suppose there exists a constant $c>0$ such that $\lambda_{\op}\left( A_n \right)\geq c$ for every $n\geq 1$. Then $\bigcap^{\infty}_{n=1}A_n\neq\emptyset$.
    \end{prop}

    \rruleline

    \np Our attack on Proposition 1.13 is like this:
    \begin{enumerate}
        \item prove its special case stated in Proposition 1.14; and
        \item prove that the general case of Proposition 1.13 can be reduced to the special case from Proposition 1.14.
    \end{enumerate}
    We first record a general compactness trick that can yield a nonempty intersection for a decreasing chain of open sets. 

    \begin{lemma}{}
        Let $\left( A_{n} \right)^{\infty}_{n=1}$ be a decreasing chain of subsets of $\R$, where every $A_n$ is a bounded nonempty open set. Suppose that
        \begin{equation*}
            \cl\left( A_{n+1} \right)\subseteq A_n
        \end{equation*}
        for all $n\geq 1$. Then $\bigcap^{\infty}_{n=1}A_n\neq\emptyset$.
    \end{lemma}

    \begin{proof}
        Let $K_n=\cl\left( A_n \right)$ for all $n\geq 1$. Then each $K_n$ is closed as a closure of a set, and bounded since $A_n$ is bounded. Hence each $K_n$ is compact. Now, by applying $\cl$ to the chain $A_1\supseteq A_2\supseteq\cdots$, we obtain
        \begin{equation*}
            K_1\supseteq K_2\supseteq\cdots.
        \end{equation*}
        This is a nested sequence of nonempty compact sets, so by the finite intersection property of compact sets, it follows that $\bigcap^{\infty}_{n=1}K_n\neq\emptyset$. Let us then pick $x\in \bigcap^{\infty}_{n=1}K_n$, and observe that
        \begin{equation*}
            x\in K_{n+1} = \cl\left( A_{n+1} \right)\subseteq A_n
        \end{equation*}
        for all $n\geq 1$, which means $x\in A_n$ for all $n\geq 1$. Thus $x\in \bigcap^{\infty}_{n=1}A_n$, and this intersection is therefore a nonempty set.
    \end{proof}

    \np We would like to use Lemma 1.15 towards the proof of Proposition 1.14. But there is a problem: in the framework of Proposition 1.14, we generally do not have strong inclusions $\cl\left( A_{n+1} \right)\subseteq A_n$ of the kind that were taken as hypothesis in Lemma 1.15.

    For illustration, consider the simple example where every $A_n$ is an open interval $A_n=\left( 0,b_n \right)$ with $b_1>b_2>\cdots$, a strictly decreasing sequence of numbers in $\left( 0,\infty \right)$. In order for these $A_n$'s to satisfy the hypothesis of Proposition 1.14, we need to have $b_n\geq c$ for some $c>0$. This ensures that $\bigcap^{\infty}_{n=1}A_n$ contains the interval $\left( 0,c \right)$, and is therefore nonempty. On the other hand, these $A_n$'s do not satisfy the hypothesis of Lemma 1.15: for instance, $\cl\left( A_2 \right)=\left[ 0,b_2 \right]\nsubseteq\left( 0,b_1 \right)=A_1$.

    In the example mentioned in the preceding paragraph, we see that the hypothesis of Lemma 1.15 would actually kick in if we would \textit{trim} a bit the left endpoints of the $A_n$'s. This is the idea which we will follow (and will turn out to work, once we set the things in the right way).

    So then let us make a formal definition for what it means to \textit{trim the endpoints of intervals} for an easily tractable open set.

    \begin{notation}{$\trim_{\epsilon}$}
        Suppose we are given $\epsilon>0$ and an easily tractable open set $A\subseteq\R$ such that $\lambda_{\op}\left( A \right)>\epsilon$. We define a new open set, denoted as $\trim_{\epsilon}\left( A \right)$, as follows: denoting the interval components of $A$ as $\left( a_1,b_1 \right),\ldots,\left( a_k,b_k \right)$, we put
        \begin{equation}
            \trim_{\epsilon}\left( A \right) = \bigcup^{k}_{j=1: b_j-a_j>\frac{\epsilon}{k}} \left( a_j+\frac{\epsilon}{2k},b_j-\frac{\epsilon}{2k} \right).
        \end{equation}
    \end{notation}

    \np In words, the trimmed set $\trim_{\epsilon}\left( A \right)$ is obtained by distributing $\epsilon$ among the $k$ component intervals $\left( a_1,b_1 \right),\ldots.,\left( a_k,b_k \right)$ of $A$, and by trying to trim each of these intervals by a length of $\frac{\epsilon}{k}$. When doing so, for every $j\in\left\lbrace 1,\ldots,k \right\rbrace$ we find one of two possibilities:
    \begin{enumerate}
        \item the length $b_j-a_j$ of $\left( a_j,b_j \right)$ is at most $\frac{\epsilon}{k}$; the interval $\left( a_j,b_j \right)$ is simply removed; or
        \item $b_j-a_j>\frac{\epsilon}{k}$, in which case we shorten the interval $\left( a_j,b_j \right)$ by removing a piece of length $\frac{\epsilon}{2k}$ at each of its ends.
    \end{enumerate}
    We note that the union indicated in [1.23] is always sure to have some sets in it; that is, there exists some index $j\in\left\lbrace 1,\ldots,k \right\rbrace$ for which $b_j-a_j>\frac{\epsilon}{k}$. Indeed, if we had $b_j-a_j\leq \frac{\epsilon}{k}$ for all $j\in\left\lbrace 1,\ldots,k \right\rbrace$, then summing over $j$ in these inequalities would give a contradiction with the assumption that $\lambda_{\op}\left( A \right)>\epsilon$.

    \np The next lemma records some properties of trimmed open sets which follow directly from the definition.

    \begin{lemma}{}
        Let $\epsilon>0$ and let $A\subseteq\R$ be an easily tractable open subset of $\R$ such that $\lambda_{\op}\left( A \right)>\epsilon$. Then $\trim_{\epsilon}\left( A \right)$ is an easily tractable open set with the properties that
        \begin{enumerate}
            \item $\lambda_{\op}\left( \trim_{\epsilon}\left( A \right) \right)\geq\lambda_{\op}\left( A \right)-\epsilon$; and
            \item $\cl\left( \trim_{\epsilon}\left( A \right) \right)\subseteq A$.
        \end{enumerate}
    \end{lemma}

    \begin{proof}
        Let $A=\bigcup^{k}_{j=1}A_j$ be the decomposition of $A$ into interval components, and assume without loss of generality that $A_1,\ldots,A_n$ have lengths more than $\frac{\epsilon}{k}$ but $A_{n+1},\ldots,A_k$ do not. This means
        \begin{equation}
            \trim_{\epsilon}\left( A \right) = \bigcup^{n}_{j=1} \trim_{\frac{\epsilon}{n}}\left( A_j \right).
        \end{equation}
        From [1.24] it is immediate that $\trim_{\epsilon}\left( A \right)$ is an easily tractable open set.

        Now note that
        \begin{flalign*}
            && \lambda_{\op}\left( \trim_{\epsilon}\left( A \right) \right) & = \sum^{n}_{j=1} \lambda_{\op}\left( \trim_{\frac{\epsilon}{n}}\left( A_j \right) \right) = \sum^{n}_{j=1} \left(\lambda_{\op}\left( A_j \right)-\frac{\epsilon}{n}\right) = \left( \sum^{n}_{j=1}\lambda_{\op}\left( A_j \right) \right)-\epsilon && \\
            && & \leq \left( \sum^{k}_{j=1}\lambda_{\op}\left( A_j \right) \right)-\epsilon =  \lambda_{\op}\left( A \right)-\epsilon,
        \end{flalign*}
        where the second equality is by Lemma 1.8. This verifies (a).

        Also note that, given an open interval $\left( a,b \right)$, with $a<b$, and $\eta<b-a$, we have
        \begin{equation}
            \cl\left(\trim_{\eta}\left( \left( a,b \right) \right)\right) = \left[ a+\frac{\eta}{2},b-\frac{\eta}{2} \right]\subseteq \left( a,b \right).
        \end{equation}
        Applying [1.25] to [1.24] gives
        \begin{equation*}
            \cl\left( \trim_{\epsilon}\left( A \right) \right) = \cl\left( \bigcup^{n}_{j=1}\trim_{\frac{\epsilon}{n}}\left( A_j \right) \right) = \bigcup^{n}_{j=1} \cl\left( \trim_{\frac{\epsilon}{n}}\left( A_j \right) \right) \subseteq \bigcup^{n}_{j=1} A_j \subseteq \bigcup^{k}_{j=1}A_j = A.
        \end{equation*}
        This verifies (b).
    \end{proof}

    \begin{boxyproof}{Proof of Proposition 1.14}
        We are given a decreasing chain $\left( A_{n} \right)^{\infty}_{n=1}$ where very $A_n$ is an easily tractable open subset of $\R$, and we are given $c>0$ such that $\lambda_{\op}\left( A-n \right)\geq c$ for every $n\geq 1$. Our goal is to prove that $\bigcap^{\infty}_{n=1}A_n\neq\emptyset$.

        In order to reach the desired conclusion, we will construct a triangular array of sets $\left( A_{m,n} \right)^{}_{n\geq m\geq 1}$. We can think of the $A_{m,n}$'s as sitting on successive rows,as shown below, and we will construct them recursively, row by row.
        \begin{equation}
            \begin{matrix}
                A_{1,1} & A_{1,2} & A_{1,3} & \cdots \\
                        & A_{2,2} & A_{2,3} & \cdots \\
                        &  &  A_{3,3} & \cdots \\
                        & & & \ddots & \\
            \end{matrix}
        \end{equation}
        We will arrange the things such that, for every $m\geq 1$, the sets $A_{m,n}$'s that appear on the $m$th row in [1.26] have the follwing properties
        \begin{enumerate}
            \item for all $n\geq m$, $A_{m,n}$ is an easily tractable open subset of $\R$;
            \item for all $n\geq m$, $\lambda_{\op}\left( A_{m,n} \right)\geq c\left( \frac{1}{2}+\frac{1}{2^m} \right)$; and
            \item $A_{m,m}\supseteq A_{m,m+1}\supseteq\cdots$.
        \end{enumerate}
        So, let us describe how we do the recursive construction of rows in [1.26]. For the top row, with $m=1$, we simply put $A_{1,n}=A_n$ for all $n\geq 1$. This clearly satisfies the conditions listed above, where (b) comes precisely to the hypothesis that $\lambda_{\op}\left( A_n \right)\geq c$ for every $n\geq 1$.

        Now suppose that, for some $m\geq 1$, we have constructed the $m$th row of the array [1.26], in such a way that the listed properties are holding. We then define
        \begin{equation}
            A_{m+1,n} = A_{m,n}\cap\trim_{\frac{c}{2^{m+1}}}\left( A_{m,m} \right)
        \end{equation}
        for all $n\geq m+1$. Note that the set $\trim_{\frac{c}{2^{m+1}}}\left( A_{m,m} \right)$ mentioned in [1.27] is indeed well-defined, since we know that $\lambda_{\op}\left( A_{m,m} \right)\geq c\left( \frac{1}{2}+\frac{1}{2^m} \right) > \frac{c}{2^{m+1}}$ . Moreover, we note that $A_{m+1,n}$ is an easily tranctable open subset of $\R$. This is because each of $A_{m,n}$ and $\trim_{\frac{c}{2^{m+1}}}\left( A_{m,m} \right)$ are easily tractable, and we invoke Exercise 1.6.

        We divide the remaining part of the proof into several claims.
        \begin{itemize}
            \item \textit{Claim 1. The sequence $\left( A_{m+1,n} \right)^{\infty}_{n=m+1}$ defined in [1.27] satisfies (with $m+1$ replacing $m$) the conditions (a), (b), (c).}

                \begin{subproof}
                    Condition (a) was verified just before the statement of the claim. Condition (c) is immediate, since for every $n\geq m+1$, we have
                    \begin{flalign*}
                        && A_{m,n+1}\subseteq A_{m,n} & \implies A_{m,n+1}\cap\trim_{\frac{c}{2^{m=1}}}\left( A_{m,m} \right)\subseteq A_{m,n}\cap\trim_{\frac{c}{2^{m=1}}}\left( A_{m,m} \right) && \\ 
                        && & \implies A_{m+1,n+1}\subseteq A_{m+1,n}.
                    \end{flalign*}
                    We are thus left to verify (b), which says that $\lambda_{\op}\left( A_{m+1},n \right)\geq c\left( \frac{1}{2}+\frac{1}{2^{m+1}} \right)$, for every $n\geq m+1$. In order to get this, we observe the inclusion
                    \begin{equation}
                        A_{m+1,n}\supseteq\trim_{\frac{c}{2^{m+1}}}\left( A_{m,n} \right)
                    \end{equation}
                    for all $n\geq m+1$. The inclusion [1.28] holds because $\trim_{\frac{c}{2^{m+1}}}\left( A_{m,n} \right)$ is contained in each of the two sets that are intersected when we define $A_{m+1,n}$. That is, we have $\trim_{\frac{c}{2^{m+1}}}\left( A_{m,n} \right)\subseteq A_{m,n}$, and we also have $\trim_{\frac{c}{2^{m+1}}}\left( A_{m,n} \right)\subseteq\trim_{\frac{c}{2^{m+1}}}\left( A_{m,m} \right)$, which is obtained by applying the $\trim_{\frac{c}{2^{m+1}}}$ operation in the inclusion $A_{m,n}\subseteq A_{m,m}$.

                    We then apply $\lambda_{\op}$ in the inclusion [1.28] and find that
                    \begin{flalign*}
                        && \lambda_{\op}\left( A_{m+1,n} \right)&\geq\lambda_{\op}\left( \trim_\frac{c}{2^{m+1}}\left( A_{m,n} \right) \right) \geq\lambda_{\op}\left( A_{m,n} \right)-\frac{c}{2^{m+1}}&& \\ 
                        && & \geq c\left( \frac{1}{2}+\frac{1}{2^m} \right)-\frac{c}{2^{m+1}}\geq c\left( \frac{1}{2}+\frac{1}{2^{m+1}} \right),
                    \end{flalign*}
                    exactly as we wanted.\hfill\textit{(Claim 1 is verified)}
                \end{subproof}

            \item \textit{Claim 2. For every $n\geq m+1$, $A_{m+1,n}\subseteq A_{m,n}$ and $\cl\left( A_{m+1,n} \right)\subseteq A_{m,m}$.}

                \begin{subproof}
                    The first inclusion is clear from [1.27]. For the second inclusion, we use the other set in the intersection indicated in [1.27], and we combine that with Lemma 1.16:
                    \begin{equation*}
                        A_{m+1,n}\subseteq\trim_{\frac{c}{2^{m+1}}}\left( A_{m,m} \right)
                    \end{equation*}
                    implies
                    \begin{equation*}
                        \cl\left( A_{m+1,n} \right)\subseteq\cl\left( \trim_{\frac{c}{2^{m=1}}}\left( A_{m,m} \right) \right)\subseteq A_{m,m}.\eqno\text{\textit{(Claim 2 is verified)}}
                    \end{equation*}
                \end{subproof}
        \end{itemize} 

        The reward for working to construct the triangular array of $A_{m,n}$'s now comes with the following observation.
        \begin{itemize}
            \item \textit{Claim 3. The sequence of open sets $\left( A_{m,m} \right)^{\infty}_{m=1}$ satisfy the hypotheses of Lemma 1.15. That is, each $A_{m,m}$ is a bounded open set and $\cl\left( A_{m+1,m+1} \right)\subseteq A_{m,m}$ for all $m\geq 1$.}

                \begin{subproof}
                    Every $A_{m,m}$ is a bounded open set (as a consequence of being easily tractable) and is nonempty, since $\lambda_{\op}\left( A_{m,m} \right)\geq c\left( \frac{1}{2}+\frac{1}{2^m} \right)>0$. Moreover, the second inclusion recorded in Claim 2 says in particular that $\cl\left( A_{m+1,m+1} \right)\subseteq A_{m,m}$ for every $m\geq 1$. Thus all the hypotheses of Lemma 1.15 are being satisfied.\hfill\textit{(Claim 3 is verified)}
                \end{subproof}
        \end{itemize} 

        Lemma 1.15 gives us that $\bigcap^{\infty}_{m=1}A_{m,m}\neq\emptyset$. But for every $m\geq 1$, a repeated use of the first inclusion of Claim 2 gives
        \begin{equation*}
            A_{m,m} \subseteq A_{m-1,m}\subseteq\cdots\subseteq A_{1,m}=A_m.
        \end{equation*}
        Hence if we pick $x\in\bigcap^{\infty}_{m=1}A_{m,m}$. this $x$ will also belong to $\bigcap^{\infty}_{m=1}A_m$; the letter set is therefore nonempty, as we had to show.
    \end{boxyproof}

    \subsection{Lebesgue Measure of Bounded Measurable Sets}

    In this subsection we get acquainted with some of the \textit{Lebesgue measurable} subsets of $\R$ (namely, those that are bounded), and with how \textit{Lebesgue measure} is measuring such sets. More precisely, we will introduce a collection $\mM_{\bdd}$ of (some) bounded subsets of $\R$ and a length-measuring map $\lambda_{\bdd}:\mM_{\bdd}\to\left[ 0,\infty \right)$, such that $\lambda_{\bdd}$ agrees with $\lambda_{\op}$ on $\mT$ (the collection of all open subsets of $\R$) and $\lambda_{\bdd}$ agrees with $\lambda_{\cp}$ on $\mK$ (the collection of all compact subsets of $\R$).

    \begin{definition}{\textbf{Lebesgue Measurable} Bounded Set}
        Let $A\subseteq\R$ be bounded. We say that $A$ is \emph{Lebesgue measurable} if
        \begin{equation}
            \forall\epsilon>0\exists G,K\subseteq\R\left[ G\text{ is bounded and open}, K\text{ is compact}, K\subseteq A\subseteq G, \lambda_{\op}\left( G \right)-\lambda_{\cp}\left( K \right)<\epsilon \right].
        \end{equation}
        That is, given any $\epsilon>0$, we can find bounded open $G\subseteq\R$ and compact $K\subseteq\R$ such that $K\subseteq A\subseteq G$ and that $\lambda_{\op}\left( G \right)-\lambda_{\cp}\left( K \right)<\epsilon$.

        We shall write
        \begin{equation*}
            \mM_{\bdd} = \left\lbrace A\subseteq\R:A\text{ is bounded and Lebesgue measurable} \right\rbrace.
        \end{equation*}
    \end{definition}
    
    \np Note that, in [1.29], $\lambda_{\op}\left( G \right)$ is sure to be a finite quantity, since $G$ is assuemd to be bounded. More formally speaking, there are $a,b\in\R$ such that $G\subseteq\left( a,b \right)$, which means $\lambda_{\op}\left( G \right)=b-a<\infty$.

    \np In fact, the last condition in [1.29], $\lambda_{\op}\left( G \right)-\lambda_{\cp}\left( K \right)<\epsilon$ can be phrased in the form
    \begin{equation}
        \lambda_{\op}\left( G\setminus K \right)<\epsilon.
    \end{equation}
    Indeed, $G\setminus K$ is open, and the definition of $\lambda_{\cp}\left( \cdot \right)$ was made in such a way that we have
    \begin{equation*}
        \lambda_{\cp}\left( K \right)+\lambda_{\op}\left( G\setminus K \right)=\lambda_{\op}\left( G \right).
    \end{equation*}
    Hence the difference $\lambda_{\op}\left( G \right)-\lambda_{\cp}\left( K \right)$ is precisely equal to $\lambda_{\op}\left( G\setminus K \right)$.

    \begin{prop}{}
        Let $A\in\mM_{\bdd}$. Then
        \begin{equation}
            \inf\left\lbrace \lambda_{\op}\left( G \right): \text{$G$ is a bounded open set containing $A$} \right\rbrace = 
            \sup\left\lbrace \lambda_{\op}\left( K \right): \text{$K$ is a compact set contained in $A$} \right\rbrace.
        \end{equation}
    \end{prop}

    \begin{proof}
        Consider the set
        \begin{equation*}
            T = \left\lbrace \lambda_{\cp}\left( K \right): K\text{ is a compact set contained in $A$} \right\rbrace.
        \end{equation*}
        Clearly, $T$ is nonempty -- for instance $0\in T$, as we find by looking at the compact set $K=\emptyset\subseteq A$. We want to argue that $T$ is bounded above. To that end, we make the following observation.

        \begin{itemize}
            \item \textit{Claim 1. Let $G$ be a bounded open subset of $\R$ such that $G\subseteq A$ (such sets are sure to exist, due to the assumption that $A$ is bounded). Then $\lambda_{\op}\left( G \right)$ is an upper bounded for the set $T$.}

                \begin{subproof}
                    We have to show that $\lambda_{\op}\geq t$ for all $t\in T$. Let us fix a $t\in T$ for which we verify this inequality. We pick a compact set $K\subseteq A$ such that $\lambda_{\cp}\left( K \right)=t$, and we argue like this: from $K\subseteq A\subseteq G$, it follows in particular that $K\subseteq G$. We thus find that
                    \begin{equation*}
                        \lambda_{\op}\left( G \right) = \lambda_{\cp}\left( K \right)+\lambda_{\op}\left( G\setminus K \right) = t+\lambda_{\op}\left( G\setminus K \right)\geq t,
                    \end{equation*}
                    as required.\hfill\textit{(Claim 1 is verified)}
                \end{subproof}
        \end{itemize} 

        Due to Claim 1, it makes sense to consider the quantity $\sup\left( T \right)$. Moreover, Claim 1 gives us that
        \begin{equation}
            \sup\left( T \right)\leq\lambda_{\op}\left( G \right)
        \end{equation}
        for any bounded open set containing $A$. 

        Now, let us consider the set of nonnegative numbers
        \begin{equation*}
            S = \left\lbrace \lambda_{\op}\left( G \right): G\text{ is a bounded open set containing $A$} \right\rbrace.
        \end{equation*}
        This is nonempty and bounded below (by $0$), thus it makes sense to consider the quantity $\inf\left( S \right)$. The inequality [1.32] implies
        \begin{equation*}
            \sup\left( T \right)\leq\inf\left( S \right).
        \end{equation*}
        Hence, we are left to prove the opposite inequality
        \begin{equation}
            \inf\left( S \right)\leq\sup\left( T \right).
        \end{equation}
        In order to establish [1.33], we will resort to the old trick of showing that
        \begin{equation}
            \forall n\in\N\left[ \inf\left( S \right)\leq\sup\left( T \right)+\frac{1}{n} \right].
        \end{equation}

        Now, we know that for every $n\in\N$ we can find a compact set $K_n\subseteq A$ and a bounded open set $G_n\subseteq\R$ containing $A$, such that $\lambda_{\op}\left( G_n \right)-\lambda_{\cp}\left( K_n \right)$. This uses the hypothesis on $A$ which is provided in [1.29], with $\frac{1}{n}$ playing the role of $\epsilon$. We thus have:
        \begin{flalign*}
            && \inf\left( S \right)&\leq\lambda_{\op}\left( G_n \right) && \\ 
            && & \leq\lambda_{\cp}\left( K_n \right)+\frac{1}{n} && \\
            && & \leq\sup\left( T \right)+\frac{1}{n},
        \end{flalign*}
        and [1.34] follows.
    \end{proof}

    \clearpage

    \begin{definition}{\textbf{Lebesgue Measure} of a Bounded Lebesgue Measurable Set}
        Let $A\subseteq\R$ be bounded Lebesgue measurable. Then the quantity appearing in either side of [1.31] is called the \emph{Lebesgue measure} of $A$.
    \end{definition}

    \np In our further discussions it will come in handy to know that one can use a sequential approach to the definition of $\mM_{\bdd}$ and the map $\lambda_{\bdd}$, as follows.

    \begin{prop}{}
        Let $A\subseteq\R$ be bounded. Suppose there exist a sequence $\left( G_{n} \right)^{\infty}_{n=1}$ of bounded open subsets of $\R$ and a sequence $\left( K_{n} \right)^{\infty}_{n=1}$ of compact subsets of $\R$ such that $K_n\subseteq A\subseteq G_n$ for all $n\geq 1$, and such that
        \begin{equation}
            \lim_{n\to\infty}\lambda_{\op}\left( G_n \right)-\lambda_{\cp}\left( K_n \right)=0.
        \end{equation}
        Then
        \begin{enumerate}
            \item $A\in\mM_{\bdd}$; and
            \item both $\left( \lambda_{\op}\left( G_n \right) \right)^{\infty}_{n=1}, \left( \lambda_{\cp}\left( K_n \right) \right)^{\infty}_{n=1}$ are convergent, to the same limit $\lambda_{\bdd}\left( A \right)$.
        \end{enumerate}
    \end{prop}

    \begin{proof}
        \begin{enumerate}
            \item Suppose $\epsilon>0$ is given. Then we can find $n\in\N$ such that $\lambda_{\op}\left( G_n \right)-\lambda_{\cp}\left( K_n \right)<\epsilon$ by [1.35]. This means $G_n, K_n$ are such that $G_n$ is bounded and open, $K_n$ is compact, $K_n\subseteq A\subseteq G_n$, and $\lambda_{\op}\left( G_n \right)-\lambda_{\cp}\left( K_n \right)<\epsilon$. Thus $A$ is Lebesgue measurable.

            \item In [1.31] we have a description of $\lambda_{\bdd}\left( A \right)$ in the form of an infimum, which implies that $\lambda_{\op}\left( G_n \right)\geq\lambda_{\bdd}\left( A \right)$ for all $n\geq 1$. In [1.31] we also have a description of $\lambda_{\bdd}\left( A \right)$ in the form of a supremum, which implies that $\lambda_{\cp}\left( K_n \right)\leq\lambda_{\bdd}\left( A \right)$ for every $n\geq 1$. Upon processing a bit these inequalities, we find that
                \begin{equation}
                    \forall n\in\N\left[ \lambda_{\bdd}\left( A \right)\leq\lambda_{\op}\left( G_n \right)\leq\lambda_{\bdd}\left( A \right)+\left( \lambda_{\op}\left( G_n \right)-\lambda_{\cp}\left( K_n \right) \right)\right].
                \end{equation}
                But our hypothesis [1.35] implies that $\lim_{n\to\infty}\lambda_{\bdd}\left( A \right)+\left( \lambda_{\op}\left( G_n \right)-\lambda_{\cp}\left( K-n \right) \right)=\lambda_{\bdd}\left( A \right)$. Hence by applying the squeeze theorem to [1.36], we obtain that $\left( \lambda_{\op}\left( G-n \right) \right)^{\infty}_{n=1}$ is convergent, with limit equal to $\lambda_{\bdd}\left( A \right)$.

                Finally, we write
                \begin{equation}
                    \forall n\in\N\left[ \lambda_{\cp}\left( K_n \right)=\lambda_{\op}\left( G_n \right)-\left( \lambda_{\op}\left( G_n \right)-\lambda_{\cp}\left( K_n \right) \right) \right].
                \end{equation}
                By letting $n\to\infty$ in [1.37] we find that $\lim_{n\to\infty}\lambda_{\cp}\left( K_n \right)$ exists and is equal to $\lambda_{\bdd}\left( A \right)$ as well.
        \end{enumerate}
    \end{proof}

    \begin{definition}{$G_{\delta}$ Set}
        A set $A\subseteq\R$ is said to be $G_{\delta}$ when it is a countable union of open subsets of $\R$.
    \end{definition}

    \np Here are some basic properties of $G_{\delta}$ sets.
    \begin{enumerate}
        \item If $A\subseteq\R$ is $G_{\delta}$, then it can be written as $A=\bigcap^{\infty}_{n=1}G_n$ where the $G_n$'s are open and are also assumed to form a decreasing chain $G_1\supseteq G_2\supseteq\cdots$. This fact has an immediate proof. Start with some arbitrary writing $A=\bigcap^{\infty}_{n=1}U_n$ where $U_n$'s are open, and put $G_1=U_1, G_2=U_1\cap U_2,\ldots$. Then the $G_n$'s are open with $\bigcap^{\infty}_{n=1}G_n=A$, and will also form a decreasing chain.

        \item Every open $G\subseteq\R$ is $G_{\delta}$. This is clear, as we can write $G=\bigcap^{\infty}_{n=1}G_N$ where we put $G_n=G$ for every $n\in\N$.

        \item Every closed $F\subseteq\R$ is $G_{\delta}$. The standard method used to write nonempty closed $F\subseteq\R$ as a countable intersection of open sets $\bigcap^{\infty}_{n=1}G_n$ is by putting
            \begin{equation}
                G_n=\bigcup^{}_{x\in F}\left( x-\frac{1}{n},x+\frac{1}{n} \right)
            \end{equation}
            for all $n\in\N$.

            The following observation can be made in connection to the formula [1.38]: note that if $F$ is bounded (hence compact), then each open set $G_n$ provided by [1.38] are bounded as well.
    \end{enumerate}
    
    \begin{definition}{$F_{\sigma}$ Set}
        A set $B\subseteq\R$ is said to be $F_{\sigma}$ when it can be written as a countable union of closed subsets of $\R$.
    \end{definition}

    \np Recall that a subset of $\R$ is $F_{\sigma}$ if and only if its complement is $G_{\delta}$. Based on this, we obtain the counterparts of the properties of $G_{\delta}$ sets that we listed above.
    \begin{enumerate}
        \item If $B\subseteq\R$ is $F_{\sigma}$, then it can be written as $B=\bigcup^{\infty}_{n=1}F_n$ where the $F_n$'s are closed and are also assumed and also assumed to form an increasing chain $F_1\subseteq F_2\subseteq\cdots$.
        \item Every closed set is $F_{\sigma}$.
        \item Every open set is $F_{\sigma}$.
    \end{enumerate}

    \np Now back to our goal, of proving that the length-leasuring function $\lambda_{\bdd}$ on $\mM_{\bdd}$ fits with the $\lambda_{\op},\lambda_{\cp}$ from the preceding lectures.

    \begin{prop}{}
        Let $G$ be a bounded open subset of $\R$. Then $G\in\mM_{\bdd}$ and $\lambda_{\bdd}=\lambda_{\op}\left( G \right)$.
    \end{prop}

    \begin{proof}
        We know that $G$ is $F_{\sigma}$, hence we can write $G=\bigcup^{\infty}_{n=1}F_n$ where $\left( F_{n} \right)^{\infty}_{n=1}$ is an increasing chain of closed subsets of $\R$. From the fact that $G$ is bounded it follows that every $F_n$ is bounded as well -- hence $F_n$ is a compact set. Let us also record the observation that the set-differences $\left( G\setminus F_{n} \right)^{\infty}_{n=1}$ form a decreasing chain of bounded open sets, with
        \begin{equation}
            \bigcap^{\infty}_{n=1}G\setminus F_n = G\setminus \left( \bigcup^{\infty}_{n=1}F_n \right)G\setminus G=\emptyset.
        \end{equation}
        Quite importantly, we can invoke the continuity of $\lambda_{\op}$ along decreasing chains of open sets (Proposition 1.12) in connection to [1.39], so that
        \begin{equation}
            \lim_{n\to\infty}\lambda_{\op}\left( G\setminus F_n \right)=0.
        \end{equation}
        We can then put $K_n=F_n, G_n=G$ for all $n\in\N$, and this will gives a sequence of compact sets $\left( K_{n} \right)^{\infty}_{n=1}$ and a sequence of bounded open sets $\left( G_{n} \right)^{\infty}_{n=1}$ such that $K_n\subseteq G\subseteq G_n$ for all $n\in\N$, and such that
        \begin{equation*}
            \lim_{n\to\infty}\lambda_{\op}\left( G_n \right)-\lambda_{\cp}\left( K_n \right) = \lim_{n\to\infty}\lambda_{\op}\left( G_n\setminus K_n \right) = \lim_{n\to\infty}\lambda_{\op}\left( G\setminus F_n \right)=0.
        \end{equation*}
        Hence the criterion from Proposition 1.18 can be applied, so that $G\in\mM_{\bdd}$ and that $\lambda_{\bdd}\left( G \right)=\lim_{n\to\infty}\lambda_{\op}\left( G_n \right)$. But $\lambda_{\op}\left( G_n \right)=\lambda_{\op}\left( G \right)$ for all $n\in\N$, hence we arrive to the desired conclusion that $\lambda_{\bdd}\left( G \right)=\lambda_{\op}\left( G \right)$.
    \end{proof}

    \np In the same vein as for the preceding proposition, we have the following.

    \begin{prop}{}
        Let $K\subseteq\R$ be compact. Then $K\in\mM_{\bdd}$ and $\lambda_{\bdd}\left( K \right)=\lambda_{\cp}\left( K \right)$.
    \end{prop}

    \placeqed[Exercise]

    \np When looking for an example of a bounded subset of $\R$ which is neither open nor compact, one of the first candidates that comes to mind is $A=\Q\cap\left[ 0,1 \right]$. Is this Lebesgue measurable? We can show right away that it is. In fact, the rule of thumb is that all the \textit{natural} subsets of $\R$ are Lebesgue measurable; as we will see, it takes some effort to provide an example of non-measurable set.

    To be precise, what happens is that $\Q\cap\left[ 0,1 \right]\in\mM_{\bdd}$ and $\lambda_{\bdd}\left( Q\cap\left[ 0,1 \right] \right)=0$. The sets of this kind are called \textit{negligible}. One can detect them by using the following criterion.

    \begin{prop}{}
        Let $A\subseteq\R$ be bounded. Suppose there exists a sequence $\left( G_{n} \right)^{\infty}_{n=1}$ of bounded open subsets of $\R$ such that $A\subseteq G_n$ for all $n\in\N$, and such that $\lim_{n\to\infty}\lambda_{\op}\left( G_n \right)=0$. Then it follows that $A\in\mM_{\bdd}$ and $\lambda_{\bdd}\left( A \right)=0$.
    \end{prop}

    \begin{proof}
        This follows from Proposition 1.18 where we use the given $G_n$'s and also consider $K_n=\emptyset$ for all $n\in\N$.
    \end{proof}

    \clearpage

    \np By using Proposition 1.21, it is easy to check that $\Q\cap\left[ 0,1 \right]$ is negligible. This actually happens just because we are dealing with a countable set. So instead of $\Q\cap\left[ 0,1 \right]$, let us consider any bounded countable subset $A\subseteq\R$ in the following exercise.

    \begin{exercise}{}
        Let $A\subseteq\R$ be bounded and countable.
        \begin{enumerate}
            \item Prove that for every $\epsilon>0$, there is bounded open $G\subseteq\R$ containing $A$ with $\lambda_{\op}\left( G \right)<\epsilon$.
            \item By using (a) and Proposition 1.21, prove that $A$ is negligible. That is, $A\in\mM_{\bdd}$ and $\lambda_{\bdd}\left( A \right)=0$.
        \end{enumerate}
    \end{exercise}

    \begin{proof}
        \begin{enumerate}
            \item Since $A$ is countable, fix an enumeration of the elements of $A$, say $\left\lbrace a_n \right\rbrace^{}_{n\in\N}$. Now consider the union
                \begin{equation*}
                    G = \bigcup^{}_{n\in\N} \left( a_n-\frac{\epsilon}{2^{n+2}}, a_n+\frac{\epsilon}{2^{n+2}} \right)
                \end{equation*}
                which contains $A$. For each $n\in\N$, the interval $\left( a_n-\frac{\epsilon}{2^{n+2}}, a_n+\frac{\epsilon}{2^{n+2}} \right)$ has length $\frac{\epsilon}{2^{n+1}}$, so that
                \begin{equation*}
                    \lambda_{\op}\left( G \right) \leq \sum^{}_{n\in\N} \frac{\epsilon}{2^{n+1}} = \frac{\epsilon}{2} < \epsilon.
                \end{equation*}

            \item For each $n\in\N$, let $G_n$ be a bounded open subset of $\R$ containing $A$ with $\lambda_{\op}\left( G_n \right)<\frac{1}{n}$, which exists by (a). Then $\lim_{n\to\infty}\lambda_{\op}\left( G_n \right) = \lim_{n\to\infty} \frac{1}{n} = 0$, so that we can invoke Proposition 1.21 to conclude that $A\in\mM_{\bdd}$ and that $\lambda_{\bdd}\left( A \right)=0$.
        \end{enumerate}
    \end{proof}

    \subsection{Properties of $\mM_{\bdd}$ and of $\lambda_{\bdd}$}
    
    We identified a collection $\mM_{\bdd}$ of \textit{bounded measurable subsets} of $\R$. Moreover, we introduced a length-measuring function $\lambda_{\bdd}:\mM_{\bdd}\to\left[ 0,\infty \right)$, and we found that
    \begin{enumerate}
        \item every bounded open subset $A$ of $\R$ belongs to $\mM_{\bdd}$ with $\lambda_{\bdd}\left( A \right) = \lambda_{\op}\left( A \right)$; and
        \item every compact subset $K$ of $\R$ belongs to $\mM_{\bdd}$ with $\lambda_{\bdd}\left( K \right) = \lambda_{\cp}\left( K \right)$.
    \end{enumerate}
    We will see that $\mM_{\bdd}$ is closed under certain set-operations. More precisely, we will establish that bounded measurable subsets of $\R$ forms what is called a \textit{ring} of subsets of $\R$. We will also observe an additivity property of the length-measuring function $\lambda_{\bdd}$>

    Recall that we looked at the collection $\mT$ of open subsets of $\R$, and we discovered the shortcoming that $\mT$ is not closed under set-difference. One of the things we will find is that $\mM_{\bdd}$ is closed under set-difference:
    \begin{equation*}
        \forall A_1,A_1\in\mM_{\bdd}\left[ A_1\setminus A_2\in\mM_{\bdd} \right].
    \end{equation*}
    This is a step towards resolving our issue about set-differences of open sets: if $G_1,G_2$ are bounded open sets, then $G_1\setminus G_2$ may no longer be open, but it will still be in $\mM_{\bdd}$. This provides us with a notion of length for $G_1\setminus G_2$, namely it is the number $\lambda_{\bdd}\left( G_1\setminus G_2 \right)$.

    \begin{definition}{\textbf{Ring} of Sets}
        Let $X$ be a nonempty set and let $\mR$ be a collection of subsets of $X$. We say $\mR$ is a \emph{ring} of subsets of $X$ to mean that
        \begin{enumerate}
            \item $\emptyset\in\mR$;
            \item for all $A_1,A_2\in\mR$, $A_1\cup A_2\in\mR$; and\hfill\textit{closure under union}
            \item for all $A_1,A_2\in\mR$, $A_1\setminus A_2\in\mR$.\hfill\textit{closure under set-difference}
        \end{enumerate}
    \end{definition}

    Let $X$ be a nonempty set and let $\mR$ be a ring of subsets of $X$.
    \begin{enumerate}
        \item $\mR$ is closed under intersection. This is found by writing that
            \begin{equation*}
                A_1\cap A_2 = A_1\setminus \left( A_1\setminus A_2 \right)
            \end{equation*}
            for any $A_1,A_2\in\mR$.
        \item By induction, $\mR$ is closed under any finite union or intersection.
    \end{enumerate}
    In conclusion, we can do any kind of set-operations we want, involving \textit{finitely many} sets from $\R$, and we can be certain that the result of these operations will still belong to $\mR$.

    \np In the process of constructing the Lebesgue measure we constantly have to fall back on elementary statements about the length-measuring function $\lambda_{\op}$ for open sets. This happens because $\lambda_{\op}$ is the fundamental notion we are building on.

    \begin{exercise}{}
        Consider a bounded open interval $\left( a,b \right)$ with $a<b$ in $\R$. Suppose $G_1,G_2$ are open subsets of $\R$ such that $G_1\cup G_2=\left( a,b \right)$. Prove that
        \begin{equation*}
            b-a \leq \lambda_{\op}\left( G_1 \right)+\lambda_{\op}\left( G_2 \right).
        \end{equation*}
    \end{exercise}

    \begin{proof}
        It suffices to check that
        \begin{equation*}
            b'-a' \leq \lambda_{\op}\left( G_1 \right)+\lambda_{\op}\left( G_2 \right)
        \end{equation*}
        for all $a',b'\in\R$ such that $a<a'<b'<b$. So choose such $a',b'$. Now note that $\left[ a',b' \right]\subseteq\left( a,b \right)=G_1\cup G_2$. This means, if we consider the decompositions of $G_1,G_2$ into interval components,
        \begin{equation*}
            G_1 = \bigcup^{}_{i\in I}G_{1,i}, G_2 = \bigcup^{}_{j\in J}G_{2,j},
        \end{equation*}
        respectively, then $\left( \bigcup^{}_{i\in I}G_{1,i} \right)\cup\left( \bigcup^{}_{j\in J}G_{2,j} \right)$ is (the union of) an open cover for $\left[ a',b' \right]$. Since $\left[ a',b' \right]$ is compact, we can choose a finite subcover $\left\lbrace G_{1,t} \right\rbrace^{n}_{t=1}\cup\left\lbrace G_{2,s} \right\rbrace^{m}_{s=1}$. This means
        \begin{equation*}
            b'-a' = \lambda_{\cp}\left( \left[ a',b' \right] \right) \leq \lambda_{\op}\left( \left( \bigcup^{n}_{t=1}G_{1,t} \right)\cup\left( \bigcup^{n}_{s=1}G_{2,s} \right) \right) \leq \left( \sum^{n}_{t=1}\lambda_{\op}\left( G_{1,t} \right) \right) + \left( \sum^{m}_{s=1}\lambda_{\op}\left( G_{2,s} \right) \right) \leq \lambda_{\op}\left( G_1 \right)+\lambda_{\op}\left( G_2 \right),
        \end{equation*}
        as required.
    \end{proof}

    \begin{prop}{}
        Let $G_1,G_2$ be two bounded open subsets of $\R$ and consider their union $G_1\cup G_2$, which is a bounded open set as well. Then
        \begin{equation*}
            \lambda_{\op}\left( G_1\cup G_2 \right)\leq\lambda_{\op}\left( G_1 \right)+\lambda_{\op}\left( G_2 \right).
        \end{equation*}
    \end{prop}

    \begin{proof}
        Consider the decomposition into interval components for the open sets $G_1\cup G_2$. Since $G_1\cup G_2$ is bounded, this decomposition has the form
        \begin{equation*}
            G_1\cup G_2 = \bigcup^{}_{i\in I}C_i
        \end{equation*}
        with for all $i\in I$, $C_i=\left( a_i,b_i \right)$ for some $a_i,b_i\in\R$. For every $i\in I$, we look at the open sets
        \begin{equation*}
            G_i^{\left( 1 \right)} = G_1\cap C_i, G_i^{\left( 2 \right)} = G_2\cap C_i.
        \end{equation*}
        Observe that $G_i^{\left( 1 \right)}\cup G_{i}^{\left( 2 \right)} = \left( G_1\cap C_i \right)\cup\left( G_2\cap C_i \right)=\left( G_1\cup G_2 \right)\cap C_i = C_i = \left( a_i,b_i \right)$, so Exercise 1.8 applies to this situation and gives us the inequality
        \begin{equation}
            \lambda_{\op}\left( G_i^{\left( 1 \right)} \right)+\lambda_{\op}\left( G_i^{\left( 2 \right)} \right)\geq b_i-a_i
        \end{equation}
        for all $i\in I$. But we know that $I$ is countable and quantities $\lambda_{\op}\left( G_i^{\left( 1 \right)} \right),\lambda_{\op}\left( G_i^{\left( 2 \right)} \right),b_i-a_i$ are nonnegative, so that [1.41] implies
        \begin{equation}
            \left( \sum^{}_{i\in I}\lambda_{\op}\left( G_i^{\left( 1 \right)} \right) \right) + \left( \sum^{}_{i\in I}\lambda_{\op}\left( G_i^{\left( 2 \right)} \right) \right) \geq \sum^{}_{i\in I} \left( b_i-a_i \right).
        \end{equation}
        
        We are left to observe what is the right interpretation of the three sums over $I$ that have appeared in [1.42]. First of all, the very definition of $\lambda_{\op}$ gives us that
        \begin{equation}
            \lambda_{\op}\left( G_1\cup G_2 \right)=\sum^{}_{i\in I}\left( b_i-a_i \right).
        \end{equation}
        Hence the right-hand side of [1.42] we have the length $\lambda_{\op}\left( G_1\cup G_2 \right)$.

        Let us next focus on the collection of open sets $\left\lbrace G_i^{\left( 1 \right)} \right\rbrace^{}_{i\in I}$. We observe that these sets are pairwise disjoint, since for every distinct $i,j\in I$ we have
        \begin{equation*}
            G_i^{\left( 1 \right)}\cap G_j^{\left( 1 \right)} = \left( G_1\cap C_i \right)\cap \left( G_1\cap C_j \right) = G_1\cap \left( C_i\cap C_j \right) = G_1\cap\emptyset = \emptyset.
        \end{equation*}
        So we can use on this collection Proposition 1.2, which says that
        \begin{equation*}
            \sum^{}_{i\in I}\lambda_{\op}\left( G_i^{\left( 1 \right)} \right)=\lambda_{\op}\left( \bigcup^{}_{i\in I}G_i^{\left( 1 \right)} \right).
        \end{equation*}
        But
        \begin{equation*}
            \bigcup^{}_{i\in I}G_i^{\left( 1 \right)} = \bigcup^{}_{i\ni I}\left( G_1\cap C_i \right)=G_1\cap\left( \bigcup^{}_{i\in I}C_i \right)=G_1\cap\left( G_1\cup G_2 \right)=G_1.
        \end{equation*}
        Hence the conclusion we draw here is that we have
        \begin{equation}
            \sum^{}_{i\in I}\lambda_{\op}\left( G_i^{\left( 1 \right)} \right) = \lambda_{\op}\left( G_1 \right).
        \end{equation}
        A similar calculation yields the formula
        \begin{equation}
            \sum^{}_{i\in I}\lambda_{\op}\left( G_i^{\left( 2 \right)} \right) = \lambda_{\op}\left( G_2 \right).
        \end{equation}
        Combining [1.42], [1.43], [1.44], [1.45] gives the desired result.
    \end{proof}

    \begin{cor}{Subadditivity of $\lambda_{\op}$}
        For any $k\in\N$ and any bounded open sets $G_1,\ldots,G_k\subseteq\R$, we have
        \begin{equation*}
            \lambda_{\op}\left( \bigcup^{k}_{j=1}G_j \right) \leq \sum^{k}_{j=1}\lambda_{\op}\left( G_j \right).
        \end{equation*}
    \end{cor}	

    \placeqed[Use Induction!]

    \np We now turn to the verification that $\mM_{\bdd}$ is a ring of subsets of $\R$. That is, we verify that for every $A_1,A_2\in\mM_{\bdd}$, both $A_1\cup A_2$ and $A_1\setminus A_2$ are in $\mM_{\bdd}$ as well. 

    \begin{lemma}{}
        Let $K_1,\ldots,K_n\subseteq\R$ be compact and let $G_1,\ldots,G_n\subseteq\R$ be bounded and open such hat $K_j\subseteq G_j$ for all $j\in\left\lbrace 1,\ldots,n \right\rbrace$. Consider the compact set $K=\bigcup^{n}_{j=1}K_j$ and the bounded open set $G=\bigcup^{n}_{j=1}G_j$, where $K\subseteq G$. Then
        \begin{enumerate}
            \item $G\setminus K\subseteq\bigcup^{n}_{j=1}G_j\setminus K_j$; and
            \item $\lambda_{\op}\left( G\setminus K \right)\leq\sum^{n}_{j=1}\lambda_{\op}\left( G_j\setminus K_j \right)$.
        \end{enumerate}
    \end{lemma}

    \begin{proof}
        \begin{enumerate}
            \item We have to prove that every point $x\in G\setminus K$ belongs to the union $\bigcup^{n}_{j=1}G_j\setminus K_j$. So pick a point $x\in G\setminus K$. We have that $x\in G$ and $x\notin K$. The latter condition amounts to $x\notin K_1\cup\cdots\cup K_n$, and is thus saying that $x\notin K_j$ for all $j\in\left\lbrace 1,\ldots,n \right\rbrace$. On the other hand, since $x\in G=\bigcup^{n}_{j=1}G_j$, there exists an $j_0\in\left\lbrace 1,\ldots,n \right\rbrace$ such that $x\in G_{j_0}$. For this $j_0$ we find that $x\in G_{j_0}$ and $x\notin K_{j_0}$, hence that
                \begin{equation*}
                    x\in G_{j_0}\setminus K_{j_0} \subseteq \bigcup^{n}_{j=1}G_j\setminus K_j.
                \end{equation*}
            \item We have that $\lambda_{\op}\left( G\setminus K \right)\leq\lambda_{\op}\left( \bigcup^{n}_{j=1}G_j\setminus K_j \right)\leq\sum^{n}_{j=1}\lambda_{\op}\left( G_j\setminus K_j \right)$.
        \end{enumerate}
    \end{proof}

    \clearpage
    \begin{prop}{$\mM_{\bdd}$ Is Closed under Union}
        Let $A_1,A_2\in\mM_{\bdd}$. Then $A_1\cup A_2\in\mM_{\bdd}$.
    \end{prop}

    \begin{proof}
        We need to show that for every $\epsilon>0$, there exists a compact set $K\subseteq\R$ and a bounded open set $G\subseteq\R$ such that
        \begin{equation}
            K\subseteq A_1\cup A_2\subseteq G, \lambda_{\op}\left( G\setminus K \right)<\epsilon.
        \end{equation}
        So fix $\epsilon>0$. Using the hypothesis that $A_1,A_2\in\mM_{\bdd}$ we can find compact sets $K_1,K_2\subseteq\R$ and bounded open sets $G_1,G_2\subseteq\R$ such that
        \begin{equation}
            K_j\subseteq A_j\subseteq G_j, \lambda_{\op}\left( G_j\setminus K_j \right)<\frac{\epsilon}{2}
        \end{equation}
        for all $j\in\left\lbrace 1,2 \right\rbrace$. Unsurprisingly, we now put $K=K_1\cup K_2$ and $G=G_1\cup G_2$. It is clear that $K$ is compact and $G$ is bounded and open, with $K\subseteq A_1\cup A_2\subseteq G$. Moreover, Lemma 1.23 implies that
        \begin{equation*}
            \lambda_{\op}\left( G\setminus K \right)\leq\lambda_{\op}\left( G_1\setminus K_1 \right)+\lambda_{\op}\left( G_2\setminus K_2 \right)<\frac{\epsilon}{2}+\frac{\epsilon}{2} = \epsilon.
        \end{equation*}
        Thus [1.46] is obtained.
    \end{proof}

    \begin{prop}{$\mM_{\bdd}$ Is Closed under Set-difference}
        Let $A_1,A_2\in\mM_{\bdd}$. Then $A_1\setminus A_2\in\mM_{\bdd}$.
    \end{prop}

    \begin{proof}[Proof Sketch]
        We follow the same plan as in the proof of Proposition 1.24. Given an $\epsilon>0$, we find $K_1,K_2,G_1,G_2$ as indicated in [1.47], and we now look at the inclusion
        \begin{equation*}
            K_1\setminus G_2\subseteq A_1\setminus A_2\subseteq G_1\setminus K_2.
        \end{equation*}
        We have thus caught $A_1\setminus A_2$ in between the compact set $K'=K_1\setminus G_2$ and the bounded open set $G'=G_2\setminus K_1$. Then by playing out the analogue of the verifications form Lemma 1.23, we find that
        \begin{equation*}
            G'\setminus K'\subseteq\left( G_1\setminus K_1 \right)\cup\left( G_2\setminus K_2 \right),
        \end{equation*}
        with the consequence that $\lambda_{\op}\left( G'\setminus K' \right)\leq\lambda_{\op}\left( G_1\setminus K_1 \right)+\lambda_{\op}\left( G_2\setminus K_2 \right)<\epsilon$.
    \end{proof}

    \begin{cor}{$\mM_{\bdd}$ Is a Ring of Subsets of $\R$}
        $\mM_{\bdd}$ is a ring of subsets of $\R$.
    \end{cor}	

    \placeqed[Do We Need a Proof?]

    \np We now put into evidence an addition to Proposition 1.24, which kicks in if we make the extra assumption that the sets $A_1,A_2\in\mM_{\bdd}$ we are dealing with are disjoint. Here again there is an elementary statement to fall back one -- about lengths of compact sets, this time.

    \begin{exercise}{}
        Let $K_1,K_2$ be compact subsets of $\R$ such that $K_1\cap K_2=\emptyset$. Consider the union $K_1\cup K_2$, which is a compact set as well. Prove that $\lambda_{\cp}\left( K_1\cup K_2 \right)=\lambda_{\cp}\left( K_1 \right)+\lambda_{\cp}\left( K_2 \right)$.
    \end{exercise}

    \begin{proof}
        We can choose an interval $\left( a,b \right)$ that contains both $K_1,K_2$, so that we may write
        \begin{equation*}
            \begin{aligned}
                \lambda_{\cp}\left( K_1 \right) & = \left( b-a \right)-\lambda_{\op}\left( \left( a,b \right)\setminus K_1 \right) \\
                \lambda_{\cp}\left( K_1\cup K_2 \right) & = \left( b-a \right)-\lambda_{\op}\left( \left( a,b \right)\setminus \left( K_1\cup K_2 \right) \right)
            \end{aligned} .
        \end{equation*}
        We remark that the above discussion makes sense due to Lemma 1.9. But note that, this means
        \begin{flalign*}
            && \lambda_{\cp}\left( K_1\cup K_2 \right) & = \left( b-a \right) - \lambda_{\op}\left( \left( a,b \right)\setminus \left( K_1\cup K_2 \right) \right) = \left( b-a \right) - \lambda_{\op}\left( \left(\left( a,b \right)\setminus K_1\right)\setminus K_2 \right)&& \\
            && & = \left( b-a \right) - \left( \lambda_{\op}\left( \left( a,b \right)\setminus K_1 \right) - \lambda_{\cp}\left( K_2 \right) \right) = \left( \left( b-a \right)-\lambda_{\op}\left( \left( a,b \right)\setminus K_1 \right) \right)+\lambda_{\cp}\left( K_2 \right) = \lambda_{\cp}\left( K_1 \right)+\lambda_{\cp}\left( K_2 \right).
        \end{flalign*}
    \end{proof}

    \clearpage
    \begin{prop}{}
        Let $A_1,A_2\in\mM_{\bdd}$ be disjoint. Then
        \begin{equation*}
            \lambda_{\bdd}\left( A_1\cup A_2 \right) = \lambda_{\bdd}\left( A_1 \right)+\lambda_{\bdd}\left( A_2 \right).
        \end{equation*}
    \end{prop}

    \begin{proof}
        We use the sequential approach to $\lambda_{\bdd}$ stated in Proposition 1.18.

        For every $n\in\N$, let us consider some compact sets $K_n^{\left( 1 \right)},K_n^{\left( 2 \right)}$ and bounded open sets $G_n^{\left( 1 \right)},G_n^{\left( 2 \right)}$ such that
        \begin{equation*}
            K_n^{\left( j \right)}\subseteq A\subseteq G_n^{\left( j \right)}, \lambda_{\op}\left( G_n^{\left( j \right)}\setminus K_n^{\left( j \right)} \right)<\frac{1}{n}
        \end{equation*}
        for all $j\in\left\lbrace 1,2 \right\rbrace$. Then by Proposition 1.18,
        \begin{equation}
            \lim_{n\to\infty}\lambda_{\cp}\left( K_n^{\left( j \right)} \right)=\lambda_{\bdd}\left( A_j \right)
        \end{equation}
        for all $j\in\left\lbrace 1,2 \right\rbrace$. Proceeding exactly as in the proof of Proposition 1.24, we find that for all $n\in\N$, we have
        \begin{equation*}
            K_n\subseteq A_1\cup A_2\subseteq G_n, \lambda_{\op}\left( G_n\setminus K_n \right)<\frac{2}{n},
        \end{equation*}
        where $K_n=K_n^{\left( 1 \right)}\cup K_n^{\left( 2 \right)}$ and $G_n=G_n^{\left( 1 \right)}\cup G_n^{\left( 2 \right)}$. By invoking Proposition 1.18 again,
        \begin{equation}
            \lim_{n\to\infty}\lambda_{\cp}\left( K_n \right) = \lambda_{\bdd}\left( A_1\cup A_2 \right).
        \end{equation}
        Now by using the hypothesis $A_1\cap A_2=\emptyset$, we note that for all $n\in\N$, $K_n^{\left( 1 \right)}\cap K_n^{\left( 2 \right)}=\emptyset$ as well, so that
        \begin{equation*}
            \lambda_{\cp}\left( K_n^{\left( 1 \right)}\cup K_n^{\left( 2 \right)} \right) = \lambda_{\cp}\left( K_n^{\left( 1 \right)} \right)=\lambda_{\cp}\left( K_n^{\left( 2 \right)} \right)
        \end{equation*}
        by using Exercise 1.9. This implies that
        \begin{equation}
            \lim_{n\to\infty}\lambda_{\cp}\left( K_n \right) = \lim_{n\to\infty}\lambda_{\cp}\left( K_n^{\left( 1 \right)}\cup K_n^{\left( 2 \right)} \right) = \lim_{n\to\infty}\lambda_{\cp}\left( K_n^{\left( 1 \right)} \right)+\lambda_{\cp}\left( K_n^{\left( 2 \right)} \right) = \lambda_{\bdd}\left( A_1 \right)+\lambda_{\bdd}\left( A_2 \right)
        \end{equation}
        by using [1.48]. Combining [1.48], [1.50] gives the desired equality.
    \end{proof}

    \subsection{Measurable Subsets of $\R$}
    
    We now take the step of introducing the full $\sigma$-algebra $\mM$ of (possibly unbounded) Lebesgue measurable subsets of $\R$. Recall that we examined the length-measuring function $\lambda_{\op}:\mT\to\left[ 0,\infty \right]$, and we found it to have some good properties. But, nevertheless, $\mT$ is not a $\sigma$-algebra, hence $\lambda_{\op}$ is not a \textit{positive measure} used in Lebesgue integration theory. We are now about to get the fix for that: $\mT$ will be included in $\mM$, which is a $\sigma$-algebra, and $\lambda_{\op}$ will be extended to a positive measure $\lambda:\mM\to\left[ 0,\infty \right]$.

    We introduce the Lebesgue measure $\lambda:\mM\to\left[ 0,\infty \right]$ as an upgrade of the length-measuring function $\lambda_{\bdd}:\mM_{\bdd}\to\left[ 0,\infty \right)$ studied during the preceding couple of subsections. More precisely, we go as follows.

    \begin{definition}{\textbf{Lebesgue Measurable} Set, \textbf{Lebesgue Measure} of a Lebesgue Measurable Set}
        Let $M\subseteq\R$. If $M\cap A\in\mM_{\bdd}$ for all $A\in\mM_{\bdd}$, then we say $M$ is \emph{Lebesgue measurable}. We shall denote the collection of Lebesgue measurable sets by $\mM$.

        For a Lebesgue measurable set $M\in\mM$, we define
        \begin{equation*}
            \lambda\left( M \right) = \sup_{A\in\mM_{\bdd}}\lambda_{\bdd}\left( M\cap A \right),
        \end{equation*}
        which we call the \emph{Lebesgue measure} of $M$.
    \end{definition}

    \np The first thing we check is that the map $\lambda:\mM\to\left[ 0,\infty \right]$ introduced in Def'n 1.16 extends of the map $\lambda_{\bdd}:\mM_{\bdd}\to\left[ 0,\infty \right)$.

    \clearpage
    \begin{prop}{}
        $\mM\supseteq\mM_{\bdd}$.
    \end{prop}

    \begin{proof}
        Given $M\in\mM_{\bdd}$, $M\cap A\in\mM_{\bdd}$ for all $A\in\mM_{\bdd}$ since $\mM_{\bdd}$ is a ring of sets.
    \end{proof}

    \begin{prop}{}
        For all $M\in\mM_{\bdd}$, $\lambda\left( M \right)=\lambda_{\bdd}\left( M \right)$.
    \end{prop}

    \begin{proof}
        Let $M\in\mM_{\bdd}$. We have to check that
        \begin{equation}
            \sup_{A\in\mM_{\bdd}}\lambda_{\bdd}\left( M\cap A \right) = \lambda_{\bdd}\left( M \right).
        \end{equation}
        We prove [1.51] by double inequality.

        In view of the definition of a supremum as the least upper bound, we check that $\lambda_{\bdd}\left( M \right)$ is an upper bound for $\lambda_{\bdd}\left( M\cap A \right)$ where $A$ runs in $\mM_{\bdd}$. So let us choose $A\in\mM_{\bdd}$, for which we verify $\lambda_{\bdd}\left( M\cap A \right)\leq\lambda_{\bdd}\left( M \right)$. We look at the standard decomposition
        \begin{equation*}
            M = \left( M\cap A \right)\cup \left( M\setminus A \right)
        \end{equation*}
        where $M\cap A$ and $M\setminus A$ are disjoint. Since $M,A\in\mM_{\bdd}$ and $\mM_{\bdd}$ is a ring of sets, $M\cap A, M\setminus A\in\mM_{\bdd}$. This means that
        \begin{equation*}
            \lambda_{\bdd}\left( M\cap A \right)+\lambda_{\bdd}\left( M\setminus A \right) = \lambda_{\bdd}\left( M \right),
        \end{equation*}
        from which the inequality $\lambda_{\bdd}\left( M\cap A \right)\leq\lambda_{\bdd}\left( M \right)$ follows. 

        On the other hand, by noting that $M\in\mM_{\bdd}$, we have that $\lambda_{\bdd}\left( M \right)=\lambda_{\bdd}\left( M\cap M \right) \leq \sup_{A\in\mM_{\bdd}}\lambda_{\bdd}\left( M\cap A \right)$. This proves the other inequality.

        Thus [1.51] is established.
    \end{proof}

    \begin{cor}{}
        Let $K\subseteq\R$ be compact. Then $K\in\mM$ and $\lambda\left( K \right)=\lambda_{\cp}\left( K \right)$.
    \end{cor}	

    \placeqed[See Proposition 1.27, 1.28]

    \begin{exercise}{}
        Let $A,B\in\mM_{\bdd}$ be such that $A\subseteq B$. Prove that $\lambda_{\bdd}\left( A \right)\leq\lambda_{\bdd}\left( B \right)$.
    \end{exercise}

    \begin{proof}
        Note that
        \begin{equation*}
            \lambda_{\bdd}\left( B \right) = \lambda_{\bdd}\left( \left( B\setminus A \right)\cup \left( B\cap A \right) \right) = \lambda_{\bdd}\left( B\setminus A \right) + \lambda_{\bdd}\left( B\cap A \right) = \lambda_{\bdd}\left( B\setminus A \right)+\lambda_{\bdd}\left( A \right),
        \end{equation*}
        from where the desired inequality follows.
    \end{proof}

    \np For a \textit{bounded} open set $G$, there is no problem to get the analogue of Corollary 1.28.1. We say that $G$ belongs to $\mM_{\bdd}$ and has $\lambda_{\bdd}\left( G \right)=\lambda_{\op}\left( G \right)$, thus Propositoin 1.28 assures us that $G\in\mM$ and $\lambda\left( G \right)=\lambda_{\op}$. But we don't want to only look at bounded open sets -- the unbounded open sets would better be Lebesgue measurable too. 

    Here is a criterion that simplifies a bit the verification of the condition from tne definition of Lebesgue measurability.

    \begin{prop}{}
        Let $M\subseteq\R$ such that
        \begin{equation}
            \forall n\in\N \left[ M\cap\left( -n,n \right)\in\mM_{\bdd}\right].
        \end{equation}
        Then $M\in\mM$, and its Lebesgue measure can be obtained as
        \begin{equation}
            \lambda\left( M \right) = \lim_{n\to\infty}\lambda_{\bdd}\left( M\cap\left( -n,n \right) \right).
        \end{equation}
    \end{prop}

    \clearpage
    \begin{proof}
        We divide the argument into several claims.
        \begin{itemize}
            \item \textit{Claim 1. For every $A\in\mM_{\bdd}$, $M\cap A\in\mM_{\bdd}$.}

                \begin{subproof}
                    Let $A\in\mM_{\bdd}$. Using the fact that $A$ is bounded, we pick $n\in\N$ such that $A\subseteq\left( -n,n \right)$. Then $A\cap\left( -n,n \right)=A$, hence we can write
                    \begin{equation*}
                        M\cap A = M\cap\left( \left( -n,n \right)\cap A \right)=\left( M\cap\left( -n,n \right) \right)\cap A.
                    \end{equation*}
                    In this way, $M\cap A$ gets to be written as the intersection of two sets that are known to be in $\mM_{\bdd}$, namely $A$ and $M\cap\left( -n,n \right)$. Since $\mM_{\bdd}$ is closed under finite intersections, we conclude that $M\cap A\in\mM_{\bdd}$, as required.
                \end{subproof}

            \item \textit{Claim 2. The sequence $\left( \lambda_{\bdd}\left( M\cap\left( -n,n \right) \right) \right)^{\infty}_{n=1}$ is an increasing sequence of numbers in $\left[ 0,\infty \right)$.}

                \begin{subproof}
                    The statement to be checked here is that
                    \begin{equation*}
                        \forall n\in\N\left[ \lambda_{\bdd}\left( M\cap\left( -n,n \right) \right)\leq\lambda_{\bdd}\left( M\cap \left( -n-1,n+1 \right) \right) \right].
                    \end{equation*}
                    This holds simply because $M\cap\left( -n,n \right)\subseteq M\cap\left( -n-1,n+1 \right)$, and by invoking Exercise 1.10.
                \end{subproof}
        \end{itemize} 
        It is an immediate consequence of Claim 2 that $\left( \lambda_{\bdd}\left( M\cap\left( -n,n \right) \right) \right)^{\infty}_{n=1}$ has a limit $l\in\left[ 0,\infty \right]$. We now show that $l=\lambda\left( M \right)$ by using double inequalities.
        \begin{itemize}
            \item \textit{Claim 3. $l\leq\lambda\left( M \right)$.}

                \begin{subproof}
                    In view of how $l$ is defined, it suffices to show that
                    \begin{equation}
                        \forall n\in\N\left[ \lambda_{\bdd}\left( M\cap\left( -n,n \right) \right)\leq\lambda\left( M \right) \right].
                    \end{equation}
                    The definition of $\lambda\left( M \right)$ in Def'n 1.16 ensures that $\lambda\left( M \right)\leq\lambda_{\bdd}\left( M\cap A \right)$ for all $A\in\mM_{\bdd}$. But on the left-hand side of [1.54] we have precisely a quantity $\lambda_{\bdd}\left( M\cap A \right)$ with $A=\left( -n,n \right)$.
                \end{subproof}

            \item \textit{Claim 4. $l\geq\lambda\left( M \right)$.}

                \begin{subproof}
                    In view of the definition of $\lambda\left( M \right)$ as a supremum (i.e. \textit{least} upper bound), it suffices to check that $l$ is an upper bound for those quantities. Or more precisely, it suffices to check that
                    \begin{equation}
                        \forall A\in\mM_{\bdd}\left[ l\geq\lambda_{\bdd}\left( M\cap A \right) \right].
                    \end{equation}
                    So let us fix $A\in\mM_{\bdd}$, for which we will verify that [1.55] holds. Since $A$ is bounded, there is $n\in\N$ such that $A\subseteq\left( -n,n \right)$. So then we have
                    \begin{equation*}
                        l\geq\lambda_{\bdd}\left( M\cap\left( -n,n \right) \right)\geq\lambda_{\bdd}\left( M\cap A \right),
                    \end{equation*}
                    where for the first inequality we use the fact that the limit of an increasing sequence is at least every term of the sequence and for the second inequality we invoke Exercise 1.10, in connection to the inclusion $M\cap A\subseteq M\cap\left( -n,n \right)$.
                \end{subproof}
        \end{itemize} 
        Combining Claim 2, 3, 4 gives [1.53].
    \end{proof}

    \np The criterion provided by Propostiion 1.29 is very suitable for arguing that every open set $G$ is Lebesgue measurable, and that $\lambda\left( G \right)=\lambda_{\op}\left( G \right)$. But here there is a background property of $\lambda_{\op}$ that has to be mentioned. We add this to our list of such properties, and record it in the next exercise.

    \begin{exercise}{}
        Let $\left( G_{n} \right)^{\infty}_{n=1}$ be an increasing chain of open subsets of $\R$ and consider their union $G=\bigcup^{\infty}_{n=1}G_n$. Based on the facts about $\lambda_{\op}$ in Subsection 1, 2, 3, prove that $\lim_{n\to\infty}\lambda_{\op}\left( G_n \right)=\lambda_{\op}\left( G \right)$.
    \end{exercise}

    \placeqed[tl;dr]

    \np Armed with Exercise 1.11, here is then our desired statement about open sets.

    \clearpage
    \begin{prop}{}
        Let $G$ be an open subset of $\R$. Then $G\in\mM$ and $\lambda\left( G \right)=\lambda_{\op}\left( G \right)$.
    \end{prop}

    \begin{proof}
        For all $n\in\N$, we have that $G\cap\left( -n,n \right)$ is a bounded open set, and in particular it belongs to $\mM_{\bdd}$. The hypothesis of Proposition 1.29 is thus satisfied, and the said proposition gives us that $G\in\mM$ and that
        \begin{equation}
            \lambda\left( G \right) = \lim_{n\to\infty}\lambda_{\bdd}\left( G\cap\left( -n,n \right) \right).
        \end{equation}
        We now move to proving the equality $\lambda\left( G \right)=\lambda_{\op}\left( G \right)$. Since $\lambda_{\bdd}$ extends $\lambda_{\op}$ on bounded open sets, we have $\lambda_{\bdd}\left( G\cap\left( -n,n \right) \right)=\lambda_{\op}\left( G\cap\left( -n,n \right) \right)$ for all $n\in\N$. Hence [1.56] can also be written in the form
        \begin{equation}
            \lambda\left( G \right) = \lim_{n\to\infty}\lambda_{\op}\left( G\cap\left( -n,n \right) \right).
        \end{equation}
        But the limit on the right-hand side of [1.57] is equal to $\lambda_{\op}\left( G \right)$, as we see by applying Exercise 1.11 to the increasing chain of open sets
        \begin{equation*}
            G\cap\left( -1,1 \right)\subseteq G\left( -2,2 \right)\subseteq\cdots
        \end{equation*}
        which has $\bigcup^{\infty}_{n=1}\left( G\cap\left( -n,n \right) \right)=G\cap\left( \bigcup^{\infty}_{n=1}\left( -n,n \right) \right)=G\cap\R=G$. Thus $\lambda\left( G \right)=\lambda_{\op}\left( G \right)$, as required.
    \end{proof}

    \np We now put into evidence the fact that $\mM$ is an \textit{algebra} of subsets of $\R$.

    \begin{definition}{\textbf{Algebra} of Sets}
        Let $X$ be a set. We say a collection $\mA$ of subsets of $X$ is an \emph{algebra} of subsets of $X$ if $X\in\mA$ and $\mA$ is a ring of subsets of $X$.
    \end{definition}

    \np This comes a bit short of the declared goal of proving that $\mM$ is a $\sigma$-algebra of subsets of $\R$ -- we wil lfix this shortcoming on the next subsection.

    \begin{prop}{}
        $\mM$ is an algebra of subsets of $\R$.
    \end{prop}

    \begin{proof}
        \begin{enumerate}
            \item Since $\emptyset,\R$ are open, $\emptyset,\R\in\mM$.
            \item Let $M\in\mM$. Given any $A\in\mM_{\bdd}$, we can write $\left( \R\setminus M \right)\cap A = A\setminus\left( M\cap A \right)$. The set $A\setminus \left( M\cap A \right)$ is in $\lambda_{\bdd}$, since $\mM_{\bdd}$ is a ring of sets.

                We thus found that $\left( \R\setminus M \right)\cap A\in\mM_{\bdd}$ for all $A\in\mM_{\bdd}$. This means that $\R\setminus M\in\mM$. 

                In particular, given any $N\in\mM$, $M\setminus N = \left( \R\setminus N \right)\cap M\in\mM$ by (d).

            \item Let $M_1,M_2\in\mM$. Then for all $A\in\mM_{\bdd}$,
                \begin{equation*}
                    \left( M_1\cup M_2 \right)\cap A = \left( M_1\cap A \right)\cup \left( M_2\cap A \right)\in\mM_{\bdd}.
                \end{equation*}
                This means $M_1\cup M_2\in\mM$.

            \item Let $M_1,M_2\in\mM$. Then for all $A\in\mM_{\bdd}$,
                \begin{equation*}
                    \left( M_1\cap M_2 \right)\cap A = \left( M_1\cap A \right)\cap \left( M_2\cap A \right)\in\mM_{\bdd}.
                \end{equation*}
                This means $M_1\cap M_2\in\mM$.
        \end{enumerate}
        Since $\mM$ is a collection of subsets of $\R$ that has $\emptyset,\R$ and is closed under union and set-difference, $\mM$ is an algebra of subsets of $\R$.
    \end{proof}

    \np We noteiced that every compact subset of $\R$ belongs to $\mM$ (Corollary 1.28.1), but we did not say anything about general \textit{closed} subsets of $\R$, which may be unbounded. At this point it has, however, become clear that every closed set $F\subseteq\R$ belongs to $\mM$. Indeed, the complement $G=\R\setminus F$ is an open set, so we argue like this:
    \begin{flalign*}
        && & G\in\mM && \text{by Proposition 1.30} \\
        && & \implies \R\setminus G\in\mM && \text{by Proposition 1.31} \\
        && & \implies F\in\mM.
    \end{flalign*}
    Let also record the fact that the additivity property we had obtained for $\lambda_{\bdd}$ in Proposition 1.26 can be upgraded to the framework of $\lambda$ on $\mM$.

    \begin{prop}{}
        Let $M_1,M_2\in\mM$ be disjoint. Consider the union $M_1\cup M_2$, which is still in $\mM$ by Proposition 1.31. Then
        \begin{equation}
            \lambda\left( M_1\cup M_2 \right)=\lambda\left( M_1 \right)+\lambda\left( M_2 \right).
        \end{equation}
    \end{prop}

    \begin{proof}
        This is an easy application of the limit trick observed in Proposition 1.29, combined with the additivity property we already know for $\lambda_{\bdd}$. Indeed,
        \begin{equation}
            \forall n\in\N\left[ \left( M_1\cup M_2 \right)\cap\left( -n,n \right)=\left( M_1\cap\left( -n,n \right) \right)\cup \left( M_2\cap \left( -n,n \right) \right) \right],
        \end{equation}
        where $M_1\cap\left( -n,n \right)$ and $M_2\cap\left( -n,n \right)$ are two pairwise disjoint sets from $\mM_{\bdd}$. Upon applying Proposition 1.26 to this situation, we find that
        \begin{equation}
            \forall n\in\N\left[ \lambda_{\bdd}\left( M_1\cup M_2 \right)\cap\left( -n,n \right)=\lambda_{\bdd}\left( M_1\cap\left( -n,n \right) \right)+\lambda_{\bdd}\left( M_2\cap\left( -n,n \right) \right) \right].
        \end{equation}
        When we make $n\to\infty$ in [1.60], Proposition 1.29 implies [1.50], the desired equality.
    \end{proof}

    \np In order to reach our declared goal of proving that $\mM$ is a $\sigma$-algebra and that $\lambda:\mM\to\left[ 0,\infty \right]$ is a positive measure, we have to strengthen the results of Proposition 1.31, 1.32 so that they cover \textit{countable} unions, which is the topic of the next subsection.

    \subsection{$\lambda$ Is a Positive Measure}
    
    In the preceeding subsections we introduced the collection $\mM$ of Lebesgue measurable subsets of $\R$, and for every $M\in\mM$ we defined its Lebesgue measure $\lambda\left( M \right)$ as the supremum of the lengths of all its bounded truncations:
    \begin{equation*}
        \lambda\left( M \right) = \sup\left\lbrace \lambda_{\bdd}\left( M\cap A \right): A\in\mM_{\bdd} \right\rbrace\in\left[ 0,\infty \right].
    \end{equation*}
    We saw that $\lambda\left( M \right)$ can in fact be also obtained as a plain limit:
    \begin{equation*}
        \lambda\left( M \right) = \lim_{n\to\infty}\lambda_{\bdd}\left( M\cap\left( -n,n \right) \right).
    \end{equation*}
    We also noted that $\mM$ is an algebra of subsets of $\R$, and that $\lambda$ is an additive function on $M$. This has the following consequence.

    \begin{prop}{}
        Let $M,N\in\mM$ be such that $M\subseteq N$. Then $\lambda\left( M \right)\leq\lambda\left( N \right)$.
    \end{prop}

    \begin{proof}
        Let $M'=N\setminus M$. Then $M'\in\mM$ since $\mM$ is closed under intersection, so that the additivity of $\lambda$ implies
        \begin{equation*}
            \lambda\left( N \right) = \lambda\left( M' \right) + \lambda\left( M \right).
        \end{equation*}
        Then the fact that $\lambda$ is nonnegative implies $\lambda\left( N \right)\geq\lambda\left( M \right)$, as required.
    \end{proof}

    \clearpage
    \np The following are what we want to show in this subsection.
    \begin{enumerate}
        \item We want to prove that $\mM$ is closed under countable union. This will allow us to conclude that $\mM$ is a $\sigma$-algebra.
        \item We want to prove that if $\mC\subseteq\mM$ is a countable collection of pairwise disjoint Lebesgue measurable sets, then
            \begin{equation*}
                \lambda\left( \bigcup^{}_{}\mC \right) = \sum^{}_{M\in\mC}\lambda\left( M \right).
            \end{equation*}
            This will allow us to conclude that $\lambda$ is a positive measure.
    \end{enumerate}
    It turns out that the key towards obtaining the above needed properties (a) and (b) is actually lying at the preceding level of our development, where we looked at the length-measuring function $\lambda_{\bdd}:\mM_{\bdd}\to\left[ 0,\infty \right)$. So we first consider some \textit{preparations} for proving (a) and (b), which go at the level of $\mM_{\bdd}$.

    \np Note that $\lambda_{\bdd}$ definitely is not closed under infinitely countable union, since we cannot control the boundedness of such a union. However, we will see that good properties can be obtained when we \textit{force} the boundedenss of the union, by putting it among our hypotheses. We start by making precise the setting that we want to use.

    Throughout this subsection we fix a countable collection $\left\lbrace A_n \right\rbrace^{\infty}_{n=1}$ of pairwise disjoint sets from $\mM_{\bdd}$. We assume, moreover, there exists $r>0$ such that $A_n\subseteq\left( -r,r \right)$ for all $n\geq 1$. 

    We consider the union $A=\bigcup^{\infty}_{n=1}A_n$. It is clear that $A\subseteq\left( -r,r \right)$; thus it is a bounded set, but it is not so clear whether $A\in\mM_{\bdd}$. The goal is to verify that $A\in\mM_{\bdd}$, and moreover we have the \textit{correct} value for $\lambda_{\bdd}\left( A \right)$. The key point is provided by the following lemma.

    \begin{lemma}{}
        Let $\epsilon>0$. Then there is $p_0\in\N$ such that there are compact sets $K_1,\ldots,K_{p_0}$ and an open set $U$ such that
        \begin{enumerate}
            \item $K_1\subseteq A_1, \ldots, K_{p_0}\subseteq A_{p_0}$;
            \item $A\subseteq U\subseteq\left( -r,r \right)$; and
            \item $\lambda_{\op}\left( U \right)<\epsilon+\lambda_{\cp}\left( \bigcup^{p_0}_{j=1}K_j \right)$.
        \end{enumerate}
    \end{lemma}

    \begin{proof}
        Every individual set $A_n$ can be approximated in the way indicated in the definition of $\mM_{\bdd}$ (Def'n 1.11): given any $\epsilon'>0$, there are a bounded open set $G_n$ containing $A_n$ and a compact set $K_n$ contained $A_n$, with
        \begin{equation*}
            \lambda_{\op}\left( G_n \right)-\lambda_{\cp}\left( K_n \right)<\epsilon'.
        \end{equation*}
        We write down a sequence of such approximations, where for each of them we use a suitable fraction $\epsilon'$ of the $\epsilon$ given in the statement. That is, for every $n\geq 1$, we find a compact set $K_n$ and a bounded open set $G_n$ such that
        \begin{equation}
            K_n\subseteq A_n\subseteq G_n, \lambda_{\op}\left( G_n \right)<\lambda_{\cp}\left( K_n \right)+\frac{\epsilon}{2^{n+1}}.
        \end{equation}
        By replacing $G_n$ with $G_n\cap\left( -r,r \right)$ (which is still a bounded open set containing $A_n$, and has $\lambda\left( G_n\cap\left( -r,r \right) \right)\leq\lambda\left( G_n \right)$) we may assume that $G_n\subseteq\left( -r,r \right)$ for all $n\geq 1$. 

        We then split the rest of the proof into few claims.
        \begin{itemize}
            \item \textit{Claim 1. Let $U=\bigcup^{\infty}_{n=1}G_n$. Then $U$ is an open subset of $\R$ such that $A\subseteq U\subseteq\left( -r,r \right)$.}

                \begin{subproof}
                    $U$ is open as a union of open sets and $U\subseteq\left( -r,r \right)$ since each $G_n$ is a subset of $\left( -r,r \right)$. The inclusion $A\subseteq U$ is obtained by noting that $A_n\subseteq G_n$ for all $n\in\N$.
                \end{subproof}

            \item \textit{Claim 2. For every $p\in\N$, consider the open set $U_p=\bigcup^{p}_{n=1}G_n\subseteq U$. Then there is $p_0\in\N$ such that
                    \begin{equation*}
                        \lambda_{\op}\left( U_{p_0} \right)>\lambda_{\op}\left( U \right)-\frac{\epsilon}{2}.
                    \end{equation*}
                }

                \begin{subproof}
                    It is immediate that $U_1\subseteq U_2\subseteq\cdots$ and that $\bigcup^{\infty}_{n=1}U_n=U$. We are dealing with an increasing chain of open sets, so by Exercise 1.11 we have
                    \begin{equation}
                        \lim_{n\to\infty}\lambda_{\op}\left( U_n \right) = \lambda_{\op}\left( U \right).
                    \end{equation}
                    In [1.62], the limit $\lambda_{\op}\left( U \right)$ is finite -- indeed, from $U\subseteq\left( -r,r \right)$ it follows that $\lambda_{\op}\left( U \right)\leq 2r$. So then by the definition of limits there is a $p_0\in\N$ such that the claim holds.
                \end{subproof}
        \end{itemize} 
        It remains to show that the compact sets $K_1,\ldots,K_{p_0}$ and the open set $U$ satisfy the conditions listed in the statement. Note that (a), (b) are satisfied due to how $K_1,\ldots,K_{p_0}$ and $U$ were constructed. We are left to verify that our \textit{splitting} of $\epsilon$ was made judiciously enough in order to imply that (c) holds.

        Consider Lemma 1.23, which was used in the proof that $\mM_{\bdd}$ is a ring of sets. We apply this lemma in connection to the inclusions $K_1\subseteq G_1,\ldots,K_{p_0}\subseteq G_{p_0}$ to find that
        \begin{equation}
            \lambda_{\op}\left( \left( \bigcup^{p_0}_{n=1}G_n \right)\setminus\left( \bigcup^{p_0}_{n=1}K_n \right) \right)\leq \sum^{p_0}_{n=1} \lambda_{\op}\left( G_n\setminus K_n \right).
        \end{equation}
        On the right-hand side of [1.63], every $\lambda_{\op}\left( G_n\setminus K_n \right)$ can be replaced with $\lambda_{\op}\left( G_n \right)-\lambda_{\cp}\left( K_n \right)$, and is therefore less than $\frac{\epsilon}{2^{n+1}}$, deu to how $K_n$ and $G_n$ were chosen at the beginning of the proof. So then [1.63] can be continued with
        \begin{equation}
            \lambda_{\op}\left( \left( \bigcup^{p_0}_{n=1}G_n \right)\setminus\left( \bigcup^{p_0}_{n=1}K_n \right) \right)\leq \sum^{p_0}_{n=1} \frac{\epsilon}{2^{n=1}} < \frac{\epsilon}{2}.
        \end{equation}
        Now, on the left-hand side of [1.64], we have $\bigcup^{p_0}_{n=1}G_n = U_{p_0}$ and we can replace $\lambda_{\op}\left( U_{p_0}\setminus \bigcup^{p_0}_{n=1}K_n \right)$ with $\lambda_{\op}\left( U_{p_0} \right)-\lambda_{\cp}\left( \bigcup^{p_0}_{n=1}K_n \right)$. So [1.64] amounts to $\lambda_{\op}\left( U_{p_0} \right)-\lambda_{\cp}\left( \bigcup^{p_0}_{n=1}K_n \right)$, and we ahve thus obtained the inequality
        \begin{equation}
            \lambda_{\op}\left( U_{p_0} \right) < \lambda_{\cp}\left( \bigcup^{p_0}_{n=1}K_n \right)+\frac{\epsilon}{2}.
        \end{equation}
        Finally, by putting together [1.65] with the inequality from Claim 2, we find that
        \begin{equation*}
            \lambda_{\op}\left( U \right)<\lambda_{\op}\left( U_{p_0} \right)+\frac{\epsilon}{2}<\lambda_{\cp}\left( \bigcup^{p_0}_{n=1}K_n \right)+\epsilon,
        \end{equation*}
        which is what the condition (c) was asking for.
    \end{proof}

    \begin{prop}{}
        $A\in\mM_{\bdd}$ with
        \begin{equation*}
            \lambda_{\bdd}\left( A \right)=\sum^{\infty}_{n=1}\lambda_{\bdd}\left( A_n \right).
        \end{equation*}
    \end{prop}

    \begin{proof}
        To show that $A\in\mM_{\bdd}$, we have to prove that for every $\epsilon>0$, there are a compact set $K$ and a bounded open set $G$ such that $K\subseteq A\subseteq G$ and that $\lambda_{\op}\left( G \right)<\lambda_{\cp}\left( K \right)+\epsilon$. This is found by using Lemma 1.34: in the notation of the lemma, the needed open set is $U$ and the needed compact set is $\bigcup^{p_0}_{n=1}K_n$. This verifies $A\in\mM_{\bdd}$.

        Now, denote $l=\sum^{\infty}_{n=1}\lambda_{\bdd}\left( A_n \right)$. We then have
        \begin{equation*}
            l = \lim_{p\to\infty} \sum^{p}_{n=1}\lambda_{\bdd}\left( A_n \right) = \sup_{p\in\N} \sum^{p}_{n=1}\lambda_{\bdd}\left( A_n \right),
        \end{equation*}
        hence the inequality $\lambda_{\bdd}\left( A \right)\geq l$ will follow if we can prove that
        \begin{equation}
            \forall p\in\N \left[ \lambda_{\bdd}\left( A \right)\geq \sum^{p}_{n=1}\lambda_{\bdd}\left( A_n \right)\right].
        \end{equation}
        But the inequality [1.66] follows easily from known properties of $\lambda_{\bdd}$:
        \begin{flalign*}
            && \sum^{p}_{n=1}\lambda_{\bdd}\left( A_n \right) & = \lambda_{\bdd}\left( \bigcup^{p}_{n=1}A_n \right) && \text{since $\lambda_{\bdd}$ is additive}\\ 
            && & \leq \lambda_{\bdd}\left( A \right) && \text{since $\lambda_{\bdd}$ is increasing}
        \end{flalign*}
        for all $p\in\N$. This concludes that $\lambda_{\bdd}\left( A \right)\geq l$.

        It remains to show that $\lambda_{\bdd}\left( A \right)\leq l$. For this inequality, it is convenient to resort to the trick of \textit{leaving an $\epsilon$ of a room}: it is sufficient to prove that
        \begin{equation}
            \forall\epsilon>0\left[ \lambda_{\bdd}\left( A \right)<l+\epsilon \right].
        \end{equation}
        We then fix $\epsilon>0$, for which we verify $\lambda_{\bdd}\left( A \right)<l+\epsilon$. In connection to this $\epsilon$, we appeal to Lemma 1.34, and we take in the compact sets $K_1,\ldots,K_{p_0}$ and the bounded open set $U$ provided by the lemma.
        \begin{flalign*}
            && \lambda_{\bdd}\left( A \right) & \leq \lambda_{\bdd}\left( U \right) && \text{since $\lambda_{\bdd}$ is increasing} \\
            && & = \lambda_{\op}\left( U \right) && \\
            && & < \epsilon + \lambda_{\cp}\left( K_1\cup\cdots\cup K_{p_0} \right) && \text{by definition of $K_1,\ldots,K_{p_0}$} \\
            && & = \epsilon + \lambda_{\bdd}\left( K_1\cup\cdots\cup K_{p_0} \right) && \\
            && & = \epsilon + \sum^{p_0}_{n=1}\lambda_{\bdd}\left( K_n \right) && \text{since $\lambda_{\bdd}$ is additive} && \\
            && & \leq \epsilon + \sum^{p_0}_{n=1} \lambda_{\bdd}\left( A_n \right) && \text{since $\lambda_{\bdd}$ is increasing} && \\
            && & = \epsilon+l.
        \end{flalign*}

        We have now verified both inequalities $\lambda_{\bdd}\left( A \right)\geq l$ and $\lambda_{\bdd}\left( A \right)\leq l$, and this concludes the proof.
    \end{proof}

    \np We now return to our goal of proving that $\mM$ is a $\sigma$-algebra of subsets of $\R$.

    \begin{lemma}{}
        Let $\left\lbrace M_n \right\rbrace^{\infty}_{n=1}$ be a collection of pairwise disjoint sets from $\mM$. Then $\bigcup^{\infty}_{n=1}M_n\in\mM$.
    \end{lemma}

    \begin{proof}
        Denote $M=\bigcup^{\infty}_{n=1}M_n$. In order to prove that $M\in\mM$, it suffices to check that (by Proposition 1.29)
        \begin{equation}
            \forall k\in\N\left[ M\cap\left( -k,k \right)\in\mM_{\bdd} \right].
        \end{equation}
        We thus fix $k\in\N$, for which we verify that $M\cap\left( -k,k \right)\in\mM_{\bdd}$.

        Define $A_n=M_n\cap\left( -k,k \right)$ for all $n\in\N$. Every $A_n$ belongs to $\mM_{\bdd}$ and is obviously contained in $\left( -k,k \right)$. Moreover, for distinct indices $m,n\in\N$ we have that $A_m\cap A_n=\emptyset$, due to the fact that $A_m\subseteq M_m, A_n\subseteq A_n$, and we have as hypothesis that $M_m\cap M_n=\emptyset$. It follows that Proposition 1.35 can be applied to the collection $\left\lbrace A_n \right\rbrace^{\infty}_{n=1}$: it gives us the conclusion that $\bigcup^{\infty}_{n=1}A_n\in\mM_{\bdd}$. But
        \begin{equation*}
            \bigcup^{\infty}_{n=1}A_n = \bigcup^{\infty}_{n=1} \left( M_n\cap \left( -k,k \right) \right) = \left( -k,k \right)\cap \bigcup^{\infty}_{n=1}M_n = M\cap\left( -k,k \right).
        \end{equation*}
        So we have obtained that $M\cap\left( -k,k \right)\in\mM_{\bdd}$, as required.
    \end{proof}

    \begin{prop}{}
        Let $\left\lbrace M_n \right\rbrace^{\infty}_{n=1}$ be a collection of sets in $\mM$. Then $\bigcup^{\infty}_{n=1}M_n\in\mM$.
    \end{prop}

    \begin{proof}
        Unlike what we had in Lemma 1.36, here there is no assumption that the sets $M_n$ would be pairwise disjoint. But we can nevertheless fall back on Lemma 1.36 by using the following trick: consider the collection $\left\lbrace M_n' \right\rbrace^{\infty}_{n=1}$ defined by
        \begin{equation*}
            M_n' = M_n\setminus \bigcup^{n-1}_{m=1}M_m
        \end{equation*}
        for all $n\in\N$. The fact that $\mM$ is an algebra of sets ensures that $M_n'\in\mM$ for all $n\in\N$. On the other hand, basic verifications involving set operations show us that $\bigcup^{\infty}_{n=1}M_n'=\bigcup^{\infty}_{n=1}M_n$ and that $M_n'\cap M_m'=\emptyset$ for all distinct $m,n\in\N$. Thus Lemma 1.36 applies to the collection $\left\lbrace M_n' \right\rbrace^{\infty}_{n=1}$, and concludes that $\bigcup^{\infty}_{n=1}M_n=\bigcup^{\infty}_{n=1}M_n'\in\mM$.
    \end{proof}

    \begin{cor}{}
        $\mM$ is a $\sigma$-algebra of subsets of $\R$.
    \end{cor}	

    \placeqed[See Proposition 1.31, 1.37]

    \np Lastly, we are going to show that $\lambda$ is a positive measure.

    \begin{lemma}{}
        Let $\left\lbrace M_n \right\rbrace^{\infty}_{n=1}$ be a collection of pairwise disjoint sets from $\mM$ and let $l=\sum^{\infty}_{n=1}\lambda\left( M_n \right)\in\left[ 0,\infty \right]$. On the other hand, consider the set $M=\bigcup^{\infty}_{n=1}M_n$, which is still in $\mM$. We have the inequality
        \begin{equation}
            l\leq\lambda\left( M \right).
        \end{equation}
    \end{lemma}

    \begin{proof}
        Let us write explicitly that
        \begin{equation*}
            l = \lim_{p\to\infty} \sum^{p}_{n=1}\lambda\left( M_n \right) = \sup_{p\in\N} \sum^{p}_{n=1}\lambda\left( M_n \right).
        \end{equation*}
        Since the supremum is a \textit{least} upper bound, we see that [1.67] will follow as soon as we prove that
        \begin{equation}
            \forall p\in\N\left[ \sum^{p}_{n=1}\lambda\left( M_n \right)\leq\lambda\left( M \right) \right].
        \end{equation}
        And indeed, for each $p\in\N$, the inequality in [1.68] follows from Proposition 1.33. 
    \end{proof}

    \begin{lemma}{}
        Consider the setting from Lemma 1.38. We also have:
        \begin{equation}
            l\geq\lambda\left( M \right).
        \end{equation}
    \end{lemma}

    \begin{proof}
        Since $\lambda\left( M \right)=\lim_{k\to\infty}\lambda_{\bdd}\left( M\cap\left( -k,k \right) \right)$, it suffices to prove that
        \begin{equation}
            \forall k\in\N\left[ \lambda_{\bdd}\left( M\cap\left( -k,k \right) \right)\leq l \right].
        \end{equation}
        We fix $k\in\N$ for which we verify that $\lambda_{\bdd}\left( M\cap\left( -k,k \right) \right)\leq l$ holds. 

        Let us resort again to the tick of the \textit{$\epsilon$ of room}. For the $k$ that we fixed, the desired inequality will follow if we can show that
        \begin{equation}
            \forall\epsilon>0\left[ \lambda_{\bdd}\left( M\cap\left( -k,k \right) \right)-\epsilon<l \right].
        \end{equation}
        Hence, in addition to the fixed $k$, let us also fix $\epsilon>0$, and go for the proof of [1.73].

        Observe that $M\cap\left( -k,k \right)=\left( -k,k \right)\cap\bigcup^{\infty}_{n=1}M_n = \bigcup^{\infty}_{n=1}\left( M_n\cap\left( -k,k \right) \right)$, where the sets $M_n\cap\left( -k,k \right)$ are in $\mM_{\bdd}$, pairwise disjoint, and contained in $\left( -k,k \right)$. Hence we are precisely in the situation where Proposition 1.35 can be applied. The said proposition tells us that
        \begin{equation}
            \lambda_{\bdd}\left( M\cap\left( -k,k \right) \right)=\sum^{\infty}_{n=1}\lambda_{\bdd}\left( M_n\cap\left( -k,k \right) \right)=\lim_{p\to\infty} \sum^{p}_{n=1}\lambda_{\bdd}\left( M_n\cap\left( -k,k \right) \right).
        \end{equation}
        In [1.74] we are dealing with a finite limit of an increasing sequence in $\left[ 0,\infty \right)$, so there is no problem to find a $p_0\in\N$ such that
        \begin{equation*}
            \sum^{p_0}_{n=1}\lambda_{\bdd}\left( M_n\cap\left( -k,k \right) \right)>\lambda\left( M\cap\left( -k,k \right) \right)-\epsilon.
        \end{equation*}
        Using this $p_0$ we write that
        \begin{equation*}
            l = \sum^{\infty}_{n=1}\lambda\left( M_n \right)\geq\sum^{p_0}_{n=1} \geq \sum^{p_0}_{n=1}\lambda_{\bdd}\left( M_n\cap\left( -k,k \right) \right)>\lambda\left( M\cap\left( -k,k \right) \right)-\epsilon,
        \end{equation*}
        and this leads to the inequality [1.73] we had been left to prove.
    \end{proof}

    \begin{cor}{}
        The map $\lambda:\mM\to\left[ 0,\infty \right]$ is $\sigma$-additive, and thus a positive measure.
    \end{cor}	

    \placeqed[See Corollary 1.37.1, Lemma 1.38, 1.39]











































\end{document}
