\documentclass[pmath450]{subfiles}

%% ========================================================
%% document

\begin{document}

    \section{Hilbert Spaces}

    \subsection{$\mL^p\left( \mu \right)$ Spaces and the Minkowski Inequality}
    
    \begin{notation}{$\mL^p\left( \mu \right)$}
        Given $p\in\left( 0,\infty \right)$, we denote
        \begin{equation*}
            \mL^p\left( \mu \right) = \left\lbrace f\in\bor\left( X,\R \right): \int_X\left| f \right|^p<\infty \right\rbrace.
        \end{equation*}
    \end{notation}
    
    \begin{prop}{}
        Let $p\in\left( 0,\infty \right)$. $\mL^p\left( \mu \right)$ is a linear subspace of $\bor\left( X,\R \right)$.
    \end{prop}

    \begin{proof}
        Claerly $\underline{0}\in\mL^p\left( \mu \right)$.

        Let $f,g\in\mL^p\left( \mu \right),\alpha\in\R$. Then
        \begin{equation*}
            \lplus\left( \left| \alpha f \right|^p \right) = \lplus\left( \left| \alpha \right|^p\left| f \right|^p \right) = \left| \alpha \right|^p \lplus\left( \left| f \right|^p \right)\infty.
        \end{equation*}
        Moreover,
        \begin{flalign*}
            && \lplus\left( \left| f+g \right|^p \right) & = \int^{}_{X}\left| f\left( x \right)+g\left( x \right) \right|^p\dif\mu = \int_{X} \left( 2\max\left\lbrace \left| f\left( x \right) \right|,\left| g\left( x \right) \right| \right\rbrace \right)^p\dif\mu && \\
            && & = \int_X\max\left\lbrace 2^p\left| f\left( x \right) \right|^p, 2^p\left| g\left( x \right) \right|^p \right\rbrace\dif\mu \leq 2^p\int_X \left| f\left( x \right) \right|^p + \left| g\left( x \right) \right|^p \dif\mu = 2^p\left( \lplus\left( \left| f \right|^p \right)+\lplus\left( \left| g \right|^p \right) \right)< \infty.
        \end{flalign*}
    \end{proof}

    \np As a special case, consider where $p=1$. Then note that $\mL^1\left( \mu \right)$ in Notation 4.1 coincides with our previous definition of $\lone\left( \mu \right)$.

    For any $f\in\lone\left( \mu \right)$, recall that we defined $\left\lVert f \right\rVert_{1}$ by
    \begin{equation*}
        \left\lVert f \right\rVert_{1} = \int^{}_{X}\left| f \right|\dif\mu\in\left[ 0,\infty \right).
    \end{equation*}
    This gave us a map $\left\lVert \cdot \right\rVert_{1}:\lone\left( \mu \right)\to\left[ 0,\infty \right)$, which is a \textit{seminorm}. That is,
    \begin{flalign*}
        && \forall f,g\in\lone\left( \mu \right) & \left[ \left\lVert f+g \right\rVert_{1}\leq \left\lVert f \right\rVert_{1}+\left\lVert g \right\rVert_{1} \right], && \text{\textit{triangle ineqautliy}} \\
        && \forall f\in\lone\left( \mu \right),c\in\R & \left[ \left\lVert cf \right\rVert_{1}=\left| c \right|\left\lVert f \right\rVert_{1} \right], && \text{\textit{absolute homogeneity}} \\
        && \forall f\in\lone\left( \mu \right) & \left[ \left\lVert f\right\rVert_1\geq 0 \right]. && \text{\textit{nonnegativity}}
    \end{flalign*}
    If we have in addition that
    \begin{flalign*}
        && \forall f\in\lone\left( \mu \right) & \left[ f\neq \underline{0}\implies \left\lVert f \right\rVert_{1}>0 \right], && \text{\textit{positive definiteness}}
    \end{flalign*}
    then $\left\lVert \cdot \right\rVert_{1}$ would be a \textit{norm}. Unfortunately, this is not the case in general.

    \begin{notation}{$\left\lVert \cdot\right\rVert_p$}
        Let $p\in\left( 0,\infty \right)$. For every $f\in\lp\left( \mu \right)$, we define
        \begin{equation*}
            \left\lVert f\right\rVert_p = \left( \int_X\left| f \right|^p\dif\mu \right)^{\frac{1}{p}} \in\left[ 0,\infty \right).\footnotemark[1]
        \end{equation*}
        
        \noindent
        \begin{minipage}{\textwidth}
            \footnotetext[1]{We are taking the power $\frac{1}{p}$ of the integral so that we can ensure the absolute homogeneity of $\left\lVert \cdot\right\rVert_p$.}
        \end{minipage}
    \end{notation}

    \np \textit{For what $p\in\left( 0,\infty \right)$ is $\left\lVert \cdot\right\rVert_p$ a seminorm?} We know that for $p=1$, $\left\lVert \cdot \right\rVert_{1}$ is a seminorm. 

    Now suppose $p\in\left( 0,\infty \right)\setminus \left\lbrace 1 \right\rbrace$. It is immediate from the definition that $\left\lVert \cdot\right\rVert_p$ is absolutely homogeneous. This means it suffices to check the triangle inequality for $\left\lVert \cdot\right\rVert_p$. It turns out that the answer is positive if and only if $p>1$. In case $p>1$, we call the special case of the triangle inequality \textit{Minkowski's inequality}.

    \begin{prop}{Minkowski's Inequality}
        Let $p>1$. Then
        \begin{equation*}
            \left\lVert f+g\right\rVert_p \leq \left\lVert f\right\rVert_p + \left\lVert g\right\rVert_p
        \end{equation*}
        for all $f,g\in\lp\left( \mu \right)$.
    \end{prop}
    
    \placeqed[Postponed]

    \np For $p>1$, we define a \textit{conjugate exponent} $q\in\left( 1,\infty \right)$, via requirement that $\frac{1}{p}+\frac{1}{q}=1$. That is,
    \begin{equation*}
        q = \frac{p}{p-1}.
    \end{equation*}
    There are few other equivalent definitions of $q$:
    \begin{equation*}
        \begin{aligned}
            p+q & = pq, && \\
            q\left( p-1 \right) & = p, && \\
            p-\frac{p}{q} & = 1.
        \end{aligned} 
    \end{equation*}
    The idea for proving Minkowski's inequality is by playing things about $\lp\left( \mu \right)$ against $\mL^q\left( \mu \right)$ (i.e. duality). Concretely, we have the following proposition.

    \begin{prop}{Holder's Inequality}
        Let $p,q\in\left( 1,\infty \right)$ be such that $\frac{1}{p}+\frac{1}{q}=1$.
        \begin{enumerate}
            \item If $f\in\lp\left( \mu \right),g\in\mL^q\left( \mu \right)$, then $fg\in\lone\left( \mu \right)$ with
                \begin{equation*}
                    \left\lVert fg \right\rVert_{1} \leq \left\lVert f\right\rVert_p\left\lVert g\right\rVert_q. \eqno\text{\textit{Holder's inequality}}
                \end{equation*}
            \item For any $f\in\lp\left( \mu \right)$, we have
                \begin{equation}
                    \left\lVert f\right\rVert_p = \sup\left\lbrace \left\lVert fh \right\rVert_{1}: h\in\mL^q\left( \mu \right), \left\lVert h\right\rVert_q\leq 1 \right\rbrace.
                \end{equation}
        \end{enumerate}
    \end{prop}

    \placeqed[Postponed]

    \np It is immediate from Holder's inequality that, if $h\in\mL^q\left( \mu \right)$ with $\left\lVert h\right\rVert_q\leq 1$, then
    \begin{equation*}
        \left\lVert fh \right\rVert_{1} \leq \left\lVert f\right\rVert_p\left\lVert h\right\rVert_q\leq \left\lVert f\right\rVert_p.
    \end{equation*}
    By remembering the definition of supremum, we obtain
    \begin{equation*}
        \left\lVert f\right\rVert_p \geq \sup\left\lbrace \left\lVert fh \right\rVert_{1}: h\in\mL^q\left( \mu \right), \left\lVert h\right\rVert_q\leq 1 \right\rbrace.
    \end{equation*}
    Therefore, the point of [4.1] is that we can find $h\in\mL^q\left( \mu \right)$ to also obtain
    \begin{equation*}
        \left\lVert f\right\rVert_p\leq \left\lVert fh \right\rVert_{1}.
    \end{equation*}

    \np We show that Minkowski's inequality follows easily from Proposition 4.3.

    \begin{boxyproof}{Proof of Minkowski's Inequality Assuming Proposition 4.3}
        Let $q$ be the conjugate exponent of $p$,
        \begin{equation*}
            q = \frac{p}{p-1}.
        \end{equation*}
        Use [4.1] for $f+g$ to obtain
        \begin{equation*}
            \left\lVert f+g\right\rVert_p = \sup\left\lbrace \left\lVert \left( f+g \right)h \right\rVert_{1}: h\in\mL^q\left( \mu \right), \left\lVert h\right\rVert_q\leq 1 \right\rbrace. 
        \end{equation*}
        By the definition of supremum, in order to get Minkowski's inequality, it suffices to check that
        \begin{equation}
            \left\lVert \left( f+g \right)h \right\rVert_{1} \leq \left\lVert f\right\rVert_p + \left\lVert g\right\rVert_p
        \end{equation}
        for all $h\in\mL^q\left( \mu \right)$ with $\left\lVert h\right\rVert_q\leq 1$. Hence fix $h\in\mL^q\left( \mu \right)$ with $\left\lVert h\right\rVert_q\leq 1$, for which we verify [4.2]. Indeed,
        \begin{flalign*}
            && \left\lVert \left( f+g \right)h \right\rVert_{1} & = \left\lVert fh+gh \right\rVert_{1} && \\ 
            && & \leq \left\lVert fh \right\rVert_{1} + \left\lVert gh \right\rVert_{1} && \text{triangle inequality for $\left\lVert \cdot \right\rVert_{1}$} \\
            && & \leq \left\lVert f\right\rVert_p\left\lVert h\right\rVert_q + \left\lVert g\right\rVert_p\left\lVert h\right\rVert_q && \text{Holder's inequality} \\
            && & \leq \left\lVert f\right\rVert_p + \left\lVert g\right\rVert_q.
        \end{flalign*}
    \end{boxyproof}
    
    \np We now turn to the proof of Holder's inequality.
    
    \begin{lemma}{}
        Given $p,q\in\left( 1,\infty \right)$ such that $\frac{1}{p}+\frac{1}{q}=1$, we have
        \begin{equation}
            \forall a,b\in\left[ 0,\infty \right)\left[ ab\leq\frac{1}{p}a^p+\frac{1}{q}b^q \right].
        \end{equation}
    \end{lemma}
    
    \begin{proof}
        When $a=0$ or $b=0$, [4.3] is clear. Assume $a,b>0$ and let $\alpha=\log\left( a \right),\beta=\log\left( b \right)$. Then the verification [4.3] amounts to
        \begin{equation}
            e^{\alpha}e^{\beta} \leq \frac{1}{p} e^{\alpha p} + \frac{1}{q} e^{\beta q}.
        \end{equation}
        But recall $\frac{1}{p}+\frac{1}{q}=1$, so the right-hand side of [4.4] is taking a convex combination of $e^{\alpha p}, e^{\alpha q}$. We also know that exponential functions are concave up. That is, given any $s,t\in\R, \lambda\in\left[ 0,1 \right]$, we have
        \begin{equation}
            e^{\lambda s+\left( 1-\lambda t \right)t} \leq \lambda e^{s} + \left( 1-\lambda \right)e^{t}.
        \end{equation}
        Now by letting $\lambda=\frac{1}{p}$, we have $1-\lambda=\frac{1}{q}$ freely by definition of $p,q$. Then by writing $s=\alpha p, t=\beta q$, we see that [4.4] is precisely of the form [4.5], as needed.
    \end{proof}
    
    \begin{lemma}{}
        Let $p,q\in\left( 1,\infty \right)$ with $\frac{1}{p}+\frac{1}{q}=1$ and let $f\in\mL^p\left( \mu \right),g\in\mL^q\left( \mu \right)$. Then $fg\in\lone\left( \mu \right)$, with
        \begin{equation}
            \left\lVert fg \right\rVert_{1} \leq \frac{1}{p}\left\lVert f\right\rVert^p_p + \frac{1}{q}\left\lVert g\right\rVert^q_q.
        \end{equation}
    \end{lemma}
    
    \begin{proof}
        To get [4.6] from [4.3], we take $a=\left| f\left( x \right) \right|,b=\left| g\left( x \right) \right|$ so that
        \begin{equation}
            \left| f\left( x \right)g\left( x \right) \right|\leq \frac{1}{p}\left| f\left( x \right) \right|^p+\frac{1}{q}\left| g\left( x \right) \right|^q
        \end{equation}
        for all $x\in X$. By integrating both sides of [4.7], we obtain [4.6].
    \end{proof}
    
    \clearpage

    \begin{lemma}{}
        Let $p,q\in\left( 1,\infty \right)$ with $\frac{1}{p}+\frac{1}{q}=1$ and let $f\in\mL^p\left( \mu \right),g\in\mL^q\left( \mu \right)$ with $\left\lVert f\right\rVert_p = \left\lVert g\right\rVert_q = 1$. Then $fg\in\lone\left( \mu \right)$ with
        \begin{equation}
            \left\lVert fg \right\rVert_{1}\leq 1.
        \end{equation}
    \end{lemma}
    
    \begin{proof}
        This is a special case of Lemma 4.5, as we have
        \begin{equation*}
            \left\lVert fg \right\rVert_{1} \leq \frac{1}{p}\left\lVert f\right\rVert^p_p + \frac{1}{q}\left\lVert g\right\rVert^q_q = \frac{1}{p} + \frac{1}{q} = 1.
        \end{equation*}
    \end{proof}
    
    \np Note that Lemma 4.6 is a special case of Holder's inequality: it says
    \begin{equation}
        \left\lVert fg \right\rVert_{1} \leq 1 = \left\lVert f\right\rVert_p\left\lVert g\right\rVert_q.
    \end{equation}
    The general statement of Holder's inequality will follow from [4.9]. The intuition is to \textit{rescale} $f,g$ to have norm $1$, apply the special case [4.9], and then get back Holder's inequality.
    
    \np The following is the conclusion of this subsection.

    \begin{prop}{}
        For every $p\in\left[ 1,\infty \right)$, $\left\lVert \cdot\right\rVert_p$ is a seminorm on $\mL^p\left( \mu \right)$.
    \end{prop}

    \placeqed[See Minkowski's Inequality]

    \subsection{Null-space of a Seminorm}
    
    
    
    
    
    
    
    
    
    
    
    
    
    
    
    
    
    
    
    
    
    
    
    
    
    
    
    
    
    
    
    
    
    
    
    
    
    
    
    
    
    
    
    

\end{document}
